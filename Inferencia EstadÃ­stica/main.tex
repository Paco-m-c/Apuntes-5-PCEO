\documentclass[openany]{book}
\usepackage[utf8]{inputenc}
\usepackage{verbatim}
\usepackage[hypertexnames=false]{hyperref}
\usepackage{amstext} 
\usepackage{array}   
\newcolumntype{C}{>{$}c<{$}} 


%%%%%%%%%%%%%%%%%%%%%%%

%%%%%%%%%%%%%%%%%%%%%%%
% HOLA PACO
% ESTE ES EL ARCHIVO DE LAS DEFINICIONES ESTRUCTURALES
% VERSION 1.1 NOMÁS
%
% AUTOR ORIGINAL:
% EDUARDO (CHITO) BELMONTE GUILLAMÓN
%
% ESTE ARCHIVO ES COMUNISTA, PUEDES COMPARTIRLO SI QUIERES
%%%%%%%%%%%%%%%%%%%%%%%

%----------------------------------
%     PAQUETICOS QUE SE USAN
%----------------------------------

%--------------------------
%    PARA USAR INKSCAPE
%---------------------------
\usepackage{import}
\usepackage{hyperref}
\usepackage{xifthen}
\usepackage{pdfpages}
\usepackage{transparent}

\newcommand{\incfig}[1]{%
    \def\svgwidth{\columnwidth}
    \import{./figures/}{#1.pdf_tex}
}

\newcommand{\custincfig}[2]{%
    \def\svgwidth{#1}
    \import{./figures/}{#2.pdf_tex}
}
\newcommand{\textnexttofig}[3]{
  \begin{minipage}[l]{0.45\textwidth}
    \custincfig{#1}{#2}
  \end{minipage}
  \begin{minipage}[l]{0.45\textwidth}
    #3
  \end{minipage}
}

%%%%%%%%% FIN DEL INKSCAPE

\usepackage{parskip} % Pa parrafos wapos
\setlength{\parindent}{0.5cm} % Pa la sangría
\usepackage{graphicx} % Pa meter las imágenes
\graphicspath{{Images/}} % La ruta a las imágenes

\usepackage{tikz} % Pa dibujar cosichuelas guapas

\usepackage[spanish]{babel} % PA QUE ESTÉ EN ESPAÑOL NOMÁS

\usepackage{enumitem} % Para personalizar las LISTAS YEAH

\setlist{nolistsep} % Pa que las listas estén junticas

\usepackage{booktabs} % Esta sirve para hacer tablas fancy con multicolumns y tal pero no tengo ni puta idea de usarla

\usepackage{xcolor} % PA DEFINIR LOS COLORINES
\definecolor{turquoise}{RGB}{21,103,112} % Es un turquesica así formal
\definecolor{violet}{RGB}{ 110, 6, 187 } % Color maricón

%-------------------------------------------------
%     MÁRGENES
%-------------------------------------------------

\usepackage{geometry}
\geometry{
    top=3cm,
    bottom=3cm,
    left=3cm,
    right=3cm,
    headheight=14pt,
	footskip=1.4cm,
	headsep=10pt,
}

\usepackage{avant} % Esto es una fuente para encabezados

%\usepackage{mathptmx} % Usar simbolitos matemáticos chulos

\usepackage{microtype} % Para fuentes de maricones

\usepackage[utf8]{inputenc} % Pa los acentos

\usepackage[T1]{fontenc}

%-------------------------------------------------
% Bibliografía e índice
%-------------------------------------------------

\usepackage{makeidx} % Pa hacer un índice
\makeindex

\usepackage{titletoc}   % Para manipular la tabla de contenidos

\contentsmargin{0cm}    % Para eliminar el margen por defecto

\usepackage{titlesec} % Pa cambiar los titulos skere

\titleformat
{\chapter} % command
[display] % shape
{\centering\bfseries\Huge\normalfont} % format
{\color{turquoise}  {\normalsize\MakeUppercase{Capítulo} \thechapter }} % label
{-0.5cm} % sep
{
    \color{turquoise}
    \rule{\textwidth}{3pt}
    \vspace{1ex}
    \centering
    \setcounter{ex}{0}
    \setcounter{dummy}{0}
} % before-code
[
\vspace{-0.5cm}%
\rule{\textwidth}{3pt}
] % after-code


\titleformat{\part}
[display]
{\centering\bfseries\Huge\normalfont}
{\color{turquoise} {\normalsize \MakeUppercase{Asignatura}}}
{0pt}
{\color{turquoise}
\vspace{-0.6cm}
\rule{\textwidth}{3pt}
\vspace{1ex}
\setcounter{chapter}{0}
\setcounter{section}{0}
\setcounter{dummy}{0}
\centering
}


\titleformat{\section}
{\normalfont\Large\bfseries}{\color{turquoise}\thesection\ - }{0.5em}{}

\usepackage{fancyhdr}   % Necesario para el encabezado y el pie de página

\pagestyle{fancy}   %Para modificar los encabezados
\fancyhf{}          %Para eliminar los encabezados y pies de página por defecto.
\fancyhead[LE,RO]{\sffamily\normalsize\thepage}
\fancyfoot[C]{Ampliación de Probabilidad}
%HACER

\usepackage{amsmath,amsfonts,amssymb,amsthm,cancel} % PARA LAS MATES

%   LINEA 199, HACER CAPULLADAS

\newtheoremstyle{turquoisebox}
{0pt} %Espacio encima
{0pt} %Espacio abajo
{\normalfont} % Fuente del cuerpo
{} % Cantidad de identado
{\small\ssfamily\color{turquoise}} % Fuente en la que pone "TEOREMA"
{:} % Puntuación tras el teorema
{0.25em} %Espacio tras el teorema
{\thmname{#1}\thmnumber{#2}} %No sé si esto funciona


\newcounter{dummy}[subsection]
\newcounter{ex}
\newtheorem{teoremote}[dummy]{\color{turquoise}Teorema}
\newtheorem{propositiont}[dummy]{\color{turquoise}Proposición}
\newtheorem{lemmat}[dummy]{\color{turquoise}Lema}
\newtheorem{definitionT}{\color{turquoise}Definición}[section]
\newtheorem{exerciseT}[ex]{Ejercicio}
\newtheorem{examplote}[ex]{\color{turquoise}Ejemplo}
\newtheorem{methodT}[dummy]{\color{turquoise}Método}


\RequirePackage[framemethod=default]{mdframed} % Required for creating the theorem, definition, exercise and corollary boxes

%Caja de teoremas

\newmdenv[skipabove=7pt,
skipbelow=7pt,
backgroundcolor=black!5,
linecolor=turquoise,
innerleftmargin=5pt,
innerrightmargin=5pt,
innertopmargin=5pt,
leftmargin=0cm,
rightmargin=0cm,
linewidth=3pt,
innerbottommargin=5pt]{tBox}

\newmdenv[skipabove=7pt,
skipbelow=7pt,
backgroundcolor=black!5,
linecolor=turquoise,
innerleftmargin=5pt,
innerrightmargin=5pt,
innertopmargin=5pt,
leftmargin=0cm,
rightmargin=0cm,
linewidth=1pt,
innerbottommargin=5pt]{pBox}

\newmdenv[skipabove=7pt,
skipbelow=7pt,
backgroundcolor=violet!7,
linecolor=turquoise,
innerleftmargin=5pt,
innerrightmargin=5pt,
innertopmargin=5pt,
leftmargin=0cm,
rightmargin=0cm,
rightline=false,
topline=false,
bottomline=false,
linewidth=4pt,
innerbottommargin=5pt]{mBox}

\newmdenv[skipabove=7pt,
skipbelow=7pt,
rightline=false,
leftline=true,
topline=false,
bottomline=false,
linecolor=turquoise,
innerleftmargin=5pt,
innerrightmargin=5pt,
innertopmargin=0pt,
leftmargin=0cm,
rightmargin=0cm,
linewidth=4pt,
innerbottommargin=0pt]{dBox}

\newmdenv[skipabove=7pt,
skipbelow=7pt,
rightline=false,
leftline=true,
topline=false,
bottomline=false,
backgroundcolor=black!3,
linecolor=turquoise!50,
innerleftmargin=5pt,
innerrightmargin=5pt,
innertopmargin=0pt,
innerbottommargin=5pt,
leftmargin=0cm,
rightmargin=0cm,
linewidth=4pt]{eBox}

\newmdenv[skipabove=7pt,
skipbelow=7pt,
leftline=true,
topline=false,
rightline=false,
bottomline=false,
backgroundcolor=cyan!5,
linecolor=turquoise,
innerleftmargin=5pt,
innerrightmargin=5pt,
innertopmargin=0pt,
innerbottommargin=5pt,
leftmargin=0cm,
rightmargin=0cm,
linewidth=4pt]{exBox}

\newenvironment{theorem}{\begin{tBox}\begin{teoremote}}{\end{teoremote}\end{tBox}}
\newenvironment{proposition}{\begin{pBox}\begin{propositiont}}{\end{propositiont}\end{pBox}}
\newenvironment{lemma}{\begin{pBox}\begin{lemmat}}{\end{lemmat}\end{pBox}}
\newenvironment{method}{\begin{mBox}\begin{methodT}}{\end{methodT}\end{mBox}}
\newenvironment{definition}{\begin{dBox}\begin{definitionT}}{\end{definitionT}\end{dBox}}
\newenvironment{exercise}{\begin{eBox}\begin{exerciseT}}{\hfill{\color{black}}\end{exerciseT}\end{eBox}}
\newenvironment{example}{\begin{exBox}\begin{examplote}}{\end{examplote}\end{exBox}}
\newenvironment{demonstration}{\begin{flushright}
      \color{turquoise} \textbf{Demostración}
\end{flushright}
}{\begin{flushright}
  $\square$
\end{flushright}}

\usepackage{geometry}
\geometry{
    top=3cm,
    bottom=3cm,
    left=3cm,
    right=3cm,
    headheight=14pt, 
    footskip=1.4cm,
    headsep=10pt,
}
\usepackage{graphicx}
\title{Apuntes de Inferencia Estadística}
\author{Paco Mora}
\date{\today}

\begin{document}

Mirar los apuntes de Lorencio

% \maketitle

% \chapter{Tema 2}

% \begin{exercise}
%     \textbf{Ejercicio 1.a) y 1.b)}

%     Los valores que puede tomar el vector son
%     $$ \{(0,0,0),(1,0,0),(0,1,0),(0,0,1),(1,1,0),(1,0,1),(0,1,1),(1,1,1)\} $$

%     Donde tenemos que:
%     $$ P(X_1=0,X_2=0,X_3=0) = P(X=0)^3 = (1-p)^3 $$
%     $$ P(X_1=1,X_2=0,X_3=0) = p(1-p)^2 = P(X_1=0,X_2=1,X_3=0) = P(X_1=0,X_2=0,X_3=1)$$
%     $$ P(X_1=1,X_2=1,X_3=0) = p^2(1-p) = P(X_1=1,X_2=0,X_3=1) = P(X_1=0,X_2=1,X_3=1) $$
%     $$ P(X_1=1,X_2=1,X_3=1) = p^3 $$

%     Es fácil comprobar que la suma de todas las probabilidades es 1.

%     Obtendremos la media muestral:
%     $$ \overline{X} = \dfrac{X_1+X_2+X_3}{n} $$ 

%     Que tomará los valores $ \left\{0,\dfrac{1}{3},\dfrac{2}{3},1\right\} $, calculamos sus probabilidades:
%     $$ P(\overline{X} = 0) = P(X_1=0,X_2=0,X_3=0) = (1-p)^3 $$

%     De forma análoga sacamos
%     $$ P\left(\overline{X}=\dfrac{1}{3}\right) = 3p(1-p)^2 $$
%     $$ P\left(\overline{X}=\dfrac{2}{3}\right) = 3p^2(1-p) $$
%     $$ P\left(\overline{X}=1\right) = p^3 $$


% \end{exercise}

% \begin{exercise}
%     $  $

%     Dada una m.a.s $ (X_1,X_2,...,X_n)  $ de $ X \sim Exp(d) $, obtener la distribución en el muestreo del estadístico $ S = \sum\limits_{j=1}^{n}X_j $. Intentamos ver si Exp es reproductiva:
%     $$ \phi_{S}(t) = \phi_{X}(t)^{n} = \left( \left( 1-\dfrac{it}{\alpha}    \right)^{-1} \right)^{n} = \left( 1-\dfrac{it}{\alpha} \right)^{-n} $$

%     No lo es, pero si lo comparamos con la distribución Gamma, obtenemos que $ Exp(\alpha) \equiv \gamma(a = \alpha, p=1) $

%     Vamos a ver si la distribución Gamma es reproductiva respecto a algún parámetro.

%     $$ \phi_{S}(t) = \left( 1-\dfrac{it}{a} \right)^{-np} \implies S \sim \gamma(a,np) $$

%     Luego Gamma es reproductiva, entonces tenemos que el estadístico es:
%     $$ S = \sum\limits_{j=1}^{n}X_j \sim \gamma(\alpha, n) $$
% \end{exercise}


% \begin{exercise}
%     $  $

%     $$ \phi_{X}(t) = e^{it\mu-t^2\dfrac{\sigma^2}{2}} $$

%     $$ S = \sum\limits_{j=1}^{n}X_j \implies \phi_{S}(t) = \prod_{j=1}^{n}\phi_{X_j}(t) = \phi_{X}(t)^{n} = \left( e^{it\mu-t^2\dfrac{\sigma^2}{2}} \right)^{n} = e^{it\mu n-t^2 \dfrac{\sigma^2n}{2}} $$

%     Luego tenemos $ S \sim N(n\mu,n\sigma^2) $

%     Si queremos sacar el estadístico $ \overline{X}=\dfrac{S}{n} $ haremos:
%     $$ \phi_{\overline{X}}(t) = E(e^{it\overline{X}}) = E(e^{itS/n}) = \phi_{S}\left(\dfrac{t}{n}\right) = e^{it\mu-\dfrac{t^2\sigma^2n}{2n^2}}  $$
%     $$ \overline{X}\sim N\left(t,\dfrac{\sigma^2}{n}\right) $$
% \end{exercise}

% \begin{exercise}
%     $  $

%     Obtenemos primero la distribución de $ Y = X^2 $ utilizando cambio de variable:
%     $$ x(y) = \sqrt{y}\hspace{5mm}x'(y) = \dfrac{1}{2}y^{-1/2} $$
%     $$ f_{Y}(y) = \cancel{2y^{1/2}}e^{-y/\theta}\cancel{\dfrac{1}{2}y^{-1/2}}I_{(0,+\infty)}(y)=\dfrac{1}{\theta}e^{-y/\theta}I_{(0,+\infty)} \implies$$
%     $$ \implies Y \sim Exp\left(\dfrac{1}{\theta}\right) \equiv \gamma\left(\dfrac{1}{\theta},1\right) $$

%     Para ver la media y la varianza basta consultarlo en la hoja para la distribución Gamma.
% \end{exercise}

% \begin{proposition}
%     Para la media muestral tenemos:
%     $$ E(\overline{X}) = E(X) $$
% \end{proposition}
% \begin{demonstration}
%  $$   E(\overline{X}) = E(\sum\limits_{j=1}^{n}\dfrac{X_j}{n}) = \dfrac{1}{n}\sum\limits_{j=1}^{n}E(X_j) = \dfrac{1}{n}nE(X) = E(X)$$
% \end{demonstration}

% \begin{proposition}
%     Para la media muestral tenemos:
%     $$ Var(\overline{X}) = \dfrac{Var(X)}{n} $$
% \end{proposition}

% \begin{demonstration}
%     $$ Var(\overline{X}) = Var\left(\sum\limits_{j=1}^{n}X_j \right) = \dfrac{1}{n^2}\sum\limits_{j=1}^{n}Var(X_j) = \dfrac{1}{n^2}nVar(X) = \dfrac{Var(X)}{n} $$
% \end{demonstration}

% \begin{proposition}
%     $$ \overline{X} \xrightarrow{P}E(X) $$

% \end{proposition}
% \begin{demonstration}
%     Recordemos que decimos que una sucesión converge en probabilidad si se cumple una de las siguientes condiciones equivalentes:
%     $$ \forall \varepsilon >0, \lim_{n \to \infty} P(|X_n-X|\leq  \varepsilon) = 1 $$
%     $$ \forall \varepsilon >0, \lim_{n \to \infty} P(|X_n-X|\geq  \varepsilon) = 0 $$

%     Usaremos también la desigualdad de Tchebychev:
%     $$ \forall  \varepsilon>0, P(|\overline{X}-E(\overline{X})|\geq \varepsilon) \leq  \dfrac{Var(\overline{X})}{\varepsilon^2} $$
    
%     Aplicándolo a nuestro caso:
%     $$ \forall  \varepsilon>0, P(|\overline{X}-E(X)|\geq \varepsilon) \leq  \dfrac{Var(X)}{n\varepsilon^2} \to 0 $$


% \end{demonstration}

% \begin{proposition}
%     $$ \dfrac{\overline{X}-E(X)}{\sqrt{\dfrac{Var(X)}{n}}}\to_{d}N(0,1) $$
% \end{proposition}

% \begin{demonstration}
%     Basta usar el teorema central del límite que repasamos en el Tema 1.
% \end{demonstration}

% \begin{proposition}
%     \textbf{Propiedades de la proporción muestral}

%     \begin{enumerate}
%         \item $ \sum\limits_{j=1}^{n}X_j = n\widetilde{p} \sim B(n,p) $
%         \item $ E(\widetilde{p}) = p $
%         \item $ Var(\widetilde{p}) = \dfrac{p(1-p)}{n} $
%         \item $ \widetilde{p}\to_{p}p $
%         \item $ \dfrac{\widetilde{p}-p}{\sqrt{\dfrac{p(1-p)}{n}}} \to_{d}N(0,1) $

%     \end{enumerate}

% \end{proposition}
% \begin{demonstration}
%     Todas se derivan de la Observación 1 del guion.
% \end{demonstration}

% \textbf{Observación sobre la funcion de distribución empírica}

% Si $ A = \{X_j \leq  n\} $
% $$ P(A) = P(X_j\leq n ) = P(X\leq n) = F(n) $$

% Con esta observación demostramos las siguientes propiedades:


\section*{Teorema de Wald}

 \subsection*{Consistencia}
 Sea $ X \sim F(\cdot ,\theta),\ \theta \in \Theta, \hat{\theta}_n $ es estimador de $ \theta $, definimos las siguientes operaciones:
 $$ P_{\theta_0}(\hat{\theta}_n \in A ) := P(\hat{\theta}_n\in A| \theta = \theta_0) $$
$$ E[\hat{\theta}_n] = \idotsint\limits_{\psi}^{}\hat{\theta}_n(x)L(x,\theta_0)dx $$

Se dice que $ \hat{\theta}_n $ es consistente para $ \theta \in \Theta $ si $ \hat{\theta}_n\xrightarrow{P_{\theta_0}} \theta_0\ \text{(convergencia en probabilidad)},\ \forall \theta_0 \in \Theta $

\begin{flushright}
    \textbf{Observación}
\end{flushright}

Si $ \Theta $ es finito y $ \hat{\theta}_n $ es estimador de $ \theta $ entonces se da la consistencia del estadístico si y solo si:
$$ \lim_{n \to \infty}P_{\theta_0}(\hat{\theta}_n=\theta_0)=1,\ \forall \theta_0 \in \Theta $$

\begin{theorem}
    Sea $ X $ variable aleatoria con función de distribución $ F(\cdot ,\theta) $ para $ \theta \in \Theta $, siendo $ \Theta $ un conjunto infinito. Supongamos que se verifica:
    \begin{enumerate}
        \item (A1) El soporte de $ F(\cdot ,\theta) $
    \end{enumerate}
\end{theorem}
\end{document}