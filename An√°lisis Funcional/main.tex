\documentclass[openany]{book}
\usepackage[utf8]{inputenc}
\usepackage{verbatim}
\usepackage[hypertexnames=false]{hyperref}
\usepackage{amstext} 
\usepackage{array}   
\newcolumntype{C}{>{$}c<{$}} 


%%%%%%%%%%%%%%%%%%%%%%%

%%%%%%%%%%%%%%%%%%%%%%%
% HOLA PACO
% ESTE ES EL ARCHIVO DE LAS DEFINICIONES ESTRUCTURALES
% VERSION 1.1 NOMÁS
%
% AUTOR ORIGINAL:
% EDUARDO (CHITO) BELMONTE GUILLAMÓN
%
% ESTE ARCHIVO ES COMUNISTA, PUEDES COMPARTIRLO SI QUIERES
%%%%%%%%%%%%%%%%%%%%%%%

%----------------------------------
%     PAQUETICOS QUE SE USAN
%----------------------------------

%--------------------------
%    PARA USAR INKSCAPE
%---------------------------
\usepackage{import}
\usepackage{hyperref}
\usepackage{xifthen}
\usepackage{pdfpages}
\usepackage{transparent}

\newcommand{\incfig}[1]{%
    \def\svgwidth{\columnwidth}
    \import{./figures/}{#1.pdf_tex}
}

\newcommand{\custincfig}[2]{%
    \def\svgwidth{#1}
    \import{./figures/}{#2.pdf_tex}
}
\newcommand{\textnexttofig}[3]{
  \begin{minipage}[l]{0.45\textwidth}
    \custincfig{#1}{#2}
  \end{minipage}
  \begin{minipage}[l]{0.45\textwidth}
    #3
  \end{minipage}
}

%%%%%%%%% FIN DEL INKSCAPE

\usepackage{parskip} % Pa parrafos wapos
\setlength{\parindent}{0.5cm} % Pa la sangría
\usepackage{graphicx} % Pa meter las imágenes
\graphicspath{{Images/}} % La ruta a las imágenes

\usepackage{tikz} % Pa dibujar cosichuelas guapas

\usepackage[spanish]{babel} % PA QUE ESTÉ EN ESPAÑOL NOMÁS

\usepackage{enumitem} % Para personalizar las LISTAS YEAH

\setlist{nolistsep} % Pa que las listas estén junticas

\usepackage{booktabs} % Esta sirve para hacer tablas fancy con multicolumns y tal pero no tengo ni puta idea de usarla

\usepackage{xcolor} % PA DEFINIR LOS COLORINES
\definecolor{turquoise}{RGB}{21,103,112} % Es un turquesica así formal
\definecolor{violet}{RGB}{ 110, 6, 187 } % Color maricón

%-------------------------------------------------
%     MÁRGENES
%-------------------------------------------------

\usepackage{geometry}
\geometry{
    top=3cm,
    bottom=3cm,
    left=3cm,
    right=3cm,
    headheight=14pt,
	footskip=1.4cm,
	headsep=10pt,
}

\usepackage{avant} % Esto es una fuente para encabezados

%\usepackage{mathptmx} % Usar simbolitos matemáticos chulos

\usepackage{microtype} % Para fuentes de maricones

\usepackage[utf8]{inputenc} % Pa los acentos

\usepackage[T1]{fontenc}

%-------------------------------------------------
% Bibliografía e índice
%-------------------------------------------------

\usepackage{makeidx} % Pa hacer un índice
\makeindex

\usepackage{titletoc}   % Para manipular la tabla de contenidos

\contentsmargin{0cm}    % Para eliminar el margen por defecto

\usepackage{titlesec} % Pa cambiar los titulos skere

\titleformat
{\chapter} % command
[display] % shape
{\centering\bfseries\Huge\normalfont} % format
{\color{turquoise}  {\normalsize\MakeUppercase{Capítulo} \thechapter }} % label
{-0.5cm} % sep
{
    \color{turquoise}
    \rule{\textwidth}{3pt}
    \vspace{1ex}
    \centering
    \setcounter{ex}{0}
    \setcounter{dummy}{0}
} % before-code
[
\vspace{-0.5cm}%
\rule{\textwidth}{3pt}
] % after-code


\titleformat{\part}
[display]
{\centering\bfseries\Huge\normalfont}
{\color{turquoise} {\normalsize \MakeUppercase{Asignatura}}}
{0pt}
{\color{turquoise}
\vspace{-0.6cm}
\rule{\textwidth}{3pt}
\vspace{1ex}
\setcounter{chapter}{0}
\setcounter{section}{0}
\setcounter{dummy}{0}
\centering
}


\titleformat{\section}
{\normalfont\Large\bfseries}{\color{turquoise}\thesection\ - }{0.5em}{}

\usepackage{fancyhdr}   % Necesario para el encabezado y el pie de página

\pagestyle{fancy}   %Para modificar los encabezados
\fancyhf{}          %Para eliminar los encabezados y pies de página por defecto.
\fancyhead[LE,RO]{\sffamily\normalsize\thepage}
\fancyfoot[C]{Ampliación de Probabilidad}
%HACER

\usepackage{amsmath,amsfonts,amssymb,amsthm,cancel} % PARA LAS MATES

%   LINEA 199, HACER CAPULLADAS

\newtheoremstyle{turquoisebox}
{0pt} %Espacio encima
{0pt} %Espacio abajo
{\normalfont} % Fuente del cuerpo
{} % Cantidad de identado
{\small\ssfamily\color{turquoise}} % Fuente en la que pone "TEOREMA"
{:} % Puntuación tras el teorema
{0.25em} %Espacio tras el teorema
{\thmname{#1}\thmnumber{#2}} %No sé si esto funciona


\newcounter{dummy}[subsection]
\newcounter{ex}
\newtheorem{teoremote}[dummy]{\color{turquoise}Teorema}
\newtheorem{propositiont}[dummy]{\color{turquoise}Proposición}
\newtheorem{lemmat}[dummy]{\color{turquoise}Lema}
\newtheorem{definitionT}{\color{turquoise}Definición}[section]
\newtheorem{exerciseT}[ex]{Ejercicio}
\newtheorem{examplote}[ex]{\color{turquoise}Ejemplo}
\newtheorem{methodT}[dummy]{\color{turquoise}Método}


\RequirePackage[framemethod=default]{mdframed} % Required for creating the theorem, definition, exercise and corollary boxes

%Caja de teoremas

\newmdenv[skipabove=7pt,
skipbelow=7pt,
backgroundcolor=black!5,
linecolor=turquoise,
innerleftmargin=5pt,
innerrightmargin=5pt,
innertopmargin=5pt,
leftmargin=0cm,
rightmargin=0cm,
linewidth=3pt,
innerbottommargin=5pt]{tBox}

\newmdenv[skipabove=7pt,
skipbelow=7pt,
backgroundcolor=black!5,
linecolor=turquoise,
innerleftmargin=5pt,
innerrightmargin=5pt,
innertopmargin=5pt,
leftmargin=0cm,
rightmargin=0cm,
linewidth=1pt,
innerbottommargin=5pt]{pBox}

\newmdenv[skipabove=7pt,
skipbelow=7pt,
backgroundcolor=violet!7,
linecolor=turquoise,
innerleftmargin=5pt,
innerrightmargin=5pt,
innertopmargin=5pt,
leftmargin=0cm,
rightmargin=0cm,
rightline=false,
topline=false,
bottomline=false,
linewidth=4pt,
innerbottommargin=5pt]{mBox}

\newmdenv[skipabove=7pt,
skipbelow=7pt,
rightline=false,
leftline=true,
topline=false,
bottomline=false,
linecolor=turquoise,
innerleftmargin=5pt,
innerrightmargin=5pt,
innertopmargin=0pt,
leftmargin=0cm,
rightmargin=0cm,
linewidth=4pt,
innerbottommargin=0pt]{dBox}

\newmdenv[skipabove=7pt,
skipbelow=7pt,
rightline=false,
leftline=true,
topline=false,
bottomline=false,
backgroundcolor=black!3,
linecolor=turquoise!50,
innerleftmargin=5pt,
innerrightmargin=5pt,
innertopmargin=0pt,
innerbottommargin=5pt,
leftmargin=0cm,
rightmargin=0cm,
linewidth=4pt]{eBox}

\newmdenv[skipabove=7pt,
skipbelow=7pt,
leftline=true,
topline=false,
rightline=false,
bottomline=false,
backgroundcolor=cyan!5,
linecolor=turquoise,
innerleftmargin=5pt,
innerrightmargin=5pt,
innertopmargin=0pt,
innerbottommargin=5pt,
leftmargin=0cm,
rightmargin=0cm,
linewidth=4pt]{exBox}

\newenvironment{theorem}{\begin{tBox}\begin{teoremote}}{\end{teoremote}\end{tBox}}
\newenvironment{proposition}{\begin{pBox}\begin{propositiont}}{\end{propositiont}\end{pBox}}
\newenvironment{lemma}{\begin{pBox}\begin{lemmat}}{\end{lemmat}\end{pBox}}
\newenvironment{method}{\begin{mBox}\begin{methodT}}{\end{methodT}\end{mBox}}
\newenvironment{definition}{\begin{dBox}\begin{definitionT}}{\end{definitionT}\end{dBox}}
\newenvironment{exercise}{\begin{eBox}\begin{exerciseT}}{\hfill{\color{black}}\end{exerciseT}\end{eBox}}
\newenvironment{example}{\begin{exBox}\begin{examplote}}{\end{examplote}\end{exBox}}
\newenvironment{demonstration}{\begin{flushright}
      \color{turquoise} \textbf{Demostración}
\end{flushright}
}{\begin{flushright}
  $\square$
\end{flushright}}

\usepackage{geometry}
\geometry{
    top=3cm,
    bottom=3cm,
    left=3cm,
    right=3cm,
    headheight=14pt, 
    footskip=1.4cm,
    headsep=10pt,
}
\usepackage{graphicx}
\title{Delirios de AnalFun}
\author{Paco Mora}
\date{\today}

\begin{document}

\maketitle

\chapter{Yo qué sé qué es esto}

\section{Introducción}

\begin{definition}
    
    Un espacio de medida nula de primera categoría cuando está contenido en una unión numerable de cerrados con interior vacío. Si no es de primera categoría se llama de segunda categoría.
\end{definition}


\begin{theorem}
    \textbf{(Baire)}    

    Sea $ (X,d) $ espacio métrico completo $ \{G_n\}_{n \in \mathbb{N}} $ abiertos de en $ X $, $ \overline{G}_{r} = X\ \forall n \in N $. Entonces:
    $$ \bigcap_{n=1}^{\infty}G_n \ne \emptyset $$
\end{theorem}

\textit{\textbf{//Repaso de la relación de orden}}

\begin{theorem}
    \textbf{Principio de la buena ordenación }
    Para todo conjunto $ \mathcal{S}   $, existe una relación de orden $ \leq  $ tal que $ (S, \leq ) $ está bien ordenado, $ \leq  $ es un buen orden.
\end{theorem}


\begin{theorem}
    \textbf{Lema de Zorn}

    Si $ (P,\leq ) $ es un conjunto parcialmente ordenado en el que cada cadena tiene una cota superior (para $ C $, cadena, existe $ c \in P $ tal que $ x\leq c $ para todo $ x \in C $), entonces $ P $ tiene un elemento maximal (existe $  m \in P $ tal que si $ \leq x $ entonces $ x = m $)
\end{theorem}

\begin{theorem}
    \textbf{Principio Maximal de Hasudorff}

    Cada conjunto parcialmente ordenado ($ P,\leq  $) contiene una cadena maximal.

\end{theorem}

\newpage
\begin{theorem}
    Son equivalentes:
    \begin{enumerate}
        \item El principio Maximal de Hasudorff
        \item Lema de Zorn
        \item Principio de la buena ordenación
        \item Axioma de elección
    \end{enumerate}
\end{theorem}

//Definiciones de espacio de Hilbert y de Banach


//1.2.8 del libro

//Del 1.3 ha dicho que lo leamos.

//"Los teoremas que pregunto son los que tienen nombre"


\begin{theorem}
    \textbf{De la mejor aproximación}

    Dado $ (H,<\cdot >) $ espacio de Hilbert y $ C \subset H $ cerrado y convexo. Sea $ x_0 \not  \in C $. Entonces existe un único elemento $ c_0 \in C $ tal que $ \|x_0-x\| = \inf \{\|x_0-c\|\ = \alpha :\ c \in C\} $

\end{theorem}

\begin{demonstration}
    Tomemos una sucesión $(c_n)_{n \in \mathbb{N}}$ con $ c_n \in C $ de forma que se verifique 

    %imagen

    \begin{tikzpicture}
        \draw[black, very thick] (0,0) rectangle (4,0);
        \node[circle,inner sep=1pt,fill=red,label={$\alpha$}] at (1,0) {};
        \node[circle,inner sep=1pt,fill=red,label={$\|c_n\|$}] at (2,0) {};
        \node[circle,inner sep=1pt,fill=red,label={$\alpha +\dfrac{1}{n}$}] at (3,0) {};
    \end{tikzpicture}

    Si $ c_n $ fuera de Cauchy, existe $ c_0 = \lim_{n \to \infty} c_n $. Probemos que $ (c_n) $ es de Cauchy. Para ello basta usar la identidad del paralelogramo.

    Como $ \underbrace{2\|c_n\|^2}_{2 \alpha ^2}+ \underbrace{2\|c_m\|^2}_{2 \alpha ^2}-\|c_n+c_m\|^2 = \|c_n-c_m\|^2 $

    Dividimos la expresión por 4 podemos usar la convexidad de $ C $ para el punto medio entre $ c_n $ y $ c_m $:
    $$ \dfrac{1}{2}\|c_n\|^2 + \dfrac{1}{2}\|c_m\|^2 - \left\|\dfrac{c_n+c_m}{2}\right\| ^2= \dfrac{1}{4}\|c_n-c_m\|^2 $$

    Ahora tomamos límites para ver que $ \|c_n-c_m\| \to 0 $.

\end{demonstration}

\begin{theorem}
    \textbf{(de la proyección)}    

    Sea $ M  $ un subespacio cerrado del Hilbert $ H $, entonces existen un único par de aplicaciones lineales continuas $ P,Q: H \to H $ tales que $ P(H) = M $ y $ Q(H) = M^{\perp } = \{y \in H:\ <y,m> = 0\ \forall  m \in M\} $ y $ x = Px+Qx\ \forall x \in H $

    Además se verifica:
    \begin{itemize}
        \item $ x \in M \implies Px =x,\ Qx = 0;\ x \in M ^{\perp} \implies Px = 0,\ Qx = x $
        \item $ \|x-Px\| = \inf \{\|x-y\|,\ y \in M\}\ \forall x \in H $
        \item $ \|x\|^2 = \|Px\|^2+\|Qx\|^2 $ (Pitágoras)
    \end{itemize}

    Como consecuencia $ H = M \oplus M^{\perp} $
\end{theorem}

\begin{demonstration}

    Sea $ x \in H,\ x+M $ cerrado y convexo, llamemos $ Qx $ alúnico elemento en $ x+M $ de norma mínima y definimos $ Px = x-Qx $. Vemos que $ Qx \in M^{\perp} $, $ <z,y> = 0 \forall y \in M $. Aplicando que $ Qx \equiv Z $ tiene norma mínima en $ x+M $ tendremos:
    $$ 0\leq  \|z\|^2 = <z,z> \leq  \underbrace{\|z-\alpha y\|^2}_{ \forall \alpha \in \mathbb{R}} = <z-\alpha y,z-\alpha y> = \cancel{<z,z>} - \overline{\alpha}<z,y> - \alpha <y,z> = \alpha ^2\|y\|^2  $$
    
    Tomando ahora $ \alpha = <z,y>  $ y como se tiene que cumplir siempre que la expresión es mayor o igual que 0 llegamos a $ 0 \leq  -\alpha ^2 \implies \alpha = 0 $, luego $ \operatorname{Im}(Q) \subset H^{\perp} $. Como además $ M \cap M^{\perp} = \{0\} \implies x = Px+Qx $, entonces $ H = M \oplus M^{\perp} $

    Análogamente sale el resto de los enunciados\footnote{xd}.

\end{demonstration}

\begin{lemma}
    $ M \subset H $ subespacio estricto cerrado del espacio de Hilbert H. Entonces $ \exists x_0 \ne 0, x_0 \perp M $, $ <x_0,m\geq 0 \forall m \in M $

\end{lemma}
\begin{demonstration}

    Como $ H \ne M \implies M ^{\perp} \ne \{0\} $
    $$ \{d_n:\ n=1,2,..\} \text{numerable y denso en }H$$
    
    Tomamos entonces una base ortonormal $ \{e_1,e_2,...,e_n,...\} $ tal que:
    $$ \operatorname{span} \{d_1,...,d_n,...\} = \operatorname{span} \{e_1,...,e_n,...\} $$
\end{demonstration}
    
    \begin{definition}
        
        \textbf{Conjunto ortonormal} $ \{\overline{u}_1,\overline{u_2},...\} $ en $ H:\ <u_i,u_j> = \delta_{ij} $. Tenemos además que son LI:
        $$ 0 = \|\sum\limits_{i=1}^{n}c_i 0_{i}\|^2 = <\sum\limits_{i=1}^{n}c_i 0_{i},\sum\limits_{i=1}^{n}c_i 0_{i}> = \sum\limits_{i=1}^{n}c_i^2 \implies c_i = 0,\ i=1,2,...,n$$
    \end{definition}

    \begin{proposition}
        $ M = \operatorname{span} \{u_1,u_2,...,u_n\} \subset H,\ P_{M}(x) = \sum\limits_{i=1}^{n}<x_i,u_i>u_i$. Si $ d = dist \{x,M\} $ entonces:
        $$ \|x\|^2 - \delta ^2 = \sum\limits_{i=1}^{n}|<x,u_i>|^2 $$
    \end{proposition}

    \begin{lemma}
        Sea $ \{u_1,u_2,...,u_n,...\} $ ortonormal, $ \|x\|^2 \geq \sum\limits_{i=1}^{\infty} |<x,u_i>|^2\ \forall x \in H$
    \end{lemma}
\newpage
\begin{proposition}
    $ \{u_1,u_2,...,u_n,...\} $ ortonormal en $ H $, la función:
    $$ \Lambda: H \to \ell^2\ \Lambda(x) = \left( <x,u_i>   \right)_{i=1}^{\infty} $$
    es continua y sobre
\end{proposition}

\begin{demonstration}
    $ (\xi_n) \in \ell ^2 $ encontramos $ x \in H $: $ \Lambda(x) = (\xi_n) $. Nos preguntamos si:
    $$ \sum\limits_{i=1}^{\infty} \xi_nu_n \to <\sum\limits_{n=1}^{\infty}\xi_nu_n,u_m> $$

    No se ve nada en la pizarra, ha probado que es de Cauchy para ver que es convergente.

\end{demonstration}



\begin{theorem}
    \textbf{(de la base hilbertiana)}

    Para $ \{u_1,u_2,...,u_n,...\} $ conjunto ortonormal en $ H $ (espacio de Hilbert). Son equivalentes:
    \begin{enumerate}
        \item $ \{u_1,u_2,...\} $ es ortonormal maximal.\vspace{3mm}
        \item $ \overline{\operatorname{span} \{u_1,...\}} = H $
        \item $ \forall x \in H $ se tiene $ x = \sum\limits_{n=1}^{\infty}<x,u_n>u_n $ en $ H $
        \item $ \forall x \in H,\ \forall y \in H $, se tiene $ <x,y> = \sum\limits_{n=1}^{\infty}<x,u_n>\overline{<y,u_n>} $
        \item $ \forall x \in H $, se tiene $ \|x\|^2 = \sum\limits_{n=1}^{\infty}|<x,u_n>|^2 $

    \end{enumerate}
    A la igualdad de los dos últimos puntos se le llama Identidad de Parseval
\end{theorem}

\begin{demonstration}
    Recomiendo mirar el libro.
    $ 1 \iff 2 $

    Por la definición.\\
    $ 2 \implies 3 $

    Por la desigualdad de Bessel.

    Sea $ M_n = span \{u_1,u_2,...,u_n\} $, sabemos que $ \overline{\bigcup_{n=1}^{\infty}M_n} = H $ y que:
    $$ \forall x \in H,\ P_{M_n}(x) = \sum\limits_{i=1}^{n}<x,u_i>u_i $$
    $$ \|x\|^2 = \underbrace{dist(x,M_n)^2}_{=: \delta_n \to 0 } + \sum\limits_{i=1}^{n}|<x,u_i>|^2 $$
    $$ \forall\ \varepsilon >0,\ \exists x_{\varepsilon} \in \bigcup_{i=1}^{\infty} M_n:\ \|x-x_{\varepsilon}\| < \varepsilon,\ x_{\varepsilon} = \sum\limits_{i=1}^{n}c_iu_i \in M_{P} $$
    $$ \delta_{p} = d(x,M_{p}) \leq \|x-x_{\varepsilon}\| < \varepsilon $$
    $ 3 \implies 4 $\\

    Continuidad del producto escalar\\
    $ 4 \implies 5 $
    
    Directo.\\
    $ 5 \implies 2 $

    Por la desigualdad de Bessel.
\end{demonstration}

\begin{definition}
    A una base como la anterior se le llama \textbf{base hilbertiana}. A los coeficientes se les llama coeficientes de Fourier.
\end{definition}

\begin{lemma}
    Si $ (E,\|\cdot \|) $ es un espacio de Banach con una base algebraica numerable, entonces $ E $ es finito dimensional. 
    
    Para $ E $ no completo, no es cierto.
\end{lemma}

Aquí falta un teorema que ha dictado y no me ha dado tiempo a copiar.

% Teorema que se acaba de inventar
% Dado u producto escalar en el pro
% El coeficiente del termino de grado $n$ en $phi_{n}$ puede tomarse siempre positivo y si asi se hace la sucesión $phi_{n}$ esta unívocamente determinada.

\begin{theorem}
    Sea $ <\cdot > $ un producto escalar en $ C([a,b]) $ con $ \|\cdot \|_{\infty} $ más fina que $ \|\cdot \|_{\infty} $. Sea $ \{\phi_{n}: n = 0,1,2,\dots\}  $ la sucesión de polinomios ortonormales. Entonces:
    $$ f = \sum\limits_{n=0}^{\infty}<f,\phi_n>\phi_n\ \forall f \in C[a,b] $$

    $$ \forall \varepsilon>0\ \exists N_{\varepsilon} \implies \left\|f - \sum\limits_{n=0}^{\infty}<f,\phi_{n}>\phi_n\right\|_{<\cdot >} < \varepsilon$$
\end{theorem}


\subsection{Series de Fourier}

\begin{definition}
    Un polinomio trigonométrico es una función de la forma
    $$ h(t) = \sum\limits_{n=0}^{m} \alpha_n \cos(nt)+\beta_n \sin(nt),\ \alpha_n,\beta_n \in \mathbb{R},\ m = 0,1,2,... $$

\end{definition}

\begin{lemma}
    Si $ h_1,h_2 $ son polinomios trigonométricos, su producto también lo es.
\end{lemma}

\begin{lemma}
    $ f: [- \pi, \pi] \to \mathbb{R},\ \varepsilon>0 $, entonces existe un polinomio trigonométrico $ q_{\varepsilon} $ tal que:
    $$ \int\limits_{-\pi}^{\pi}|f(t)-q_{\varepsilon}(t)|^2 dt < \varepsilon $$
\end{lemma}

\begin{exercise}
    $$ u_0(t) = \dfrac{1}{\sqrt{2\pi}},\ u_{2n+1}(t) = \dfrac{1}{\sqrt{pi}} \cos(nt),\ u_{2m}(t) = \dfrac{1}{\sqrt{pi}} \sin(mt),\ m = 1,2,... $$
    Es ortonormal en $ (C[a,b],\langle\cdot \rangle) $
\end{exercise}


\section{Teoremas de representación}

Vemos primero un primer teorema de representación.

\begin{proposition}
    Dado $ F:C[0,1]\to \mathbb{R} $  lineal y continua. Existe una única medida ($ F $ función de distribución) tal que:
    $$ F(f) = \int\limits_{0}^{1}f(t)dF(f) $$
\end{proposition}

%teorema de alexandrov ???

%topología debil

\begin{theorem}
    \textbf{Teorema de Riesz}.

    Buscar en el libro.
\end{theorem}


\begin{definition}\textbf{Topología débil del espacio de Hilbert}

    Sea $ (H,<>) $ un espacio de Hilbert. 
    $$ 
    \begin{aligned}
        \mathbb{K} \leftarrow H : x_0, & \hspace{10mm} \varepsilon>0,\  t_1,...,t_p \in H\\ <x,x_0> \hookleftarrow x
    \end{aligned}
    $$
    $$ W(x_0,\varepsilon,t_1,...,t_p) = \{z \in H: |<t;x_0-z>|< \varepsilon,\ i=1,2,...,p\} $$
\end{definition}

\begin{theorem}
    \textbf{Alaoglo-Bourbaki}

    Sea $ (H,<>) $ un espacio de Hilbert y sea $ B_{H} = \{x \in H:\ \|x\|_{<>} \leq  1\} $. Entonces $ B_{H} $ es un subconjunto débilmente compacto.
\end{theorem}


%teoremas necesarios para esto, teorema de skolin (la topo de la puntual y puntual sobre un denso conincide)
%2 t skolin (tambien coincide sobre idk)

\begin{demonstration}

    Lo vemos para el caso separable.

    Tomemos una base hilbertiana $ \{e_n\} $ de $ H $ y tomemos $ (v_n) \subseteq  B_{H} $.
    
    Notemos primero que $ |<e_{p},v_n>| \leq  1\ \forall p,n \in \mathbb{N} $. 

    Tomemos $ [0,1]^{\mathbb{N}} $, el cubo de Hilbert, que es métrico compacto.

    Por el teorema de Riesz, tomamos la forma lineal equivalente a cada elemento de la sucesión $ (v_n) $, $ v_n \mapsto <-,v_n > $ y utilizando la base, este producto puede expresarse como $ \sum\limits_{p=1}^{\infty} <x,e_{p}>e_{p} $ para cierto $ x $.

    Volviendo al cubo de Hilbert, existe una sucesión de enteros $ n_1<n_2,...,n_{k}<... $ de forma que $ (<e_{p},v_{n_{k}})_{k=1}^{\infty} $ es convergente (ya que el cubo es métrico compacto).

    Si tomamos entonces $ S = s\operatorname{span}\{e_n:\ n \in \mathbb{N}\} $, entonces $ (<s,v_{n_{k}}>)_{k=1}^{\infty} $ es convergente $ \forall  s \in S $. Falta ver que sea convergente para todo punto de $ H = \overline{S} $ que se demuestra con los teoremas de Skald que se ven a continuación.

\end{demonstration}

\section{Teoremas de Skald}

\begin{definition}
    \textbf{Familia de funciones (uniformemente) equicontinua}

    Una sucesión de funciones continuas $ (f_{i})_{i \in I} $ se dice que es equicontinua en $ x_0 $ si $ \forall  \varepsilon > 0\ \forall i \in I\  \exists \delta_{\varepsilon}$ tal que $ d(x_0,x) < \delta_{\varepsilon} \implies |f_{i}(x)-f_{i}(x_0)| < \varepsilon $. Es decir, que el $ \delta  $ necesario es el mismo para todas las funciones.
    
    De forma análoga se define el concepto de familia de funciones uniformemente equicontinua:
    
    Una sucesión de funciones uniformemente continuas $ (f_{i})_{i \in I} $ se dice que es uniformemente equicontinua en si $ \forall  \varepsilon > 0\ \forall i \in I\  \exists \delta_{\varepsilon}$ tal que $ d(y,x) < \delta_{\varepsilon} \implies |f_{i}(x)-f_{i}(y)| < \varepsilon $. Es decir, que el $ \delta  $ necesario es el mismo para todas las funciones.
\end{definition}

Las demostraciones de los teoremas se pueden encontrar en el libro General Topology de Willard (va para tarea).

\begin{theorem}
    Sea $ (K,d) $ un espacio métrico y $ \operatorname{C}(K) = \{f: K \to \mathbb{R}\ \text{continuas}\} \hookrightarrow (\mathbb{R}^{k},T_{p}) $ ($ T_{p} $ es la topología producto). 

    Si $ \phi $ es (unif.) equicontinua, entonces $ \overline{\phi}^{T_{p}}  $ son (uniformemente) continuas
\end{theorem}

\begin{theorem}
    Si $ \phi $ es equicontinua, entonces en $ \phi $ coinciden las topologías $ T_{p} $(producto) y la de convergencia puntual sobre un subconjunto $ D \subseteq K $ denso ($ \overline{D} = K $).
\end{theorem}


\begin{theorem}
    Sea $ \phi \subseteq  \operatorname{C}(K)$ y sea $ (K,d) $ métrico compacto. Entonces $ \phi $ es relativamente compacto en $ \|\cdot \|_{\infty} \iff \phi  $ es equicontinuo y $ \phi(x) = \{f(x): f \in \phi\} $ acotado $ \forall x \in K $
\end{theorem}


%Otra tarea que no me entero del enunciado: Diestel Spaces and Sequences in Banach spaces

\begin{theorem}\textbf{Lax- Milgram}


    Sea $ (H,<>) $ un espacio de Hilbert y $ B: H \times H \to \mathbb{K} (\mathbb{R}\ o\ \mathbb{C}) $ tal que:

    \begin{enumerate}
        \item $ B(\cdot ,y) $ es lineal $ \forall y \in H $ y $ B(x,\cdot ) $ es lineal conj., es decir, $ B $ es sesquilineal.
        \item $ B $es acotada: $ \exists c > 0 $ tal que $ |B(x,y)| \leq  C \|x\|\|y\| \forall x,y \in H $
        \item $ B $ es fuertemente positiva: $ \exists b > 0 $ tal que $ |B(x,y)| > b\|y\|^2 $, $ \forall  y \in H $
    \end{enumerate}

    Entonces para cualquier forma lineal y continua $ \phi: H \to \mathbb{K} $ existe un único $ y \in H $ tal que $ \phi(x) = B(x,y)  \forall x \in H$
\end{theorem}



\begin{demonstration}
    
    Para $ y  $ fijo la apliación $ x \hookrightarrow B(x,y) $ es lineal continuo. Por el teorema de Riesz, $ \exists z \in H $ tal que $ B(x,y) = <x,z> \forall x \in H $ y sea $ T $ la forma lineal que da el teorema de Riesz. 

    Tenemos que $ T(H) $ es un subespacio de $ H $. Veamos que $ T(H) = H $ y esto dará la prueba de nuevo por el teorema de Riesz. Demostremos varias cosas:

    \begin{enumerate}
        \item $ T(H) $ es cerrado.
            $$\text{Sea } z_n = Ty_n\ \text{tal que}\ \lim_{n \to \infty} z_n = z \in H, z \in T(H) $$

            $$ B(x,y_n-y_m) = <x,z_n-z_m> \forall x \in H $$
            $$ b\|y_n-y_m\|^{\cancel{2}} \leq  B(y_n-y_m,y_n-y_m) = <y_n-y_m,z_n-z_m>\leq \cancel{ \|y_n-y_m\|} \|z_n -z_m\| $$

            Luego $ (y_n) $ es de Cacuhy y $ \lim_{n \to \infty}y_n = y $ y tenemos que:
            $$ <x,z_n>  = B(x,y_n) \to B(x,y) = <x,z> = <x,Ty> \forall x \in H $$

        \item Supongamos $ T(H) \subsetneq H \implies \exists x_0 \ne 0:\ <x_0,z> = 0 \forall z \in T(H) \implies B(x_0,y) = <x_0,z> \forall y \in H,\ B(x_0,x_0) = 0 $ si $ x_0 \ne 0 $
    \end{enumerate}

    
\end{demonstration}

\section{Principio de Dirichlet}

\noindent Para esta sección consideraremos $ \Omega $ un subconjunto de $ \mathbb{R}^{n} $ abierto y acotado.

Lo que querremos estudiar en esta sección será el siguiente sistema llamado problema de Dirichlet:

$$ \left\{
\begin{array}{ll}
    \triangle u(x) = 0 & x \in \Omega\\
    u|_{\partial \Omega}(x) = f(x) & x \in \partial \Omega 
\end{array}
\right. $$

\begin{example}
    Tomemos $ n = 2 $, en esta dimensión existe el problema clásico de una placa que se calienta en los bordes. Queremos conocer el estado estacionario del sistema.
\end{example}

\begin{flushright}
    \textbf{Idea para buscar una solución}
\end{flushright}

Buscar el estado de equilibrio minimizando una energía o acción adecuada.

La energía que plantea Dirichlet es la de la llamada integral de Dirichlet:
$$ D(u) = \int\limits_{\Omega}^{}|\triangledown u | ^2 = \int\limits_{\Omega}^{} | \dfrac{\partial u}{\partial x_1}|^2 + | \dfrac{\partial u}{\partial x_2}|^2 d_1dx_2 $$

\begin{definition}
    $ C^2(\overline{\Omega}) $

    Denotamos a $ C^2(\overline{\Omega}) $ como las funciones dos veces derivables en el interior de $ \Omega $ con segunda derivada continua en $ \overline{\Omega} $.

    Las funciones con las que trabajaremos en este apartado son las de este tipo con soporte compacto, y al conjunto de ellas las denotaremos por $ C_0^2(\overline{\Omega }) $.
\end{definition}

Para ver una proposición necesitamos repasar el siguiente teorema:

% \begin{theorem}
%     \textbf{Integración por partes en $  \mathbb{R}^{n}$}

%     Se da la igualdad:

%     $$ \int\limits_{\Omega}^{} (\partial x_j u)vdx = - \int\limits_{\Omega}^{} u (\partial x_j v)dx + \int\limits_{\partial \Omega }^{}uvn_j d \theta $$
% \end{theorem}

% \newpage
% Este teorema se generaliza en el siguiente:

\begin{theorem}
    \textbf{Teorema de Gauss}

    Dada $ \Omega  $ suficientemente regular:

    $$ \int\limits_{\Omega }^{} \partial x_j wdx = \int\limits_{\Omega }^{} w n_j d \theta $$
\end{theorem}

\begin{proposition}
    Si existe $ u \in C^2(\overline{\Omega}) $ que minimiza a $ D(u) $ entre todas las funciones $ u \in C^2(\overline{\Omega})   $ con $ u|_{\partial\Omega} \equiv f $, entonces $ u $ es armónica ($ \triangle u = 0 $).

\end{proposition}

\begin{demonstration}
    
    En $ C^2(\overline{\Omega}) $, definimos $ \<\cdot \>_{D} $ por:
    $$ \<F,G\>_{D} = \int\limits_{\Omega}^{} \left( \dfrac{\partial F}{\partial x_1} \dfrac{\partial G}{\partial x_1} + \dfrac{\partial F}{\partial x_2} \dfrac{\partial G}{\partial x_2} \right) dx_1dx_2$$

    Definimos ahora $ D(u) = \<u,u\>_{D} $.

    Si $ v \in C^2(\overline{\Omega}) $ que verifica que $ v|_{\partial \Omega} = 0 \implies \forall  \varepsilon \in \mathbb{R} $ se tiene que $ D(u+ \varepsilon v) (*)\geq  D(u) $
    $$ (*) = D(u)+ \varepsilon^2 D(u) + \varepsilon D(v) + \varepsilon \<u,v\>_{D}+ \varepsilon \<v,u\>_{D} $$

    Cancelando $ D(u) $ tenemos:
    $$ \varepsilon^2 D(u) + \varepsilon D(v) + \varepsilon \<u,v\>_{D}+ \varepsilon \<v,u\>_{D}\geq  0 $$

    Como esto lo podemos hacer para un $ \varepsilon $ arbitrario, tenemos que $ \<u,v\>_{D} = 0\ \forall v \in C^2(\overline{\Omega }) $ con soporte compacto, luego:
    $$ 0 = \int\limits_{\Omega}^{} \left( \dfrac{\partial u}{\partial x_1} \dfrac{\partial v}{\partial x_1}+ \dfrac{\partial u}{\partial x_2} \dfrac{\partial v}{\partial x_2} \right) dx_1 dx_2 $$

    Utilizando el teorema de Gauss llegaremos al resultado deseado.

    $$ \int\limits_{\Omega}^{} \left( \dfrac{\partial u}{\partial x_1} \dfrac{\partial v}{\partial x_1}+ \dfrac{\partial u}{\partial x_2} \dfrac{\partial v}{\partial x_2} \right) dx_1 dx_2 = - \int\limits_{\Omega }^{} (\triangle u)v \hspace{2mm} \forall  v \in C_0^2(\overline{\Omega })\ con\ v|_{\partial \Omega } = 0 $$

    También sabemos que $ C_0^{\infty}(\overline{\Omega }) $ es denso en $ L^2(\Omega ) $. Entonces $ \langle \triangle u, v \rangle_{L^2} = 0 $ para cualquier $ v $ de un denso, luego $ \triangle u = 0 $


\end{demonstration}

% \subsection{Estudio del espacio $ C^{1}(\overline{\Omega }) $}

% Recordemos que tenemos:

% $$ \langle u, v \rangle_{D} = \int\limits_{\Omega }^{} (\triangledown u \cdot  \triangledown v)dx \hspace{5mm} \triangledown u \cdot  \triangledown v = \sum\limits_{j=1}^{N} \dfrac{\partial u}{\partial x_j} \dfrac{\partial v}{\partial x_j} $$

% $$ \|u\|_{D}^2 = \langle u, u \rangle_{D} = \|\triangledown u\|^2_{L^2(\Omega )} $$



\begin{theorem}
    \textbf{Desigualdad de Poincaré}

    $ \forall  f \in \mathcal{D}(\Omega ) =  C_{0}^{\infty}(\Omega ) $, $ \|f\|_{0}\leq \operatorname{diam}(\Omega )\|f\|_{1} $
\end{theorem}

Como consecuencia, tenemos continuidad en la inclusión:
$$ (\mathcal{D}(\Omega ), \langle ,  \rangle_{1}) \hookrightarrow (\mathcal{D}(\Omega ),\langle ,  \rangle_{0}) $$ 

Aquí faltan varias definiciones de tipos de espacios de Hilbert. Creo que están en el libro.

\begin{lemma}$  $

    \begin{enumerate}
        \item $ H_{1}^{0} \subseteq  H_0 $
        \item $ \forall v \in H_1^{0},\ \exists v_j \in H_0: \langle z, v_j \rangle_{0} = - \langle \dfrac{\partial z}{\partial x_j}, v \rangle_{0} $. Es decir:
        $$ \int\limits_{\Omega }^{} zv_j = - \int\limits_{\Omega }^{} \dfrac{\partial z}{\partial x_j}v $$
        \item Si $ u,v \in H^{0}_{1} \implies \langle u, v \rangle_{1} = \int\limits_{\Omega }^{} \sum\limits_{j=1}^{N} \dfrac{\partial n}{\partial x_j} \dfrac{\partial v}{\partial x_j} $
    \end{enumerate}

    A $ v_j $ se le llama $ \dfrac{\partial v}{\partial x_j} $
\end{lemma}

\begin{demonstration}
    Sea $ (v_n) \in \mathcal{D}(\Omega ),\ v_n \to v \in H_1^{0} $ en $ \|\cdot \|_{1} \implies \left( \dfrac{\partial v_n}{\partial x_j} \right)_{n=1}^{\infty} $ es de Cauchy en $ \langle ,  \rangle_{0} $ y converge a una función a la que llamamos $ v_j \in H_0 = L^2(\Omega )$. Por la desigualdad de Poincaré, $ (v_n)$ es de Cauchy en $ \|\cdot \|_{0}$ y su límite no podrá ser otro que $ v$. La fórmula:
    $$ \int\limits_{\Omega }^{}zv_j = - \int\limits_{\Omega }^{} \dfrac{\partial z}{\partial x_j} v $$

    Viene dada por el paso al límite de la expresión dada por el teorema de Gauss:
    $$ \int\limits_{\Omega }^{} z \dfrac{\partial v_n}{\partial x_j} = - \int\limits_{z}^{x_j}v_n $$
    
\end{demonstration}

\begin{lemma}
    \textbf{Variacional}

    $ \Omega  \subseteq  \mathbb{R}^{n}$ abierto y acotado, $ u \in L^2(\Omega )$:
    $$ \int\limits_{\Omega }^{} uv dx = 0 \forall  v \in \mathcal{D}(\Omega )$$

    Entonces $ u = 0 $ en $ L^2(\Omega )$ y $ u = 0\ \forall  x \in \Omega $ 
\end{lemma}


\begin{theorem}
    $ $
    
    Sea $ \Omega  \subseteq  \mathbb{R}^{N}$ un abierto y acotado. Sea $ f \in L^2(\Omega ) \equiv H_0$. Entonces existe una \textbf{solución débil} de la ecuación:
    $$ \left\{
    \begin{array}{l}
        \triangle u = f\\ 
        u|_{\partial \Omega } u = 0
    \end{array}
    \right. $$ 

    En el siguiente sentido:

    {\color{red}Existe una única función $ v \in H^{0}_{1}$ donde que $ \langle u, f \rangle = - \langle u, \triangle v \rangle$ $ \forall u \in \mathcal{D}(\Omega )$}

    Lo rojo aún ''está por aclarar''
\end{theorem}


\begin{demonstration}
    Si $ f \in L^2 \implies \phi: L^2 \to \mathbb{R}$ lineal y continua en $ H_0$ por Riesz. Aplicando la des. de Poincaré:
    $$ |\phi(n)| \leq  \|f\|_{0}\|u\|_{0} \leq  \operatorname{diam}(\Omega ) \|f\|_{0}\|u\|_{1}\ \forall  u \in \mathcal{D}(\Omega ) \implies \phi \text{ es continua en } (\mathcal{D}(\Omega ),\langle ,  \rangle_{1}) $$

    Entonces podemos extender $ \phi$ al completado, $ H_1^{0}$. A esta forma lineal extendida le aplicamos el teorema de Riesz.

    Existirá entonces una única $ v \in H_{1}^{0}$ tal que $ \phi(u) = \langle u, v \rangle_{1}  = \sum\limits_{j=1}^{N} \langle u_j, v_j \rangle_{0}$ para todo $ u \in H_1^{0}$.

    Si tomamos $ u \in \mathcal{D}(\Omega )$:
    $$ \langle u, v \rangle_{1} = \sum\limits_{j=1}^{N} \langle u_j, v_j \rangle_{0} = \sum\limits_{j=1}^{N} \int\limits_{\Omega }^{} \dfrac{\partial u}{\partial x_j} \dfrac{\partial v}{\partial x_j} \stackrel{Int.\ partes}{=} -\sum\limits_{i=1}^{N} \int\limits_{\Omega }^{} u\dfrac{\partial }{\partial x_j} \dfrac{\partial v_j}{\partial x_j} = \int\limits_{\Omega }^{}-u \sum\limits_{j=1}^{N}v_{jj} $$

    Donde las parciales son en el sentido generalizado visto en el lema.

\end{demonstration}


\begin{definition}
    \textbf{Definición de $ \mathcal{D}_{K}(\Omega )$ y su topología}

    Si $ \Omega  = \bigcup\limits_{n=1}^{\infty} K_n$ compactos, definimos $ \mathcal{D}(\Omega ) = \bigcup_{n=1}^{\infty} \mathcal{D}_{K_n}(\Omega )$

    En este espacio $ \mathcal{D}_{K}(\Omega )$, definimos una topología a partir de las seminormas utilizadas en la convergencia uniforme de los elementos de $ \mathcal{D}K(\Omega )$. Lo vemos en detalle:

    Los elementos de $ \mathcal{D}_{K}(\Omega )$ son funciones en $ \mathcal{D}(\Omega )$ tal que su soporte está en $ K$ y se dice que $ h_n \to h$ si $ h_n\to h$ de forma uniforme y sus diferenciales convergen de forma uniforme a la de $ h(x)$, es decir:
    $$ D^{\alpha}h_n(x) \to D^{\alpha} h(x)\ unif.\ \forall \alpha = (a_1,...,a_N)$$
\end{definition}

% \begin{theorem}
%     \textbf{Ritz-Galerkin}

%     Sea $ H$ un esp. de Hilbert y se sea $ B: H \times H \to \mathbb{R}$ bilineal, continua y fuertemente positiva. Sea $ b: H \to \mathbb{R}$ lineal y continua. Definimos $ F(x) = \dfrac{1}{2} B(x,x)-b(x)\ \forall x \in H$. Entonces,
%     $$ \min \{F(x): x \in H\} = F(x_0) \iff \Big[B(x_0,y) = b(y)\ \forall y \in H\Big]$$

% \end{theorem}

% Para demostrar esto necesitamos el método de Ritz

\section{Problemas variacionales cuadráticos}

\begin{theorem}
    \textbf{Principal de los problemas variacionales cuadráticos}

    Sea $ H$ un espacio de Hilbert y $ B: H \times H \to \mathbb{R}$ una forma bilineal simétrica, acotada, continua\footnote{$ \|y\|\|x\| d \geq  |B(x,y)|$} y fuertemente positiva\footnote{$ B(x,y) \geq  C \|x\|\|y\|\ \forall x,y$} y $ b: H \to \mathbb{R}$ una forma lineal continua. Sea $ F(x)$:
    $$ F(x) = \dfrac{1}{2}B(x,x) -b(x) \forall  x \in H $$ 

    Llamada forma bilineal cuadrática.

    Entonces se verifica la siguiente equivalencia:
    $$ \inf \{F(z):\  z \in H\} = F(x_0) \iff B(x_0,y) = b(y)\  \forall  y \in H$$

    Además, existe un único $ x_0$ que lo verifica.
\end{theorem}

Notemos que la norma $ \|\|$ y el módulo de $ B$ son normas equivalentes debido a que $ B$ es continua y fuertemente positiva.

\begin{demonstration} 

    Demostrado en 1.7.1. del libro. Aquí hay un boceto de la demostración:

    Dado $ t \in \mathbb{R},\ x,y \in H$, se tiene:
    $$ F(x+ty) = \dfrac{1}{2} B(x+ty,x+ty) -b(x+ty) = \dfrac{1}{2} \Big(B(x,x) +B(x,ty) + B(ty,x) + B(ty,ty) \Big) - \Big ( b(x)+tb(y)\Big) = $$
    $$ = \dfrac{1}{2} \Big(B(x,x) +2B(x,ty) + t^2B(y,y) \Big) - \Big ( b(x)+tb(y)\Big) = \dfrac{t^2}{2}B(y,y) + t \Big( B(x,y) - b(y) \Big) +\dfrac{1}{2} B(x,x) -b(x)$$

    Si el $ x_0$ es donde se alcanza el extremo inferior, se tiene que:

    $$ F(x_0) \leq  \dfrac{t^2}{2}B(y,y) + t \Big( B(x_0,y) - b(y) \Big) + \underbrace{\dfrac{1}{2} B(x_0,x_0) -b(x_0)}_{F(x_0)} $$

    Y entonces $ B(x_0,y) = b(y)$. (???)

    ''El recíproco se hace con la misma expresión y derivando''

    Para la unicidad, basta con aplicar el teorema de Riesz para el producto escalar $ \langle ,  \rangle_{B}$

\end{demonstration}

\begin{method}
    \textbf{Método de aproximación de Ritz-Galerkin}

    Dado $ B(x_0,y) = b(y) \forall y \in H$, y tomando $ H = \overline{\bigcup\limits_{n=1}^{\infty} H_n \uparrow}$ con cada $ H_n$ finito dimensional.

    Tomamos $ \operatorname{span}\{e_j^{n}: j = 1,2,...,N\}$ a una base de cada $ H_n$
    
    Al problema $ P$ presentado en el anterior teorema (la parte izquierda de la equivalencia) restringido a $ H_n$ queda:
    $$ P|_{H_n}  \equiv B(x,y) = h(y) \forall y \in H_n \longrightarrow B(x,e_{j}^{n}) = b(e_{j}^{n}) = b(e_j^{n})\ j = 1,2,...,N$$

    Entonces si la solución a cada problema es $ x_n$, se tiene que:
    $$ \lim_{n \to \infty} x_n = x_0 $$
\end{method}

\begin{demonstration}
    $$ B(x_n,y) = b(y)\ \forall  y \in H_n $$
    $$ B(x_0,y) = b(y)\ \forall  y \in H $$

    Restando llegamos a:
    $$ 0 = B(x_n-x_0,y) \forall y \in H_n $$

    Luego el $ x_n-x_0$ es la proyección ortogonal sobre $ H_n$ asociada a $ \langle ,  \rangle_{B} $ de $ x_0$ sobre $ x_n$. Entonces es la mejor aproximación sobre $ H_n$ como $ H = \overline{\bigcup\limits_{n=1}^{\infty} H_n \uparrow}$, tenemos que $ \lim_{n \to \infty}x_n = x_0$
\end{demonstration}

\section{Operaciones diferenciales y soluciones débiles }

Definamos el operador:

$$ L = \sum\limits_{|\alpha|\leq n}^{} a_{\alpha} \left( \dfrac{\partial }{\partial x} \right)^{\alpha} $$

Con:
$$ \left( \dfrac{\partial }{\partial x} \right)^{\alpha} = \dfrac{\partial ^{(\alpha)}}{\partial x_1^{\alpha_1}\partial x_2^{\alpha_2}...\partial x_N^{\alpha_N}} $$
$$ \alpha = (\alpha_1,...,\alpha_n),\ |\alpha| = \alpha_1+...+\alpha_n $$

\textbf{Problema:}

Dada $ f:\Omega \to\mathbb{R}$  encontrar $ u$ tal que $ L(u) = f$ con $ \Omega \subseteq \mathbb{R}^{N}$ abierto.

\begin{proposition}
    $ $

    Adjunto de $ L^{*} = \sum\limits_{|\alpha|\leq n}^{}(-1)^{|\alpha|} \overline{a}_{\alpha} \left( \dfrac{\partial }{\partial x} \right)^{\alpha}$ verifica que:
    $$ \langle L \phi, \psi \rangle = \langle \phi, L^{*}\psi \rangle \forall \phi,\psi\  \in \mathcal{D}(\Omega ) $$

\end{proposition}

\begin{flushright}
    \textbf{Observación}
\end{flushright}
$\forall f \in L^2(\Omega )$, si $ u \in C^{n}(\Omega )$ verifica que $ L u = f$, entonces $ \langle f, \psi \rangle = \langle u, L^{*}\psi \rangle\ \forall \psi \in \mathcal{D}(\Omega )$

\begin{definition}
    \textbf{Solución débil}

    Si $ f \in L^2(\Omega )$, $ u \in L^2(\Omega )$ es una \textbf{solución débil} de la ecuación $ L u = f$ siempre que $ \langle f, \psi \rangle = \langle u, L^{*}\psi \rangle\ \forall \psi \in \mathcal{D}(\Omega )$
\end{definition}


\begin{example}
    En $ \mathbb{R}, L = \dfrac{d}{dx}$ con $ \Omega  = (0,1),\ u,f \in L^2(\Omega )$

    Entonces $ Lu = f$ en sentido débil sii $ \exists F \in :[0,1] \to \mathbb{R}$ absolutamente continua y tal que $ F(x) = u(x)$ para casi todo punto (p.c.t) $ x \in [0,1]$ y $ F'(x) = f(x)$ p.c.t. $ x \in (0,1)$
\end{example}

Hay otros ejemplos en el libro (sección 1.10)

\section{Teorema de Radon-Nykodin}

\begin{theorem}
    Sea $ \Omega  $ es de medida. $ \Sigma$ una $ \sigma$-álgebra y $ \mu,\nu: \Sigma \to \mathbb{R}^{+}$ medidas finitas.

    Si $ \nu$ es absolutamente continua respecto de $ \mu$, es decir, $ \mu(E) = 0 \implies \nu(E) = 0$.

    Entonces existe $ g:\Omega  \to \mathbb{R}^{+}$ integrable respecto a $ \mu$ tal que:
    $$ \nu(E) = \int\limits_{E}^{}f d\mu \ \forall E \in \Sigma $$
\end{theorem}

\begin{demonstration}
    (Prueba de Von-Neumann)
    

    Sea $ H = L^2(\Omega ,\Sigma, \mu + \nu)$, entonces podemos definir la forma lineal $ \phi: H \to \mathbb{R}$ tal que $ \phi(x) = \int\limits_{}^{}xd\mu$ es lineal y continua para la norma asociada $ \|\cdot \|_{L^2(\mu)}$ y, por tanto, lo es para la norma asociada $ \|\cdot \|_{L^2(\mu+\nu)}   $ y aplicando el teorema de Riesz, existe una única función $ y \in L^2(\mu+\nu)$ tal que:
    $$ \int\limits_{}^{} x d\mu = \phi(x) = \langle x, y \rangle_{L^2(\mu+\nu)} = \int\limits_{\Omega }^{} xy d(\mu+\nu) = \int\limits_{\Omega }^{}xy d\mu * \int\limits_{\Omega }^{}xy d\nu $$

    Luego tenemos:

    $$ \int\limits_{\Omega }^{}x(1-y)d\mu = \int\limits_{\Omega }^{}xy d\nu \ \forall x \in H $$

    \textbf{Ejercicio:} Probar que $ 0<y\leq 1$ p.c.t con relación a $ \mu$.

    Definimos entonces $ g=\dfrac{1-y}{y}$. Entonces:

    $$ \int\limits_{\Omega }^{} (xy)g d\mu = \int\limits_{\Omega }^{}(xy)d \nu \ \forall x \in H $$
    $$ \int\limits_{\Omega }^{}u g d\mu = \int\limits_{\Omega }^{}u d\nu \ \forall x \in H,\ u=xg $$

    Haciendo $ u = \chi_{E}$ tenemos:
    $$ \int\limits_{\Omega }^{} g d\mu = \int\limits_{\Omega }^{} \chi_{E} g d\mu = \int\limits_{\Omega }^{}\chi_{E}d \nu = \nu(E) $$

\end{demonstration}

\chapter{Operadores Lineales}

\section{Introducción}

A no ser que se diga lo contrario, en este tema denotaremos a $ X$ como un espacio de Banach sobre $ \mathbb{K} = \mathbb{R}$ o $ \mathbb{C}$

\begin{definition}
    \textbf{Norma de un operador lineal y $ \mathcal{L}(X,Y)$}

    Sea $ X,Y$ espacios de Banach y $ T:X\to Y$ lineal y continua. Definimos la norma de $ T$ como:
    $$ \|T\| := \sup \{\|Tx\|:\ x \in  B_{X}\footnote{Bola unidad}\} $$

    Al espacio normado $\Big ( \{T: X \to Y\ lineal\ continuo\} \Big) $ lo denotamos por $ \mathcal{L}(X,Y)$.  Este espacio es completo cuando lo es $ Y$.

\end{definition}

\begin{flushright}
    \textbf{Observación}
\end{flushright}

Esta norma es la menor constante tal que $ \|Tx\|\leq  \|T\|\|x\| \ \forall x \in X$


\subsection{Nota sobre limites iterados}

Cuando tenemos una sucesión:
$$ \begin{pmatrix}
    a_{11} & a_{12} & ... & a_{1n} & ... & \to & \alpha_1\\
    a_{21} & a_{22} & ... & a_{2n} & ... & \to & \alpha_2\\
    \vdots & \vdots & ... & \vdots\\ 
    \downarrow & \downarrow & ... & \downarrow & ... \\ 
    \beta_1 & \beta_2 & ... & \beta_n

\end{pmatrix}  $$

Nos preguntamos entonces cuando se tiene que:
$$ \lim_{p \to \infty} \alpha_{p} = \lim_{n \to \infty} \beta_n $$

\begin{proposition}
    Si existe el límite doble, $ \lim_{p \to \infty}\lim_{n \to \infty} a_{pn}$, entonces se da la igualdad.
\end{proposition}

\begin{theorem}
    Si se converge por filas o por columnas se da la convergencia \textbf{uniforme}, entonces se da la igualdad.
\end{theorem}


\begin{flushright}
    \textbf{Observación}
\end{flushright}

Lo realizado sobre límites iterados también es válido para redes y filtros.

\begin{definition}
    \textbf{Operador $ T^*$}

\end{definition}

\section{Inversión de operadores lineales}

En esta sección veremos que los operadores invertibles en dimensión infinita forman un conjunto abierto (como ocurre en $ \mathbb{R}^{n}$). 

En general, estudiaremos la invertibilidad de un operador $ T: X \to X$ ($ X$ un Hilbert) estudiando la de $ (T- \lambda  Id)$. En ese caso, $ \lambda $ será un valor propio. Obteniendo los vectores propios asociados $ \{e_j\}_{j=1}^{\infty}$. Entonces si estos vectores forman una base Hilbertiana, entonces $ X = \operatorname{span} \{e_j\}$ y $ \forall x \in H$:
$$ x = \sum\limits_{j=1}^{\infty} \langle x, e_j \rangle e_j  $$
$$ T(x) = \sum\limits_{j=1}^{ \infty} \langle x, e_j \rangle \lambda_j e_j $$

Cuando tengamos esto, podremos resolver la ecuación $ T(x) = y$, donde queremos obtener la $ x$ a partir de $ T $ e $ y$.

Esto lo podremos hacer para un operador compacto (la bola unidad va a un conjunto (relativamente) compacto) y simétrico.

\begin{theorem}
    \textbf{De Von Neumann} \label{T Von Neumann}

    Si $ K \in \mathcal{L}(X)$ invertible, $ L,A \in \mathcal{L}(X)$ y sea $ L = K - A$. Entonces si $ \|A\| < \dfrac{1}{\|K ^{-1}\|}$ entonces $ L$ es invertible.
\end{theorem}

\begin{demonstration}
    \textbf{Caso 1:} $ K = Id$. Probaremos entonces que la bola de radio 1 está dentro de los elementos invertibles.

    Consideremos $ Id-B$ con $ \|B\|<1$. Veremos que esta diferencia es invertible. 

    Definimos $ S = \sum\limits_{i=0}^{\infty} B^{n}$. Veremos que esta serie es convergente.

    Supongamos que la serie es \textit{normalmente} convergente, es decir, $ \sum\limits_{}^{} \|B^{n}\| < \infty$.
    
    Por lo tanto,
    $$ \sum\limits_{n=0}^{\infty}B^{n} \text{ es de Cauchy} \implies \text{ es convergente} $$ 

    La serie es \textit{normalmente} convergente ya que dados dos operadores $ S,T$ tenemos $ \| S \circ T \| \leq  \|S\|\|T\|$. Entonces podemos la serie para ver que es convergente:

    $$ \sum\limits_{n=0}^{\infty}\|B^{n}\| \leq  \sum\limits_{n=0}^{\infty}\|B\|^{n} < \infty $$

    Siendo la última convergente al ser una serie geométrica (recordemos que $ \|B\| < 1$)

    % Y se puede ver fácilmente que $ S$ es la inversa de $ Id-B$: ''es lo mismo que ver que $ \sum\limits_{n=0}^{\infty}c^{n} = \dfrac{1}{1-c}$ pero con matrices infinitas''. 

    Veamos que $ S = (Id-B) ^{-1}$. Tenemos que $ B \circ S = B \circ \left(\sum\limits_{n=0}^{\infty}B^{n}\right)$. Como la composición es una función bilineal continua, podemos pasar $ B$ a dentro del sumatorio:
    $$ B \circ S= \sum\limits_{n=0}^{\infty} B^{n+1} = S - Id \implies (Id - B) S = Id$$

    De igual forma tenemos:
    $$ S \circ B = \sum\limits_{n=0}^{\infty}B^{n} \circ B = \sum\limits_{n=0}^{\infty} B^{n+1} = S - Id \implies S(Id - B) = Id $$

    \textbf{Caso 2:} Como $ K$ es invertible, tenemos que:  $(K-A) = K (Id - K ^{-1} A) $ será invertible cuando es composición de invertibles\footnote{Queda como ejercicio demostrarlo}.

    En primer lugar, $ K$ es invertible, y tomando $ B = K ^{-1}A$, tenemos que:
    $$ \|B\| = \|K ^{-1}A\| \leq  \|K ^{-1}\| \|A\| < 1 $$

    Y usando el caso 1, tenemos que $ (Id - K ^{-1} A)$ es invertible $ \implies $ $ K-A$ es invertible.

    Además, utilizando ambos casos podemos escribir $ (K-A) ^{-1}$ como:
    \begin{equation}
        (K - A)^{-1} = \Big ( K (Id-K ^{-1} A) \Big) ^{-1} = (Id - L ^{-1}A) ^{-1} \circ K ^{-1} = \sum\limits_{n=0}^{\infty} (K ^{-1} A)^{n} K^{-1}
        \label{calc inv}
        \end{equation}
    

\end{demonstration}

\begin{exercise}
    \textbf{Ejercicio propuesto}

    Dado un espacio normado $ (Z,\|\cdot \|)$, es completo sii toda serie normalmente convergente en $ Z$ es convergente en $ Z$.
\end{exercise}

\begin{exercise}
    \textbf{Ejercicio propuesto}

    $ \mathcal{L}(X)$ es completo.
\end{exercise}


\begin{definition}
    \textbf{Resolvente y espectro}

    Dado $ M: X\to X$ se definen el resolvente y el espectro respectivamente como:
    $$ \rho(M):= \{\lambda  \in \mathbb{C}:\ (\lambda Id-M) \text{ es invertible}\} $$
    $$ \sigma(M):= \{\lambda  \in \mathbb{C}:\ (\lambda Id-M) \text{ NO es invertible}\} $$
\end{definition}

\begin{theorem}
    Dado $ M \in \mathcal{L}(X)$, se tiene:
    
    \begin{enumerate}
        \item $ \rho(M)$ es abierto.
        \item $ \rho(M ) \to \mathcal{L}(X)$ tal que $ \lambda \mapsto (\lambda  Id -M) ^{-1}$ es analítica.
    \end{enumerate}
\end{theorem}

Esta prueba, por algún motivo, demuestra ambos puntos: 
\begin{demonstration}


    Si $ \lambda  \in \rho(M)$, podemos aplicar el teorema de Von Neumann para $ K = \lambda  Id - M$ y $ A = \lambda  Id$ y tendremos que:
    $$ (\lambda  -h) Id -M = (\lambda  Id-M) -hId \text{ será invertible si $ h$ es suficientemente pequeño} $$

    La fórmula \eqref{calc inv} nos da:
    \begin{equation}
        \Big ( (\lambda  -h)Id -M \Big ) ^{-1} = \sum\limits_{n=0}^{\infty} \Big ( (\lambda  Id -M)^{-1} h \Big ) ^{n} (\lambda Id - M) ^{-1} = \sum\limits_{n=0}^{\infty} \Big ( (\lambda  Id- M) ^{-1} \Big ) ^{n-1} h^{n}\ \text{en}\ \mathcal{L}(X) 
        \label{inv segunda}
        \end{equation}
    

    Siendo esta serie convergente cuando $ |h|<\|\lambda  Id -M ^{-1}\| ^{-1}$ (la condición que necesitábamos para aplicar el teorema).

\end{demonstration}

\begin{theorem}
    \textbf{Gelfund}

    $ \forall M \in \mathcal{L}(X)$, el espectro $ \sigma(M)$ es compacto no vacío.
\end{theorem}

\begin{demonstration}
    Tomamos una bola $ B(0,\|M\|)$, entonces tenemos $ \sigma(M) \subseteq B(0,\|M\|)$.
    
    Sea $ \xi \in \mathbb{C}$ tal que $ |\xi| > \|M\| \implies \xi \not \in \sigma(M)$ (por \eqref{calc inv} y usando el Teorema de Von Neumann \ref{T Von Neumann}). Para aplicar el teorema, tomemos $ B = \xi ^{-1} M $. Entonces \eqref{calc inv} nos da:
    $$ (\xi Id -M) ^{-1} = \xi^{-1} (Id -M \xi ^{-1}) = \sum\limits_{n=0}^{\infty}M^{n}\xi^{-(n+1)} $$

    Esta serie converge cuando $ \|M\xi ^{-1}\| < 1 $, es decir, cuando $ |\xi| > \|M\|$. Por lo tanto, $ \sigma(M) $ es compacto.

    Vemos ahora que es no vacío por reducción al absurdo. Supongamos $ \rho(M) = \mathbb{C}$ y definimos la función entera:
    $$ 
    \begin{aligned}
        \phi:\rho(M) = & \mathbb{C} & \to & L(X)\\ 
        & \lambda  & \mapsto & (\lambda Id-M) ^{-1}
    \end{aligned}
    $$

    Esta $ \phi$ verifica que $ \phi'(\lambda )= Id $ debido a \eqref{inv segunda}.

    Si $ \lim_{\lambda  \to \infty}\|\phi(x)\| = 0 \implies \phi $ es constante por el Teorema de Liouville, lo que nos lleva a una contradicción.

    Vemos este límite, tenemos que:
    $$ \|\phi(\lambda )\| = \|(\lambda  Id -M)^{-1}\| = \|\lambda ^{-1}(Id-M ^{-1}) ^{-1}\| \leq  |\lambda | ^{-1}\sum\limits_{n=0}^{\infty} \|M ^{-1}\lambda \|^{n} = \dfrac{1}{|\lambda| (1-\|M\lambda ^{-1}\|) } \to 0 $$
\end{demonstration}

\begin{definition}
    \textbf{Adjunto de un operador, $ T^*$}

    % Sea $ H$ un Hilbert y $ T: H \to H$ un operador lineal definimos un $ \phi$ como:
    % $$ 
    % \begin{aligned}
    %     \phi: & H & \to & \mathbb{C}\\ 
    %     & y & \mapsto & \langle y, Tx \rangle
    % \end{aligned} $$

    % Por el teorema de representación de Risz, tenemos un $ $

\end{definition}

\end{document} 