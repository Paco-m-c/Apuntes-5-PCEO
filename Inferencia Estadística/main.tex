\documentclass[openany]{book}
\usepackage[utf8]{inputenc}
\usepackage{verbatim}
\usepackage[hypertexnames=false]{hyperref}
\usepackage{amstext} 
\usepackage{array}   
\newcolumntype{C}{>{$}c<{$}} 


\input{structure}
\usepackage{geometry}
\geometry{
    top=3cm,
    bottom=3cm,
    left=3cm,
    right=3cm,
    headheight=14pt, 
    footskip=1.4cm,
    headsep=10pt,
}
\usepackage{graphicx}
\title{Apuntes de Inferencia Estadística}
\author{Paco Mora}
\date{\today}

\begin{document}

\maketitle

\chapter{Tema 2}

\begin{exercise}
    \textbf{Ejercicio 1.a)}

    Los valores que puede tomar el vector son
    $$ \{(0,0,0),(1,0,0),(0,1,0),(0,0,1),(1,1,0),(1,0,1),(0,1,1),(1,1,1)\} $$

    Donde tenemos que:
    $$ P(X_1=0,X_2=0,X_3=0) = P(X=0)^3 = (1-p)^3 $$
    $$ P(X_1=1,X_2=0,X_3=0) = p(1-p)^2 = P(X_1=0,X_2=1,X_3=0) = P(X_1=0,X_2=0,X_3=1)$$
    $$ P(X_1=1,X_2=1,X_3=0) = p^2(1-p) = P(X_1=1,X_2=0,X_3=1) = P(X_1=0,X_2=1,X_3=1) $$
    $$ P(X_1=1,X_2=1,X_3=1) = p^3 $$

    Es fácil comprobar que la suma de todas las probabilidades es 1.

\end{exercise}

\end{document}