\documentclass[openany]{book}
\usepackage[utf8]{inputenc}
\usepackage{verbatim}
\usepackage[hypertexnames=false]{hyperref}
\usepackage{amstext} 
\usepackage{array}   
\newcolumntype{C}{>{$}c<{$}} 


\input{structure}
\usepackage{geometry}
\geometry{
    top=3cm,
    bottom=3cm,
    left=3cm,
    right=3cm,
    headheight=14pt, 
    footskip=1.4cm,
    headsep=10pt,
}
\usepackage{graphicx}
\title{Apuntes de Álgebra Conmutativa}
\author{Paco Mora}
\date{\today}

\setcounter{secnumdepth}{0}
\begin{document}
\maketitle

\tableofcontents

\chapter{Tema 1}

\begin{exercise}
    \textbf{Ejercicio Propuesto}

    Sea $ A = \mathbb{Z}_{n} $, con $ n  $ entero >1 y $ \overline{r} \in \mathbb{Z}_{n} $. Demostrar:
    \begin{itemize}
        \item $ \overline{r} $ cancelable $ \iff \overline{r}$ invertible $ \iff $ $ mcd(r,n)=1 $
        \item $ \overline{r} $ nilpotente $ \iff $ todos los divisores primos de $ n $ dividen a $ r $.
    \end{itemize}
\end{exercise}

La siguiente proposición generaliza el ejercicio anterior.

\begin{proposition}
    Sea $ A  $ un anillo finito y sea $ a \in A $. Entonces $ a  $ es cancelable sii es invertible.
\end{proposition}

\begin{demonstration}

    Definimos
    $$ \lambda_n: A \to A\hspace{5mm} \lambda_n(x)=ax\ \forall x \in A$$

    Es inyectiva, $ \lambda_n(x) = \lambda_n(y) \iff ax = ay \implies_{a\ cancel.} x=y $

    Por lo tanto, y como $ A $ es finito, $ \lambda_n $ es biyectiva y $ 1 \in Im(\lambda_n) \iff \exists b \in A\ |\ \lambda_n(b)=1 $
\end{demonstration}


\begin{proposition}
    $ A $ reducido $ \iff $ Nil($ A $) = $ \{\text{elem nilpotentes de}\ A\} = \{0\} $
\end{proposition}

\begin{demonstration}

    $ \ \implies$

    $ A $ reducido sii $ \forall a \in A $, $ a^2=0 \implies a=0 $

    $ \impliedby $

    Por reduc. al absurdo, supongamos $ b \in Nil(A) \setminus \{0\}  \implies \exists n >0 $ (mínimo) con $ b^{n}=0 \implies b^{n-1} \ne 0 $

    Pero entonces, $ (b^{n-1})^{2} = b^{2n-2}=0  $ y $ 2n-2\geq n  $ para $ n\geq 2 $, luego llegamos a una contradicción.
\end{demonstration}

\begin{exercise}
    \textbf{Ejercicio Propuesto}

    $ \mathbb{Z}_{n}$ es un anillo reducido $ \iff $ $ n $ es libre de cuadrados.

\end{exercise}\vspace{5mm}
\textbf{Demostración del 1.9(ii)}

\begin{demonstration}
    $ a/b $ y $ a/c \implies \exists b',c' \in D /\ ab'=b,\ ac'=c$\dots
    Sean ahora $ r,s \in D $ arbitrarios y veamos que $ a/rb+sc $
    $$ rb+rc=r(ab)+s(ac') = arb' =  asc' = a(rb'+sc') \implies a | rb+sc \implies \cancel{b}1 = \cancel{b}(dc) $$
\end{demonstration}

% \begin{flushright}
%     \textbf{NOTA}
% \end{flushright}

% Hay muchos ejemplos de subconjuntos de $ G \subset A  $ tal que $ G $ no es un ideal.

\begin{exercise}
    \textbf{Ejercicio propuesto}

    Sean $ G_1,G_2 \subset A $. Demostrar que $ (G_1)(G_2)= (G_1\cdot G_2) $. En particular, el producto de ideales principales es un ideal principal.
\end{exercise}

\begin{flushright}
    \textbf{Observación}

\end{flushright}


$ IJ \subset I \cap J $ (estricto en general: $ A = \mathbb{Z},\ I =(2),\ J=(4),\ IJ=(8),\ I\cap J = (4) $)


\begin{example}
    \textbf{Aplicación del teorema de la correspondencia}

    Los ideales de $ \mathbb{Z}_{n} $ están en correspondencia con los divisores positivos de $ n $.

    $$ \mathcal{L}(\mathbb{Z}_{n}) \to \{d>0:\ d/n\} $$

    Pero los ideales de $ \mathbb{Z}_{n} $ son isomorfos a $ \{I \unlhd \mathbb{Z}\ :\ n\mathbb{Z} \subset  I\}$ por el teorema de la correspondencia, entonces:
    $$\{I \unlhd \mathbb{Z}\ :\ n\mathbb{Z} \subset  I\} = \{d\mathbb{Z} \unlhd \mathbb{Z}:\ n\mathbb{Z} \subset  d\mathbb{Z}\} = \{d\mathbb{Z}:\ d|n\}\cong \{d>0:\ d/n\}$$
\end{example}

\begin{proposition}
    \textbf{Proposición 1.31 extendido} (la prueba es la de los apuntes)
    Sean $ A,B_1,...,B_n $ anillos y sean $ g_{i}:A\to B_i $ homomorf. de anillos.
    \begin{enumerate}
        \item $ \phi:A\to B_1\times ...\times B_n $, dado por $ \phi(a) = (g_1(a),...,g_n(a)) $ es un homomorf. de anillos con núcleo $ \bigcap_{i=1}^{n}\operatorname{Ker}(g_i) $
        \item Si los $ \operatorname{Ker}(g_i) $ son comaximales dos a dos, entonces se verifica:
            \begin{enumerate}
                \item $ \operatorname{Im}(\phi) = \operatorname{Im}(g_1)\times...\times \operatorname{Im}(g_n)$
                \item $ \operatorname{Ker}(\phi) = \operatorname{Ker}(g_1)\cdots \operatorname{Ker}(g_n) $
                \item Se tiene un isom. de anillos: $ \dfrac{A}{\operatorname{Ker}(g_1)\cdots\operatorname{Ker}(g_n)} \cong \operatorname{Im}(g_1) \times ...\times \operatorname{Im}(g_n) $
            \end{enumerate} 
    \end{enumerate}

\end{proposition}

\begin{demonstration}    
    1.
    $$ \operatorname{Ker}(\phi) = \{a \in A:\ (g_1(a),...,g_n(a)) = (0,...,0)\} = \{a \in A:\ g_i(a) = 0\ \forall i \} = \cap_{i=1}^{n}\operatorname{Ker}(g_i)$$
    2.\\2.b
    
    Si los $ \operatorname{Ker}(g_i) $ son comaximales dos a dos entonces:
    $$ \operatorname{Ker} (\phi) = \operatorname{Ker}(g_1)\cdots \operatorname{Ker}(g_n) $$

    Con lo que tenemos 2b).\\
    2.a

    Si $ (b_1,...,b_n)  \in \operatorname{Im}(\phi) \implies (b_1,...,b_n) = \phi(a) = (g_1(a),...,g_n(a))$ para algún $ a \in A \implies b_i \in \operatorname{Im}(g_i)\ \forall i$. Por tanto, $ (b_1,...,b_n) \in \operatorname{Im}(g_1)\times...\times \operatorname{Im}(g_n) $

    Si probamos ahora que $ (0,...,x_i,0,...,0) \in \operatorname{Im}(\phi)\ \forall  x_i \in \operatorname{Im}(g_i) $, entonces toda $n$-upla $(x_1,...,x_n) \in \operatorname{Im}(\phi)  $ en $ \operatorname{Im}(\phi_1)\times...\times \operatorname{Im}(\phi_n) $. Como los núcleos son comaximales dos a dos.

    $$ \operatorname{Ker}(g_i) + \left( \cap_{j\ne i} \operatorname{Ker}(g_j) = A \implies 1 = a+b,\ a \in \operatorname{Ker}(g_i),\ b \in \cap_{j\ne i} \operatorname{Ker}(g_j) \right) $$

    Como $ x_i \in \operatorname{Im}(g_i) \implies \exists u \in A:\ g_i(u) = x_i $, entonces:
    $$ x_i = 1\cdot x_i = (a+b) g_i(u) = g_i((a+b)u) $$

    Luego entonces:
    $$ \phi(bu) = (g_1(bu),...,g_i(bu),...,g_n(bu)) = (0,...,0,g_i(bu),0,...,0) $$
    $$ x_i = g_i(u) = g_i(au+bu) = \cancel{g_i(a)g_i(u)}+g_i(bu) $$
    
    Con lo que queda demostrado 2.b.\\
    2.c.

    Basta utilizar 2.a), 2.b) y el primer teorema de isomorfía.

\end{demonstration}

\begin{definition}
    \textbf{Conjunto inductivo}

    Un \textbf{conjunto inductivo} es un conjunto ordenado $ S $ tal que todo subconjunto totalmente ordenado no vacío tiene una cota superior en $ S $
\end{definition}

\begin{lemma}
    \textbf{Lema de Zorn}

    Todo conjunto inductivo no vacío tiene un elemento maximal.
\end{lemma}

\begin{demonstration}
    Fijemos $ I  \trianglelefteq A,\ I \ne A $ ideal propio.
    $$ S_{I} = \{ J\trianglelefteq A:\ J \text{ ideal propio e }I \subset  J  \} $$

    $ S_{I} $ es inductivo y $ \ne \emptyset (I \in S_{I})$

    Sea $ Y $ un subconjunto totalmente ordenado $ \ne \emptyset $ de $ S_{I} $. Tomo $ m = \bigcup_{J \in T} J $. Porbemos que $ m $ es un ideal propio tal que $ I \subset  m $. Lo que implica que $ m \in S_{I} $.

    Sean $ a,b \in m \implies \left\{
    \begin{array}{l}
        a \in \bigcup_{J \in T}J \iff \exists J \in T:\ a \in J\\
        b \in \bigcup_{J \in T}J \iff \exists J' \in T:\ b \in J'\\
    \end{array}
    \right.
    $ 

    Si tomamos por ejemplo que $ J \subset  J' $, entonces $ a,b \in J' \implies a-b \in J' \implies a-b \in m $

    Notemos entonces que un elemento maximal de $ S_{I} $ es también un ideal maximal.
\end{demonstration}


\begin{exercise}
    $ I,P \unlhd A,   $ siendo $ P $ primo.  Probar que existe un primo minimal sobre $ I $, pongamos $ q $ tal que $ q \subset P $
\end{exercise}

\begin{lemma}
    \textbf{Lema de Krull}

    $ A $ anillo, $ I \unlhd $ A y $ S \subset A $ un subconjunto multiplicativo. Suponemos que $ I \cap S = \emptyset $ y consideremos $ \mathcal{L}_{I,S} = \{J \unlhd A:\ I \subset  J,\ J \cap S = \emptyset\} $. Se verifica:
    \begin{enumerate}
        \item $ \mathcal{L}_{I,S}  $ es un conjunto inductivo.
        \item Cualquier elemento maximal de $ \mathcal{L}_{I,S} $ es un ideal primo.
    \end{enumerate}

\end{lemma}

\begin{demonstration}
    1.

    Hemos de probar que si $ \mathcal{J} \subset \mathcal{I}_{I,S}  $ es un subconjunto totalmente ordenado $ \ne \emptyset \implies $ tiene una cota superior en $ \mathcal{L}_{I,S} $.
    
    Habría que comprobar que $ \widetilde{J} = \bigcup_{J \in \mathcal{J}}J $ es un ideal.

    Como tenemos que $ I \subset  \widetilde{J} $ y $ S \cap \widetilde{J} = S \cap (\bigcup J) = \bigcup_{J \in \mathcal{J}} (S\cap J) = \emptyset $

    Entonces $ \widetilde{J} $ es una cota superior de $ \mathcal{J} $ en $ \mathcal{L}_{I,S} $.\\
    2.

    Sean $ a,b \in A $ tales que $ ab \in P $. Por reducc. al absurdo, supongamos que $ a \not \in P $ y $ b \not \in P $. Entonces:
    $$ \left\{
    \begin{array}{l}
        P \subsetneq P + (a)\\
        P \subsetneq P + (b)
    \end{array}
    \right\} \implies P+(a),P+(b) \not \in \mathcal{L}_{I,S} \iff \left\{
    \begin{array}{l}
        (P+(a)) \cap S \ne \emptyset\\
        (P+(b)) \cap S \ne \emptyset
    \end{array}
    \right\} $$

    Sean entonces $ s \in (P+(a)) \cap S  $ y $ s' \in (P+(b))\cap S $. Entonces:
    $$ \left\{
    \begin{array}{l}
        s = p+ar\\
        s'=p'+br'
    \end{array}
    \right.\hspace{10mm} p,p' \in P,\ r,r' \in A $$

    $$ ss' = (p+ar)(p'+br') = pp'+pbr' + arp' + abrr' \in P \implies P \cap S \ne \emptyset $$

    Con lo que llegamos a una contradicción
\end{demonstration}


\begin{proposition}
    Sea $ A  $ un anillo e $ I \unlhd A $ un ideal \textbf{propio}. Son equivalentes:
    \begin{enumerate}
        \item Si $ a \in A $ y $ a^{n} \in I $, para algún $ n>0 $, entonces $ a \in I $
        \item Śi $ a \in A $ y $ a^2 \in I $, entonces $ a \in I $
        \item $ I $ es una intersección de ideales primos.
        \item $ I $ es la intersección de los ideales primos minimales sobre $ I $.
    \end{enumerate}

\end{proposition}

\begin{demonstration}
    $ 1\implies 2. $

    Directa.\\
    $ 2\implies 1 $

    Si $ n = 1 \implies a'=a \in I$, podemos suponer que $ a \not \in I $ y que existe $ n > 1 $, $ a^{n} \in I $ tal que $ a^{n-1}\not \in I $. Entonces tenemos:
    $$ (a^{n-1})^2 = a^{2n-2} = \underbrace{a^{n}}_{\in I}\underbrace{a^{n-2}}_{\in A} \implies (a^{n-1})^2 \in I \implies a^{n-1} \in I$$

    Con lo que tenemos una contradicción y $ a \in I $.\\
    $ 4\implies 3. $

    Directa.\\
    $ 3\implies 4. $

    Supongamos que $ \exists (P_{\lambda})_{\lambda \in \Lambda} $ ideales primos tales que $ I = \bigcap_{\lambda \in \Lambda}P_{\lambda} $

    $$ \forall  \lambda \in \Lambda,\ I \subset P_{\lambda} \implies\footnote{Por el último ejercicio propuesto.} \exists Q_{\lambda} \text{ primo minimal sobre $ I $ tal que } I \subset  Q_{\lambda} \subset  P_{\lambda}\implies  $$
    $$ \implies I \subset  \bigcap_{\lambda \in \Lambda} Q_{\lambda} \subset  \bigcap_{\lambda \in \Lambda} P_{\lambda} = I \implies I = \cap _{\lambda \in Q_{\lambda}}  $$
    $$ I \subset  \bigcap _{\substack{Q \in \operatorname{Spec}(A)\\ Q\ minimal\ I}} Q \subset  \bigcap _{\lambda \in \Lambda}Q _{\lambda} = I $$

    Con lo que tenemos 4.\\
    $ 3\implies 2. $

    Si $ a ^2 \in I = \bigcap_{\lambda \in \Lambda}P_{\lambda} \iff a^2 \in P_{\lambda},\ \forall  \lambda \in \Lambda \implies a \in P_{\lambda},\ \forall \lambda \in \Lambda \iff a \in \bigcap _{\lambda \in \Lambda }P_{\lambda } = I $\\
    $ 1\implies 4. $

    Sean $ \mathcal{Q} = \{\text{ideales primos minimales sobre }I\} $. Queremos probar que $ I = \cap_{Q \in \mathcal{Q}}Q $.  La inclusión $ \subset  $ es directa.

    Supongamos ahora que $ I\subsetneq \bigcap _{Q \in \mathcal{Q}}Q \implies  $ tomamos $ x \in \cap_{Q \in \mathcal{Q}}Q  $ tal que $  x \not \in I $.

    Como $ x \not \in I \implies x^{n}\not \in I,\ \forall n \geq  0$. Aplicamos ahora el lema de Krull con $ I $ y $ S = \{x^{n}:\ n\geq  0\} $. 

    Entonces $ \mathcal{L}_{I,S} = \{J \unlhd A:\ I \subset  J,\ J \cap S = \emptyset\} $ tiene un elemento maximal, pongamos $ P $, que es primo. Entonces:
    $$ \left\{
    \begin{array}{l}
        S \cap P = \emptyset\\
        I \subset  P 
    \end{array}
    \right\} \implies\footnote{Por el ejercicio de nuevo.} \exists Q \text{ primo minimal sobre } I:\ I \subset  Q' \subset  P \implies S \cap Q' = \emptyset$$

    Con lo que llegamos a una contradicción porque $ x \in Q' $

\end{demonstration}

\begin{definition}
    \textbf{Ideal radical}

    Un ideal que cumpla las condiciones de la anterior proposición se dice que es \textbf{radical}.

\end{definition}


\begin{definition}
    \textbf{Radical de un ideal}

    Sea $ I \unlhd A $ ideal propio, $ \sqrt{I}:= \{x \in A:\ x^{n}\in I,\ \text{ para algún }n>0\} $
\end{definition}

\begin{proposition}
    \textbf{Sustituye al Corolario 1.4.6}

    Dado $ I \properideal A $ ideal propio, el subconjunto $ \sqrt{I}  $ es un ideal radical de $ A $ y puede ser descrito por cada una de las siguientes formas equivalentes:
    \begin{enumerate}
        \item El menor ideal radical que contiene a $ I $.
        \item La intersección de todos los ideales radicales que contienen a $ I $.
        \item La intersección de todos los ideales primos que contienen a $ I $.
        \item La intersección de todos los ideales primos minimales que contienen a $ I $.
    \end{enumerate}
\end{proposition}


\begin{demonstration}
    Vemos primero que $ \sqrt{I} $ es un ideal radical de $ A $.

    Hemos de probar:
    $$ \left\{
    \begin{array}{l}
        \left.
        \begin{array}{l}
            a)\ x+y \in \sqrt{I}\ \forall x,y \in \sqrt{I}\\
            b)\  ax \in \sqrt{ I}\ \forall x \in \sqrt{I},\ a \in A
        \end{array}
        \right\}\ ideal\\
        c)\ Si\ a^{n}\in \sqrt{I},\ con\ n>0 \implies a \in \sqrt{I}
    \end{array}
    \right. $$

    Vemos en primer lugar b):

    $$(ax)^{n} = a^{n}x^{n} \implies (Como\  x^{n} \in I,\ a^{n}x^{n}\in I) \implies (ax)^{n} \in I \implies ax \in \sqrt{I} $$

    a) se demuestra utilizando el binomio de Newton:
    $$ y,x \in \sqrt{I} \implies \exists m,n>0:\ x^{m} \in I,\ y^{n} \in I $$

    Sin pérdida de generalidad, supongamos $ m = n $
    $$ (x+y)^{2n} = \sum\limits_{i=0}^{2n} \binom{2n}{i} x^{i}y^{2n-i} \in I \implies x+y \in I $$

    Para ver c), sea ahora $ a^{n}\in \sqrt{I} \implies \exists m >0:\ (a^{n})^{m} \in I \implies a^{nm}\in I \implies a \in \sqrt{I} $.

    Con lo que $ \sqrt{I} $ es un ideal radical.\\
    1.

    Sea $ J \unlhd A$ ideal radical y propio tal que $ I \subset J $. Queremos ver que $ \sqrt{I}\subset J $.

    Sea $ x \in \sqrt{I} \implies \exists n> 0:\ x^{n}\in I\implies x^{n}\in J \implies_{J\ radical}x \in J $\\
    2.

    Es consecuencia inmediata de 1.\\
    3.

    Sea $ \mathcal{V}(I) = \{P \in \operatorname{Spec}(A):\ I \subset P\}\implies? \sqrt{I} = \bigcap_{P \in \mathcal{V}(I)}P $.

    La inclusión $ \subset  $ es directa con la afirmación 1 y por ser la intersección un ideal radical. Para la otra, sabemos que $ \sqrt{I} =  $ intersección de los ideales primos minimales sobre $ \sqrt{I} $. Entonces:
    $$ \sqrt{I} = \bigcap _{\substack{Q \in \operatorname{Spec}(A)\\ Q\ minimal / \sqrt{I}}} Q \supseteq \bigcap_{P \in \mathcal{V}(I)}P  $$

    Luego ya tenemos la igualdad.\\
    4.

    Se demuestra aplicando el ejercicio.

\end{demonstration}

\begin{example}
    Tomamos el caso $ (I) = 0 $
    $$ \sqrt{(0)} = \{x \in A:\ x^{n} = 0\} = \{\text{nilpotentes de } A\} =: \operatorname{Nil}(A) $$

    \begin{enumerate}
        \item $ \operatorname{Nil}(A) $ es el menor ideal radical de $ A $ 
        \item $ \operatorname{Nil}(A) $ es la intersección de todos los ideales radicales de $ A $.
        \item $ \operatorname{Nil}(A) $ es la intersección de todos los ideales primos de $ A $.
        \item $ \operatorname{Nil}(A) = \bigcap_{P \in \operatorname{MinSpec}(A)}P$
    \end{enumerate}
\end{example}

\section{Ejercicios}


\setcounter{ex}{1}
\begin{exercise}
    $$ x,y \in \mathcal{U}(A)\implies xyy ^{-1}x ^{-1} = 1 \implies xy \in \mathcal{U}(A) $$

    $$  xy \in \mathcal{U}(A  ) \implies \exists w \in A:\ xyw = 1 \implies \left\{
    \begin{array}{l}
        x ^{-1}= yw \\
        y ^{-1} = wx
    \end{array}
    \right. $$

\end{exercise}


\begin{exercise}$  $

    \begin{flushright}
        \textbf{En este ejercicio hay una errata, está por solucionar}
    \end{flushright}

    Sabemos que en un anillo finito, las unidades y los elementos cancelables son los mismos. Luego $ |\mathcal{U}(\mathbb{Z}_{n})| = |\{cancelables\}|$. Además sabemos que $ |\{divisores\ de\ cero\}| = n - |\mathcal{U}(\mathbb{Z}_{n})|  $ . Además sabemos que:
    $$ |\mathcal{U}(\mathbb{Z})_{n}| = \phi(n ) = p_1^{\alpha_1-1}\cdots p_{r}^{\alpha_{r}-1} (p_1-1)\cdots (p_{r}-1) $$

    Entonces,
    $$ |\{divisores\ de\ cero\}| = p_1^{\alpha_1-1} \cdots p_{r}^{\alpha_{r}-1} (p_1\cdots p_{r}- \prod_{i=1}^{r}(p_i-1)) $$

    Vemos entonces el cardinal de $ \operatorname{Nil}(\mathbb{Z}_{n}) $:

    $$ \overline{k} = k+n\mathbb{Z} \in  \operatorname{Nil}(\mathbb{Z})_{n} \iff \text{todos los $  p_i$ dividen a }k $$
    $$ \overline{k} \in \operatorname{Nil}(\mathbb{Z})_{n} \iff \exists t>0:\ \overline{k}^{t} = \overline{0}\ en\ \mathbb{Z}_{n}\iff \exists t>0:\ n/k^{t} \implies \text{todos los $ p_i $ dividen a }k$$

    Recíprocamente:
    $$ k = p_1^{\beta_1}\cdots p_{r}^{\beta_{r}},\ con\ 0<\beta_i \leq  \alpha_i\ \forall i = 1,...,r $$

    $$ |\operatorname{Nil}(\mathbb{Z})_{n}| = \alpha_1\cdots \alpha_{r} $$
\end{exercise}


\begin{exercise}
    $$ \mathcal{U}(\mathbb{Z}_{24}) = \{\overline{k}:\ \operatorname{mcd}(k,n) = 1\} = \{cancelables\} = \{\overline{1},\overline{5},\overline{7},\overline{11},\overline{13},\overline{17},\overline{19},\overline{23}\} $$
    $$ \{\text{divisores de cero}\} = \mathbb{Z}_{24} \setminus \mathcal{U}(\mathbb{Z}_{24}) $$


\end{exercise}

\setcounter{ex}{5}

\begin{exercise}
    $  $
    
    Recordemos primero que $ p \in A $ es primo sii $ (p) $ es un ideal primo.

    $ f:A \to B\ homomorf $. Si $ a $ satisface $ (P) $,¿ $ f(a) $ cumple $ (P) $?
    
    \begin{flushright}
        \textbf{Apartado a)}
    \end{flushright}
    
    Si $ a \in \mathcal{U}(A) \implies \exists a ^{-1} \in A:\ a\cdot a ^{-1} = 1 \implies f(a)f(a ^{-1}) = f(1) = 1 \implies f(a) \in \mathcal{U}(B)$.

\begin{flushright}
    \textbf{Apartado b)}
\end{flushright}

    % Encontraremos un contraejemplo, sea $ f: \mathbb{Z} \to \mathbb{Z}_{6} $ con la proyección canónica, basta tomar $ a = 3 $


Tomando $ \mathbb{Z} \to \dfrac{\mathbb{Z}[X]}{(2x)} $ homomorfismo inyectivo. El 2 es cancelable en $ \mathbb{Z} $ pero no lo es en el anillo destino.


\begin{flushright}
    \textbf{Apartado c)}
\end{flushright}

Sea $ a \in A $ divisor de 0 $ \implies \exists b \in A \setminus \{0\}:\ ab = 0 \implies f(a)f(b) = 0  $

Cuando $ f $ es inyectiva: sí, porque $ f(b)  \ne 0 $. En otro caso:

Sean $ m,n > 1,\ mn\mathbb{Z} \subset  n\mathbb{Z} \implies  $ tomamos un homomorfismo de anillos suprayectivo:

$$ \dfrac{\mathbb{Z}}{mn\mathbb{Z}} \to \mathbb{Z}\dfrac{\mathbb{Z}}{n\mathbb{Z}} $$

Tomando $ m $, $ n $ tales que $ \operatorname{mcd}(n,m) = 1 $ tenemos que $ \overline{m} $ es divisor de cero pero su imagen, $ [m] \in \mathcal{U}(\mathbb{Z}_{n}) $

\begin{flushright}
    \textbf{Apartado d)}
\end{flushright}

Si $ a \in A $, existe un exponente $ n > 0 $ tal que  $ a^{n} = 0 \implies f(a)^{n} = f(a^{n}) = 0 $, entonces $ f(a) $ es nilpotente.

\begin{flushright}
    \textbf{Apartado e)}
\end{flushright}

De forma parecida al apartado anterior, vemos que si $ e = e^2  $ en $ A $, al aplicar $ f $ tenemos que $ f(e) = f(e)^2 \implies f(e) $ es idempotente.

\begin{flushright}
    \textbf{Apartado f)}
\end{flushright}

Basta tomar la inclusión de $ \mathbb{Z}  $ en $ \mathbb{Q} $ para tener un contraejemplo (no suprayectivo). Para el caso suprayectivo planteamos un ejercicio:

\textbf{Ejercicio:} Sea $ \overline{k} = kp^{t}\mathbb{Z} $ es irreducible en $ \mathbb{Z}_{p^{t}} \iff \overline{k} = \overline{p} \overline{u}$, siendo $ \overline{u} \in \mathcal{U}(\mathbb{Z}_{p^{t}}) $. Más generalmente: Sea $A $ un anillo y  $ p \in A $ tales que $ (p)  $ es el único ideal maximal de $ A $- Entonces los elementos irreducibles de $ A $ son los de la forma $ pu $, siendo $ u \in \mathcal{U}(A) $ ($ p $ es el único irreducible de $ A $ salvo asociados)

Construimos en base a este ejercicio el homomorfismo suprayectivo formado por la proyección $ \mathbb{Z} \to \mathbb{Z}_{p^{t}} $. Dado $ q \ne p $ primo, su imagen es $ \overline{q} \in \mathcal{U}(\mathbb{Z}_{p^{t}}) \implies \overline{q}  $ no es irreducible.

\begin{flushright}
    \textbf{Apartado g)}
\end{flushright}

\textbf{Ejercicio:} Sea $ A $ un dominio y $ p \in A $. Si $ p $ es primo entonces es irreducible. Cuando $ A $ es un DIP, se verifica también el recíproco.

Como los contraejemplos del apartado anterior parten de $ \mathbb{Z} $ y los irreducibles y los primos son iguales en $ \mathbb{Z} $, podemos usar los mismos contraejemplos en este apartado.

\hrulefill

Vamos a resolver ahora el primero de los ejercicios planteados:

\textbf{Ejercicio:} Sea $ \overline{k} = kp^{t}\mathbb{Z} $ es irreducible en $ \mathbb{Z}_{p^{t}} \iff \overline{k} = \overline{p} \overline{u}$, siendo $ \overline{u} \in \mathcal{U}(\mathbb{Z}_{p^{t}}) $. Más generalmente: Sea $A $ un anillo y  $ p \in A $ tales que $ (p)  $ es el único ideal maximal de $ A $- Entonces los elementos irreducibles de $ A $ son los de la forma $ pu $, siendo $ u \in \mathcal{U}(A) $ ($ p $ es el único irreducible de $ A $ salvo asociados)

Dado $ p = ab $, veamos si $ p $ es irreducible. Supongamos que $ a \not \in \mathcal{U}(A) \implies (a) \unlhd A \implies (a) \subset  (p)\ \text{porque $ (p) $ es el único ideal maximal.} \implies a= pa',\ siendo\ a' \in A $

$$  \implies p = ab = pa'b \iff p(1-a'b) = 0 \left\{
\begin{array}{l}
    1-a'b \in \mathcal{U}(A) \text{ no, porque implicaría}\\ \hspace{5mm} \text{una contradicción } (p=0) \\
    1-a'b \not \in \mathcal{U}(A)
\end{array}
\right. $$

$$ 1-a'b \not \in \mathcal{U}(A) \implies (1-a'b) \subset (p),\ \text{pero no puede darse $(a'b) \subset (p)$, porque tendríamos }$$
$$ 1 = 1-a'b+a'b \in (p) \implies a'b \in \mathcal{U}(A) \implies b \in \mathcal{U}(A)$$

Sea $ q \in A $ irreducible $ \implies q \not \in \mathcal{U}(A) \iff (q) \properideal A \implies (q) \subset (p) \implies q=pu$, para algún $ u \in A $

\hrulefill

Vemos ahora los recíprocos.

\begin{flushright}
    \textbf{Apartado a)}
\end{flushright}

La inclusión de $ \mathbb{Z} $ a $ \mathbb{Q} $ y tomando $ a = f(a) = 3 $ tenemos un contraejemplo no suprayectivo, para el sobre, tomamos la proyecctión de $ \mathbb{Z} $ en $ \mathbb{Z}_{3} $.

\begin{flushright}
    \textbf{Apartados b,c)}
\end{flushright}

Basta aplicar el contrarrecíproco de $ f(a)\ cancelable \implies a\ cancelable $ y $ f(a)\ divisor\ de\ 0 \implies a\ divisor\ de\ 0 $

\begin{flushright}
    \textbf{Apartado d)}
\end{flushright}

$ f(a) $ es nilpotente  $\iff f(a) $ tal que $ \exists n>0 $ tal que $ f(a)^{n} = 0 \implies f(a^{n}) = 0\iff a^{n} \in \operatorname{Ker}(f) $.

Si $ f $ es inyectiva, sí se cumple la cadena de sii.

Si $ f $ es sobre, tomamos el contraejemplo de la proyección de $ \mathbb{Z} $ en $ \mathbb{Z}_{n} $ con un producto de primos

\begin{flushright}
    \textbf{Apartado e)}
\end{flushright}

De forma parecida al apartado anterior:

$$ f(a) = f(a^2)\iff a-a^2 \in \operatorname{Ker}(f) $$

Si $ f $ es inyectiva, sí se cumple. 

En el caso sobre, tomamos la proyección de $ \mathbb{Z} $ en $ \mathbb{Z}_{6} $, entonces 7 no es idempotente y $ f(7) = \overline{1} $ no lo es.

\begin{flushright}
    \textbf{Apartado f)}
\end{flushright}


% $ a $ no puede ser una unidad, porque lo sería también $ f(a) $.

% Análogamente $ a $ no puede ser 0, porque lo sería también $ f(a) $.

% Si $ a=bc \implies f(a) = f(b)f(c) \implies f(b) \in \mathcal{U}(B)\ \acute o f(c) \in \mathcal{U}(B)$

Para el caso sobre, tomamos la aplicación $ \mathbb{Z} \to \mathbb{Z}_{p^{t}} $ y el elemento $ (p^{t}+1)p \leadsto \overline{p} $

La idea para obtener el caso inyectivo es tomar un elemento como $ 2\cdot 3 $ no irreducible, y llevar uno de sus factores a una unidad. Tomamos la aplicación:
$$ \mathbb{Z} \hookrightarrow \mathbb{Z}\left[\dfrac{1}{2}\right] = \{q \in \mathbb{Q}:\ q = \dfrac{m}{2^{r}},\ m \in \mathbb{Z},\ r\geq  0\} $$

Dejamos como ejercicio ver que 3 es irreducible en $ \mathbb{Z}[1/2] $

\end{exercise}

\setcounter{ex}{6}

\begin{exercise}
    $  $
    \begin{flushright}
        \textbf{Apartado a)}
    \end{flushright}
    Si $ m<0 \implies \mathcal{U}(\mathbb{Z}(\sqrt{m}))$ es finito. 
    $$ N(a+b\sqrt{m}) = 1\iff a^2-mb^2 = 1 \iff a^2+b\sqrt{-m}^2 = 1 $$

    $ \implies (a,b\sqrt{-m}) $ está en la circunferencia de centro $ (0,0) $ y radio 1 y su 1ª componente $ a $ es entera.

    $$ \implies \mathcal{U}(Z(\sqrt{m})) \subset  \{a+b\sqrt{m}:\ (a,b\sqrt{-m}) \in \{(1,0),(-1,0),(0,1),(0,-1)\}\} $$

    \begin{flushright}
        \textbf{Observación}

    \end{flushright}

    $$ a = 0\iff b\sqrt{-m} = \pm 1 \implies \left\{
    \begin{array}{l}
        b = \pm 1\\ \sqrt{-m} = 1
    \end{array}
    \right. 
    \implies -m = 1 \implies m = -1 $$

    Luego $ \mathcal{U}(\mathbb{Z}[\sqrt{m}]) = \{-1,1\} $ salvo cuando $ m = -1 $ en que $ \mathcal{U}(\mathbb{Z}(i)) = \{1,-1,i,-i\} $


    \begin{flushright}
        \textbf{Apartado b)}
    \end{flushright}
    
    % Si $ m > 0 $y $ |\mathcal{U}(\mathbb{Z}[\sqrt{m}])| > 2 \implies |\mathcal{U}(\mathbb{Z}[\sqrt{m}])| = \infty $

    Supongamos que $ |\mathcal{U}(\mathbb{Z}[\sqrt{m}]) | > 2 $ y cojamos $ \alpha = a +b\sqrt{m} \ne \pm 1 $

    Tomamos $X:= \{1,\alpha,\alpha^2,...\} =  $ el subgrupo multiplicativo de $ \mathcal{U}(\mathbb{Z}[\sqrt{m}])$ generado por$ \alpha $.

    Si $ X $ es finito $ \implies \exists n > 0 :\ \alpha ^{n} = 1 $. Elegimos $ n $ mínimo con esa propiedad $  \implies  $ $ \alpha  $ raíz $ n- $ésima (primitiva)  de 1.

    Como $ m > 0 \implies \alpha = a+b\sqrt{m} \in \mathbb{R} \implies  \alpha = \pm 1\ \text{(contradice que el que }\alpha \ne \pm 1) $

    \begin{flushright}
        \textbf{Apartado c)}
    \end{flushright}
    
    Por las conclusiones tomadas en el apartado a). Se tiene que $ \mathcal{U}(\mathbb{Z}(\sqrt{-11})) = \{1,-1\} $

    Se trata de ver ahora que $ x = 1+\sqrt{-11} $ e $ y = 1-\sqrt{-11} $ son irreducibles. Como son conjugados, bastará con ver que uno solo de ellos es irreducible.

    En primer lugar, no es cero ni una unidad. Pongamos $ x = (a+b\sqrt{-11})(c+d\sqrt{-11}) $. Tomando normas:
    $$ 12 = N(x) = N(a+b\sqrt{-11})N(c+d\sqrt{-11}) $$

    Si ni $ a+b\sqrt{-11} $ ni $  c+d\sqrt{-11}$ son unidades $ \implies $ las combinaciones posibles de normas son $ (2,6),(3,4),(4,3),(6,2) $. En cualquier caso, la norma de uno de ambos es 2 o 3. Sin pérdida de generalidad, vamos a suponer que la norma de $ (a+b\sqrt{-11}) $ es 2 o 3, en cualquiera de los casos un primo $ p $.

    $$ \{2,3\} \ni p = N(a+b\sqrt{-11}) = a^2+11b^2 \implies_{b\ entero} b = 0  \implies a^2 = p $$

    Lo cual es imposible porque $ a $ es entero, hemos llegado a una contradicción  y $ x $ es irreducible.

    Ahora tenemos que $ xy = (1+\sqrt{-11})(1-\sqrt{-11}) = 12 = 2\cdot 2\cdot 3 $

    Basta ver ahora que 2 y 3 son irreducibles en $ \mathbb{Z}[\sqrt{-11}] $
    $$  p = (a+b\sqrt{-11})(c+d\sqrt{-11}) \implies_{tomando\ normas} p^2 = N(a+b\sqrt{-11})N(c+d\sqrt{-11})  $$

    Supongamos que ninguno de estos dos es 1, tenemos que $ N(a+b\sqrt{-11}),N(c+d\sqrt{-11}) = p $ y aplicando un razonamiento como el anterior, tenemos que es imposible y entonces $ p $ es irreducible.

    \begin{flushright}
        \textbf{Apartado d)}
    \end{flushright}
    
    Nos preguntamos si cuando un primo entero $ p>1 $, ¿ es irreducible en $ \mathbb{Z}[\sqrt{-3}] $?

    Tomamos una factorización $ p = (a+b\sqrt{-3})(c+d\sqrt{-3}) $ y tomamos normas:
    $$ p^2 = N(a+b\sqrt{-3})N(c+d\sqrt{-3}) $$

    Entonces tenemos:
    $$ p\ \text{irreducible} \iff N(a+b\sqrt{-3}) = 1\ \acute o N(c+d\sqrt{-3}) = 1 $$
    $$ p\ \text{no es irreducible}\iff N(a+b\sqrt{-3}) = p = N(c+d\sqrt{-3}) \iff\footnote{utilizando la ecuación de antes y que $  \mathbb{Z} $ es un dominio} N(a+b\sqrt{-3}) = p $$

    Como conclusión, tenemos que $ \mathbb{Z} $ es irreducible en $ \mathbb{Z}[\sqrt{-3}] $ sii la ecuación $ x^2+3y^2 = p $ no tiene solución en $ \mathbb{Z} \times \mathbb{Z}  $.

    Basta aplicar ahora este resultado a los 4 números a los que nos piden comprobar si son o no irreducibles.

\end{exercise}

\setcounter{ex}{8}

\begin{exercise}
    $  $

    Supongamos que $ (b,X) $ es principal y tenemos $ f \in A[X]:\ (b,X) = (f) \implies $ 
    $$ \implies \left\{
    \begin{array}{l}
        X = f(X)g(X),\ con\ g,h \in A[X]\\
        b = f(X)h(X)
    \end{array}
    \right. 
    \implies_{X = 0} \left\{
    \begin{array}{l}
        0 = f(0)g(0)\\ b = f(0)h(0)
    \end{array}
    \right.\ \text{(igualdades en }A) $$

    Notemos que $ b $ cancelable $  \implies f(0) $ cancelable:

    Si fuese $ f(0)  $ no cancelable (= divisor de 0) $ \implies $
    $$ \exists c \in A \setminus \{0\}:\ f(0)c = 0 \implies \left\{
    \begin{array}{l}
        bc = 0\\ c\ne 0 
    \end{array}
    \right\} \implies b\ \text{no es cancelable (contradicción)} $$

    $$ f(0) \implies g(0) =0 \implies g(X) = X\cdot g'(X) \implies X = f(X)g(X) = X f(X)g'(X)\implies$$
    $$\implies_{X\ cancelable\ en\ A[X]} 1 = f(X)g'(X) \implies  f \in \mathcal{U}(A[X]) \implies (b,X) = (f) = A[X] $$

    Entonces $ 1 = br(X)+Xs(X) $ para ciertos $ r,s \in A[X] \implies_{X=0} 1 = br (0)\implies b \in \mathcal{U}(A)$, lo cual es una contradicción ya que sabemos que $ b $ no es invertible.

    Falta ver que $ (X,Y) $ no es principal en $ A[X,Y] $
    $$ A[X,Y] \cong (A[Y])[X] $$

    $ Y $ no es cancelable y no unidad en $ A[X] $, basta aplicar ahora el ejercicio.
\end{exercise}


\begin{exercise}
    $  $

    \begin{flushright}
        \textbf{Apartado a)}
    \end{flushright}
    

    $$ IJ_1 = IJ_2 \cancel{\implies} J_1 = J_2 $$

    Tomaremos $ J_2 = 0 $ y $ I = J_1 = (\overline{2})  $ en $ \mathbb{Z}_{4} $

    \begin{flushright}
        \textbf{Apartado b) Enunciado modificado}
    \end{flushright}
    
    Todo ideal principal en un dominio cancela (para el producto de ideales)

    $ I = (y) $ y tenemos que $ IJ_1 = IJ_2 \implies? J_1 = J_2 $

    Basta con probar que $ J_1 \subset  J_2 $.

    Sea $ z \in J_1 \implies yz \in IJ_1 = IJ_2 \implies yz = \sum\limits_{i=1}^{t}y_iz_i $
    $$ y_i \in I = (y) \implies y_i = a_iy \text{ , para algún }a_i \in A \implies yz = \sum\limits_{i=1}^{t}(a_iy)z_i = y \sum\limits_{i=1}^{t}a_iz_i  $$ 
    $$  \implies z = \sum\limits_{i=1}^{t}a_iz_i \implies z \in J_2$$


\end{exercise}

\begin{exercise}
    Este ejercicio no está resuelto pero es muy importante. 
\end{exercise}

\begin{exercise}
    \textbf{bis}
    
    \textbf{Sea $ A = A_1\times ... \times A_m $, donde los $ A_i $ son anillos locales (Ej1.21). Probar que los idempotentes de $ A $ son las $ m- $uplas $ (e_1,...,e_m)  $ tales que $ e_i \in \{0,1\}\ \forall i = 1,...,m$. Como aplicación, describir un método para calcular todos los elementos idempotentes de $ \mathbb{Z}_{n} $, para $ n>1 $. Particularizarlo a $ \mathbb{Z}_{4200} $.}

    Utilizando el ejercicio 5.e), tenemos que $ e = (e_1,...,e_m) $ es idempotente en $ A \iff e_i $ es idempotente en $ A\ \forall i = 1,...,m $

    La primera parte se reduce a probar que si $ B $ es un anillo local, entonces sus únicos idempotentes son 0,1.

    \begin{demonstration}
        Supongamos que $ e = e^2 \in B,\ e \not \in \{0,1\} \implies e,1-e$ son idempotentes (1.12(b)) y $ e,1-e \not \in \mathcal{U}(B) $ (1.12(c)). Por tanto $ (e),(1-e) $ son ideales propios de $ B \implies (e),(1-e) \subset  m:=$ único ideal maximal de $ B $. Entonces, $ (e)+(1-e)\subset m \implies 1=e+(1-e) \in m $

        Con lo que tenemos una contradicción porque $m \subsetneq B$

    \end{demonstration}

    \textbf{Describimos el método:} $ n = p_1^{\mu_1}\cdots p_{t}^{\mu_{t}}\implies\footnote{Teorema chino de los restos} \mathbb{Z}_{n} \cong \mathbb{Z}_{p_1^{\mu_1}}\times \mathbb{Z}_{p_t^{\mu_t}}$ que lleva $ \overline{a} \hookrightarrow (\overline{a},...,\overline{a}) $

    Entonces en $ \mathbb{Z}_{p^{t}} $, el único ideal maximal es $ (\overline{p}) $ ($ p $ primo).

    Vemos el caso de $ n = 4200 = 2^3\cdot 3\cdot 5^2\cdot 7$, tomamos el isomorfismo de anillos:
    $$ \mathcal{U}:\mathbb{Z}_{4200} \to \mathbb{Z}_{8}\times\mathbb{Z}_{3}\times\mathbb{Z}_{25}\times\mathbb{Z}_{7} $$
    $$ \overline{a} \hookrightarrow (a+8\mathbb{Z},a+32\mathbb{Z},a+25\mathbb{Z},a+7\mathbb{Z}) $$

    Hay $ 2^{4} = 16 $ idempotentes:
    Calculamos el idempotente $ \overline{e} \in Z_{4200}    $ tal que $ \phi(\overline{e}) = (\overline{1},\overline{0},\overline{1},\overline{0}) $

    Luego se nos queda el sistema de congruencias:
    $$ \left\{
    \begin{array}{l}
        e \equiv 1\ (mod\ 8)\\
        e \equiv 0\ (mod\ 3)\\
        e \equiv 1\ (mod\ 25)\\
        e \equiv 0\ (mod\ 7)\\
    \end{array}
    \right. $$

    Las ecuaciones que son congruentes con 1 se pueden agrupar en $ x \equiv\ (mod\ 200 = 8\cdot 25) $, de forma análoga nos queda, $ x\equiv\ (mod\ 21) $.
    $$ \left\{
    \begin{array}{l}
        x = 1+200t\\ 
        x = 21s
    \end{array}
    \right.
    \implies 1+200t = 21s \implies 1 = 21s+200(-t) $$

    Y utilizando la identidad de Bézout y el algoritmo de Euclides obtendremos una solución, en este caso es ($ s = -19,\ t = -2 $)

\end{exercise}

\setcounter{ex}{11}
\begin{exercise}$  $
    \begin{flushright}
        \textbf{Apartado a)}
    \end{flushright}
    
    $$ a \in (e) \iff a = ea $$
    $ \implies $

    Clara.\\
    $ \implies $
    
    Sea $ a \in (e) \implies a = ex $, con $ x \in A \implies ea = e^2x = ex = a $

    $$ \left\{
    \begin{array}{l}
        a = ea\\ b = eb
    \end{array}
    \right\} ab = e^2 ab = eab $$
    
    Si $ e  $ es una unidad, entonces $ e = 1 $
    $$ e = ^2 \implies 1 = e $$

    \begin{flushright}
        \textbf{Apartado c)}
    \end{flushright}
    % $$ (1-e)^2 = 1+e^2-2e = 1+e-2e = 1-3 $$
    Sea $ a \in (e) \cap (f) \implies \left\{
    \begin{array}{l}
        a = ea\implies fa = fea = 0 \\ a = fa 
    \end{array}
    \right\} \implies a = 0 $

    Y tenemos que $ (e)+(f) = A $ ya que $ e+f = 1 $.

    Como anillos (no ideales), tenemos el isomorfismo de anillos $ A \to (e) \times (f) $ dado por $ a\hookrightarrow (ae,af) $

    \begin{flushright}
        \textbf{Apartado d)}
    \end{flushright}
    
    $$ 1 = e+f,\ e \in I,\ f \in J \implies e+f = 1 = 1^2 = (e+f)^2 = e^2+\underbrace{2ef}_{I \cap J = \{0\}}+f^2 = e^2+f^2 $$

    Hemos descompuesto el 1 como suma de elementos de $ I,J $ de dos formas distintas. Como la suma es directa, tenemos entonces que $ e = e^2,\ f= f^2 $. Luego $ e $ es idempotente, $ f = 1-e $ y tenemos $ (e) \subset\ I,\ (1-e) \subset J $. Falta ver que se da la igualdad:
    $$ Sea\ x \in I \implies x =x\cdot 1 = x(e+f)=xe+\underbrace{\cancel{xf}}_{\in I \cap J = \{0\}} $$


\end{exercise}

\setcounter{ex}{13}

\begin{exercise}
    Supongamos que $ f $ es suprayectivo y vamos a probar que si $ P \unlhd B$ y $ f ^{-1}(P) $ es primo (en $ A $). Entonces $ P $ es primo en $ B $.

    Usando el primer teorema de isomorfía, $ \dfrac{A}{\operatorname{Ker}(f)}\cong B $:

    $$ \operatorname{Spec}(B)\to \operatorname{Spec}\left(\dfrac{A}{\operatorname{Ker}(f)}\right) $$
    $$ Q \hookrightarrow \overline{f}^{-1}(q) = \{a \in \dfrac{A}{\operatorname{Ker}(f)}:\ f(a) \in Q\} = \dfrac{f ^{-1}(q)}{\operatorname{Ker}(f)} $$

    Y utilizando 1.38.2 de los apuntes de Alberto tenemos la biyección:
    $$ \{\substack{\text{ideales primos que}\\\text{contienen a }A}\} \to \operatorname{Spec}(\dfrac{A}{\operatorname{Ker}(f)}) $$

    Esta biyección lleva $ f ^{-1}(P) \hookrightarrow \dfrac{f ^{-1}(P)}{\operatorname{Ker}(f)} \in \operatorname{Spec}(\dfrac{A}{\operatorname{Ker}(f)}) $
\end{exercise}

\setcounter{ex}{16}

\begin{exercise}
    \textbf{Consideración previa general.}

    Sea $ I \properideal A $ y queremos identificar $ (\overline{a}_{1,...,\overline{a}_{m}}) = $ ideal de $ A/I $ generado por $ \{\overline{a}_{1},...,\overline{a}_{m}\} $.
    $$ (\overline{a}_{1},...,\overline{a}_{m}) = \dfrac{J}{I},\ \text{para cierto }J \ideal A:\ I \subset J $$

    Sea ahora $ J = (a_1,...,a_m) +I $, tendremos que comprobar si:
    $$ (\overline{a}_{1},...,\overline{a}_{m}) =? \dfrac{(a_1,...,a_m)+I}{I} $$

    La inclusión $ \subset  $ es directa porque cada uno de los $ \overline{a}_{i} $ se incluye en $\dfrac{(a_1,...,a_m)+I}{I}$.

    Para la inclusión $ \supset $, tenemos:
    $$ \overline{z} \in \dfrac{(a_1,...,a_m)+I}{I} \implies z +I = b+y+I,\ con\ b \in (a_1,...,a_{r}),\ y \in I \implies_{y \in I} z+I = b+I$$

    Por tanto, todos los elementos de $ \dfrac{(a_1,...,a_m)+I}{I} $ son de la forma $ b+I =\overline{b}$, donde $ b \in (a_1,...,a_m) $, pero $ b = r_1a_1+...+r_ma_m $ con $ r_i \in A\ \forall i =1,...,m $. Tomando ahora clases tenemos:
    $$ \overline{b} = \overline{r}_1\overline{a}_1+...+\overline{r}_m\overline{a}_m \implies \overline{b} \in (\overline{a}_{1},...,\overline{a}_{m}) $$

\hrulefill

    \begin{flushright}
        \textbf{Observación:}
    \end{flushright}

    Llamamos $ B = \dfrac{K[X,Y,Z]}{(XY,XZ)},\ A = K[X,Y,Z],\ I = (XY,XZ) $.

    \begin{flushright}
        \textbf{Apartado a)}
    \end{flushright}
    
    $$ (\overline{X},\overline{Y})\ = \dfrac{(X,Y)+(XY,XZ)}{(XY,XZ)} \left( \substack{Obs:\ (XY,XZ)\\ \subseteq (X,Y)} \right) = \dfrac{(X,Y)}{(XY,XZ)}$$

    \begin{flushright}
        \textbf{Apartado b)}
    \end{flushright}
    
    Razonamiento parecido al apartado anterior:
    $$ (\overline{X},\overline{Z}) = \dfrac{(X,Z)}{(XY,XZ)} $$

    \begin{flushright}
        \textbf{Apartado c)}
    \end{flushright}
    
    Razonamiento como en a).
    $$ (\overline{Y},\overline{Z}) = \dfrac{(Y,Z)}{(XY,XZ)} $$
    
    \begin{flushright}
        \textbf{Apartado d)}
    \end{flushright}

    Razonamiento como en a).
    $$ (\overline{X}) = \dfrac{(X)}{(XY,XZ)} $$

    \begin{flushright}
        \textbf{Apartado e)}
    \end{flushright}
    
    $$ (\overline{Y}) = \dfrac{(Y)+(XY,XZ)}{(XY,XZ)} = \left( \substack{Obs:\ (XY)\\ \subseteq (Y)} \right) = \dfrac{(Y,XZ)}{XY,XZ} $$
    
    \begin{flushright}
        \textbf{Apartado f)}
    \end{flushright}
    $$ (\overline{Z}) = \dfrac{(Z)+(XY,XZ)}{(XY,XZ)} = \dfrac{(Z,XY)}{(XY,XZ)} $$

    \hrulefill

    Usaremos ahora que $ P  $ es primo $ \iff\ B/P$ es dominio, tomaremos $ P = J/I $ y $ B = A/I $, lo que nos queda:
    
    $ J/I  $ es primo en $ A/I \iff \dfrac{A/I}{J/I}$ es dominio $ \iff\footnote{Segundo teorema de isomorfía} A/J $ dominio. 

    Volvamos ahora a cada caso particular:

    \begin{flushright}
        \textbf{Apartado a)}
    \end{flushright}
    
    $$ \dfrac{(X,Y)}{(XY,XZ)}\ \text{primo }\iff \dfrac{K[X,Y.Z]}{(X,Y)}\cong K[Z]\ dominio $$

    \begin{flushright}
        \textbf{Apartado b,c)}
    \end{flushright}

    Análogos al a).

    \begin{flushright}
        \textbf{Apartado d)}
    \end{flushright}
    $$ \dfrac{K[X,Y,Z]}{(X) \cong_{ejercicio} K[Y,Z]} $$

    \begin{flushright}
        \textbf{Apartado e)}
    \end{flushright}
    
    $$ \dfrac{K[X,Y,Z]}{(Y,XZ)}\ \text{no es dominio porque }\overline{X}\overline{Z} = \overline{0}\ y\  \overline{X}\ne \overline{0} \ne \overline{Z}  $$

    \begin{flushright}
        \textbf{Apartado f)}
    \end{flushright}
    
    Análogo al anterior

\end{exercise}

\textbf{Ejercicio utilizado en el anterior ejercicio:} Sea $ B $ anillo y $ X_1,...,X_n $ variables sobre $ B $. Para cada subconjunto $ J \subset\ \mathbb{N}_{n} = \{1,...,n\} $ consideremos la composición de homomorfismos de anillos:
$$ B[X_i:\ i \in \mathbb{N}_{n}] \hookrightarrow^i B[X_1,...,X_n] \xrightarrow{\pi} \dfrac{B[X_1,...,X_m]}{(X_j: j \in J)} $$

Probar que $ \pi \circ i  $ es un isomorfismo de anillos.


\begin{exercise}
    


    $$ I+(x) = \{a+xf:\ a \in I,\ f \in A[X]\} = \{f \in A[X]:\ g(0) \in I\} $$

    \begin{flushright}
        \textbf{Apartado a)}
    \end{flushright}
    
    Se reduce a probar que cada uno es primo si y solo si $ A/I $ es dominio si y solo si $ A[X]/I[X] $ es dominio si y solo si $ A[X]/I+(x) $ es dominio.

    Tenemos un homomorfismo de anillos:
    $$ \dfrac{A}{I} \to \dfrac{A[X]}{I+(x)}\ tal\ que\ \overline{a}=a+I \hookrightarrow [a] $$

    Claramente está bien definido y es homomorfismo de anillos (conserva suma y multiplicación)
    
\end{exercise}

\end{document}