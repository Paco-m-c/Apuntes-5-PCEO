\documentclass[openany]{book}
\usepackage[utf8]{inputenc}
\usepackage{verbatim}
\usepackage[hypertexnames=false]{hyperref}
\usepackage{amstext} 
\usepackage{array}   
\newcolumntype{C}{>{$}c<{$}} 


%%%%%%%%%%%%%%%%%%%%%%%

%%%%%%%%%%%%%%%%%%%%%%%
% HOLA PACO
% ESTE ES EL ARCHIVO DE LAS DEFINICIONES ESTRUCTURALES
% VERSION 1.1 NOMÁS
%
% AUTOR ORIGINAL:
% EDUARDO (CHITO) BELMONTE GUILLAMÓN
%
% ESTE ARCHIVO ES COMUNISTA, PUEDES COMPARTIRLO SI QUIERES
%%%%%%%%%%%%%%%%%%%%%%%

%----------------------------------
%     PAQUETICOS QUE SE USAN
%----------------------------------

%--------------------------
%    PARA USAR INKSCAPE
%---------------------------
\usepackage{import}
\usepackage{hyperref}
\usepackage{xifthen}
\usepackage{pdfpages}
\usepackage{transparent}

\newcommand{\incfig}[1]{%
    \def\svgwidth{\columnwidth}
    \import{./figures/}{#1.pdf_tex}
}

\newcommand{\custincfig}[2]{%
    \def\svgwidth{#1}
    \import{./figures/}{#2.pdf_tex}
}
\newcommand{\textnexttofig}[3]{
  \begin{minipage}[l]{0.45\textwidth}
    \custincfig{#1}{#2}
  \end{minipage}
  \begin{minipage}[l]{0.45\textwidth}
    #3
  \end{minipage}
}

%%%%%%%%% FIN DEL INKSCAPE

\usepackage{parskip} % Pa parrafos wapos
\setlength{\parindent}{0.5cm} % Pa la sangría
\usepackage{graphicx} % Pa meter las imágenes
\graphicspath{{Images/}} % La ruta a las imágenes

\usepackage{tikz} % Pa dibujar cosichuelas guapas

\usepackage[spanish]{babel} % PA QUE ESTÉ EN ESPAÑOL NOMÁS

\usepackage{enumitem} % Para personalizar las LISTAS YEAH

\setlist{nolistsep} % Pa que las listas estén junticas

\usepackage{booktabs} % Esta sirve para hacer tablas fancy con multicolumns y tal pero no tengo ni puta idea de usarla

\usepackage{xcolor} % PA DEFINIR LOS COLORINES
\definecolor{turquoise}{RGB}{21,103,112} % Es un turquesica así formal
\definecolor{violet}{RGB}{ 110, 6, 187 } % Color maricón

%-------------------------------------------------
%     MÁRGENES
%-------------------------------------------------

\usepackage{geometry}
\geometry{
    top=3cm,
    bottom=3cm,
    left=3cm,
    right=3cm,
    headheight=14pt,
	footskip=1.4cm,
	headsep=10pt,
}

\usepackage{avant} % Esto es una fuente para encabezados

%\usepackage{mathptmx} % Usar simbolitos matemáticos chulos

\usepackage{microtype} % Para fuentes de maricones

\usepackage[utf8]{inputenc} % Pa los acentos

\usepackage[T1]{fontenc}

%-------------------------------------------------
% Bibliografía e índice
%-------------------------------------------------

\usepackage{makeidx} % Pa hacer un índice
\makeindex

\usepackage{titletoc}   % Para manipular la tabla de contenidos

\contentsmargin{0cm}    % Para eliminar el margen por defecto

\usepackage{titlesec} % Pa cambiar los titulos skere

\titleformat
{\chapter} % command
[display] % shape
{\centering\bfseries\Huge\normalfont} % format
{\color{turquoise}  {\normalsize\MakeUppercase{Capítulo} \thechapter }} % label
{-0.5cm} % sep
{
    \color{turquoise}
    \rule{\textwidth}{3pt}
    \vspace{1ex}
    \centering
    \setcounter{ex}{0}
    \setcounter{dummy}{0}
} % before-code
[
\vspace{-0.5cm}%
\rule{\textwidth}{3pt}
] % after-code


\titleformat{\part}
[display]
{\centering\bfseries\Huge\normalfont}
{\color{turquoise} {\normalsize \MakeUppercase{Asignatura}}}
{0pt}
{\color{turquoise}
\vspace{-0.6cm}
\rule{\textwidth}{3pt}
\vspace{1ex}
\setcounter{chapter}{0}
\setcounter{section}{0}
\setcounter{dummy}{0}
\centering
}


\titleformat{\section}
{\normalfont\Large\bfseries}{\color{turquoise}\thesection\ - }{0.5em}{}

\usepackage{fancyhdr}   % Necesario para el encabezado y el pie de página

\pagestyle{fancy}   %Para modificar los encabezados
\fancyhf{}          %Para eliminar los encabezados y pies de página por defecto.
\fancyhead[LE,RO]{\sffamily\normalsize\thepage}
\fancyfoot[C]{Ampliación de Probabilidad}
%HACER

\usepackage{amsmath,amsfonts,amssymb,amsthm,cancel} % PARA LAS MATES

%   LINEA 199, HACER CAPULLADAS

\newtheoremstyle{turquoisebox}
{0pt} %Espacio encima
{0pt} %Espacio abajo
{\normalfont} % Fuente del cuerpo
{} % Cantidad de identado
{\small\ssfamily\color{turquoise}} % Fuente en la que pone "TEOREMA"
{:} % Puntuación tras el teorema
{0.25em} %Espacio tras el teorema
{\thmname{#1}\thmnumber{#2}} %No sé si esto funciona


\newcounter{dummy}[subsection]
\newcounter{ex}
\newtheorem{teoremote}[dummy]{\color{turquoise}Teorema}
\newtheorem{propositiont}[dummy]{\color{turquoise}Proposición}
\newtheorem{lemmat}[dummy]{\color{turquoise}Lema}
\newtheorem{definitionT}{\color{turquoise}Definición}[section]
\newtheorem{exerciseT}[ex]{Ejercicio}
\newtheorem{examplote}[ex]{\color{turquoise}Ejemplo}
\newtheorem{methodT}[dummy]{\color{turquoise}Método}


\RequirePackage[framemethod=default]{mdframed} % Required for creating the theorem, definition, exercise and corollary boxes

%Caja de teoremas

\newmdenv[skipabove=7pt,
skipbelow=7pt,
backgroundcolor=black!5,
linecolor=turquoise,
innerleftmargin=5pt,
innerrightmargin=5pt,
innertopmargin=5pt,
leftmargin=0cm,
rightmargin=0cm,
linewidth=3pt,
innerbottommargin=5pt]{tBox}

\newmdenv[skipabove=7pt,
skipbelow=7pt,
backgroundcolor=black!5,
linecolor=turquoise,
innerleftmargin=5pt,
innerrightmargin=5pt,
innertopmargin=5pt,
leftmargin=0cm,
rightmargin=0cm,
linewidth=1pt,
innerbottommargin=5pt]{pBox}

\newmdenv[skipabove=7pt,
skipbelow=7pt,
backgroundcolor=violet!7,
linecolor=turquoise,
innerleftmargin=5pt,
innerrightmargin=5pt,
innertopmargin=5pt,
leftmargin=0cm,
rightmargin=0cm,
rightline=false,
topline=false,
bottomline=false,
linewidth=4pt,
innerbottommargin=5pt]{mBox}

\newmdenv[skipabove=7pt,
skipbelow=7pt,
rightline=false,
leftline=true,
topline=false,
bottomline=false,
linecolor=turquoise,
innerleftmargin=5pt,
innerrightmargin=5pt,
innertopmargin=0pt,
leftmargin=0cm,
rightmargin=0cm,
linewidth=4pt,
innerbottommargin=0pt]{dBox}

\newmdenv[skipabove=7pt,
skipbelow=7pt,
rightline=false,
leftline=true,
topline=false,
bottomline=false,
backgroundcolor=black!3,
linecolor=turquoise!50,
innerleftmargin=5pt,
innerrightmargin=5pt,
innertopmargin=0pt,
innerbottommargin=5pt,
leftmargin=0cm,
rightmargin=0cm,
linewidth=4pt]{eBox}

\newmdenv[skipabove=7pt,
skipbelow=7pt,
leftline=true,
topline=false,
rightline=false,
bottomline=false,
backgroundcolor=cyan!5,
linecolor=turquoise,
innerleftmargin=5pt,
innerrightmargin=5pt,
innertopmargin=0pt,
innerbottommargin=5pt,
leftmargin=0cm,
rightmargin=0cm,
linewidth=4pt]{exBox}

\newenvironment{theorem}{\begin{tBox}\begin{teoremote}}{\end{teoremote}\end{tBox}}
\newenvironment{proposition}{\begin{pBox}\begin{propositiont}}{\end{propositiont}\end{pBox}}
\newenvironment{lemma}{\begin{pBox}\begin{lemmat}}{\end{lemmat}\end{pBox}}
\newenvironment{method}{\begin{mBox}\begin{methodT}}{\end{methodT}\end{mBox}}
\newenvironment{definition}{\begin{dBox}\begin{definitionT}}{\end{definitionT}\end{dBox}}
\newenvironment{exercise}{\begin{eBox}\begin{exerciseT}}{\hfill{\color{black}}\end{exerciseT}\end{eBox}}
\newenvironment{example}{\begin{exBox}\begin{examplote}}{\end{examplote}\end{exBox}}
\newenvironment{demonstration}{\begin{flushright}
      \color{turquoise} \textbf{Demostración}
\end{flushright}
}{\begin{flushright}
  $\square$
\end{flushright}}

\usepackage{geometry}
\geometry{
    top=3cm,
    bottom=3cm,
    left=3cm,
    right=3cm,
    headheight=14pt, 
    footskip=1.4cm,
    headsep=10pt,
}
\usepackage{graphicx}
\title{Apuntes de Álgebra Conmutativa}
\author{Paco Mora}
\date{\today}

\setcounter{secnumdepth}{0}
\begin{document}
\maketitle

\tableofcontents

\chapter{Tema 1}

\begin{exercise}
    \textbf{Ejercicio Propuesto}

    Sea $ A = \mathbb{Z}_{n} $, con $ n  $ entero >1 y $ \overline{r} \in \mathbb{Z}_{n} $. Demostrar:
    \begin{itemize}
        \item $ \overline{r} $ cancelable $ \iff \overline{r}$ invertible $ \iff $ $ mcd(r,n)=1 $
        \item $ \overline{r} $ nilpotente $ \iff $ todos los divisores primos de $ n $ dividen a $ r $.
    \end{itemize}
\end{exercise}

La siguiente proposición generaliza el ejercicio anterior.

\begin{proposition}
    Sea $ A  $ un anillo finito y sea $ a \in A $. Entonces $ a  $ es cancelable sii es invertible.
\end{proposition}

\begin{demonstration}

    Definimos
    $$ \lambda_n: A \to A\hspace{5mm} \lambda_n(x)=ax\ \forall x \in A$$

    Es inyectiva, $ \lambda_n(x) = \lambda_n(y) \iff ax = ay \implies_{a\ cancel.} x=y $

    Por lo tanto, y como $ A $ es finito, $ \lambda_n $ es biyectiva y $ 1 \in Im(\lambda_n) \iff \exists b \in A\ |\ \lambda_n(b)=1 $
\end{demonstration}


\begin{proposition}
    $ A $ reducido $ \iff $ Nil($ A $) = $ \{\text{elem nilpotentes de}\ A\} = \{0\} $
\end{proposition}

\begin{demonstration}

    $ \ \implies$

    $ A $ reducido sii $ \forall a \in A $, $ a^2=0 \implies a=0 $

    $ \impliedby $

    Por reduc. al absurdo, supongamos $ b \in Nil(A) \setminus \{0\}  \implies \exists n >0 $ (mínimo) con $ b^{n}=0 \implies b^{n-1} \ne 0 $

    Pero entonces, $ (b^{n-1})^{2} = b^{2n-2}=0  $ y $ 2n-2\geq n  $ para $ n\geq 2 $, luego llegamos a una contradicción.
\end{demonstration}

\begin{exercise}
    \textbf{Ejercicio Propuesto}

    $ \mathbb{Z}_{n}$ es un anillo reducido $ \iff $ $ n $ es libre de cuadrados.

\end{exercise}\vspace{5mm}
\textbf{Demostración del 1.9(ii)}

\begin{demonstration}
    $ a/b $ y $ a/c \implies \exists b',c' \in D /\ ab'=b,\ ac'=c$\dots
    Sean ahora $ r,s \in D $ arbitrarios y veamos que $ a/rb+sc $
    $$ rb+rc=r(ab)+s(ac') = arb' =  asc' = a(rb'+sc') \implies a | rb+sc \implies \cancel{b}1 = \cancel{b}(dc) $$
\end{demonstration}

% \begin{flushright}
%     \textbf{NOTA}
% \end{flushright}

% Hay muchos ejemplos de subconjuntos de $ G \subset A  $ tal que $ G $ no es un ideal.

\begin{exercise}
    \textbf{Ejercicio propuesto}

    Sean $ G_1,G_2 \subset A $. Demostrar que $ (G_1)(G_2)= (G_1\cdot G_2) $. En particular, el producto de ideales principales es un ideal principal.
\end{exercise}

\begin{flushright}
    \textbf{Observación}

\end{flushright}


$ IJ \subset I \cap J $ (estricto en general: $ A = \mathbb{Z},\ I =(2),\ J=(4),\ IJ=(8),\ I\cap J = (4) $)


\begin{example}
    \textbf{Aplicación del teorema de la correspondencia}

    Los ideales de $ \mathbb{Z}_{n} $ están en correspondencia con los divisores positivos de $ n $.

    $$ \mathcal{L}(\mathbb{Z}_{n}) \to \{d>0:\ d/n\} $$

    Pero los ideales de $ \mathbb{Z}_{n} $ son isomorfos a $ \{I \unlhd \mathbb{Z}\ :\ n\mathbb{Z} \subset  I\}$ por el teorema de la correspondencia, entonces:
    $$\{I \unlhd \mathbb{Z}\ :\ n\mathbb{Z} \subset  I\} = \{d\mathbb{Z} \unlhd \mathbb{Z}:\ n\mathbb{Z} \subset  d\mathbb{Z}\} = \{d\mathbb{Z}:\ d|n\}\cong \{d>0:\ d/n\}$$
\end{example}

\begin{proposition}
    \textbf{Proposición 1.31 extendido} (la prueba es la de los apuntes)
    Sean $ A,B_1,...,B_n $ anillos y sean $ g_{i}:A\to B_i $ homomorf. de anillos.
    \begin{enumerate}
        \item $ \phi:A\to B_1\times ...\times B_n $, dado por $ \phi(a) = (g_1(a),...,g_n(a)) $ es un homomorf. de anillos con núcleo $ \bigcap_{i=1}^{n}\operatorname{Ker}(g_i) $
        \item Si los $ \operatorname{Ker}(g_i) $ son comaximales dos a dos, entonces se verifica:
            \begin{enumerate}
                \item $ \operatorname{Im}(\phi) = \operatorname{Im}(g_1)\times...\times \operatorname{Im}(g_n)$
                \item $ \operatorname{Ker}(\phi) = \operatorname{Ker}(g_1)\cdots \operatorname{Ker}(g_n) $
                \item Se tiene un isom. de anillos: $ \dfrac{A}{\operatorname{Ker}(g_1)\cdots\operatorname{Ker}(g_n)} \cong \operatorname{Im}(g_1) \times ...\times \operatorname{Im}(g_n) $
            \end{enumerate} 
    \end{enumerate}

\end{proposition}

\begin{demonstration}    
    1.
    $$ \operatorname{Ker}(\phi) = \{a \in A:\ (g_1(a),...,g_n(a)) = (0,...,0)\} = \{a \in A:\ g_i(a) = 0\ \forall i \} = \cap_{i=1}^{n}\operatorname{Ker}(g_i)$$
    2.\\2.b
    
    Si los $ \operatorname{Ker}(g_i) $ son comaximales dos a dos entonces:
    $$ \operatorname{Ker} (\phi) = \operatorname{Ker}(g_1)\cdots \operatorname{Ker}(g_n) $$

    Con lo que tenemos 2b).\\
    2.a

    Si $ (b_1,...,b_n)  \in \operatorname{Im}(\phi) \implies (b_1,...,b_n) = \phi(a) = (g_1(a),...,g_n(a))$ para algún $ a \in A \implies b_i \in \operatorname{Im}(g_i)\ \forall i$. Por tanto, $ (b_1,...,b_n) \in \operatorname{Im}(g_1)\times...\times \operatorname{Im}(g_n) $

    Si probamos ahora que $ (0,...,x_i,0,...,0) \in \operatorname{Im}(\phi)\ \forall  x_i \in \operatorname{Im}(g_i) $, entonces toda $n$-upla $(x_1,...,x_n) \in \operatorname{Im}(\phi)  $ en $ \operatorname{Im}(\phi_1)\times...\times \operatorname{Im}(\phi_n) $. Como los núcleos son comaximales dos a dos.

    $$ \operatorname{Ker}(g_i) + \left( \cap_{j\ne i} \operatorname{Ker}(g_j) = A \implies 1 = a+b,\ a \in \operatorname{Ker}(g_i),\ b \in \cap_{j\ne i} \operatorname{Ker}(g_j) \right) $$

    Como $ x_i \in \operatorname{Im}(g_i) \implies \exists u \in A:\ g_i(u) = x_i $, entonces:
    $$ x_i = 1\cdot x_i = (a+b) g_i(u) = g_i((a+b)u) $$

    Luego entonces:
    $$ \phi(bu) = (g_1(bu),...,g_i(bu),...,g_n(bu)) = (0,...,0,g_i(bu),0,...,0) $$
    $$ x_i = g_i(u) = g_i(au+bu) = \cancel{g_i(a)g_i(u)}+g_i(bu) $$
    
    Con lo que queda demostrado 2.b.\\
    2.c.

    Basta utilizar 2.a), 2.b) y el primer teorema de isomorfía.

\end{demonstration}

\begin{definition}
    \textbf{Conjunto inductivo}

    Un \textbf{conjunto inductivo} es un conjunto ordenado $ S $ tal que todo subconjunto totalmente ordenado no vacío tiene una cota superior en $ S $
\end{definition}

\begin{lemma}
    \textbf{Lema de Zorn}

    Todo conjunto inductivo no vacío tiene un elemento maximal.
\end{lemma}

\begin{demonstration}
    Fijemos $ I  \trianglelefteq A,\ I \ne A $ ideal propio.
    $$ S_{I} = \{ J\trianglelefteq A:\ J \text{ ideal propio e }I \subset  J  \} $$

    $ S_{I} $ es inductivo y $ \ne \emptyset (I \in S_{I})$

    Sea $ Y $ un subconjunto totalmente ordenado $ \ne \emptyset $ de $ S_{I} $. Tomo $ m = \bigcup_{J \in T} J $. Porbemos que $ m $ es un ideal propio tal que $ I \subset  m $. Lo que implica que $ m \in S_{I} $.

    Sean $ a,b \in m \implies \left\{
    \begin{array}{l}
        a \in \bigcup_{J \in T}J \iff \exists J \in T:\ a \in J\\
        b \in \bigcup_{J \in T}J \iff \exists J' \in T:\ b \in J'\\
    \end{array}
    \right.
    $ 

    Si tomamos por ejemplo que $ J \subset  J' $, entonces $ a,b \in J' \implies a-b \in J' \implies a-b \in m $

    Notemos entonces que un elemento maximal de $ S_{I} $ es también un ideal maximal.
\end{demonstration}


\begin{exercise}
    $ I,P \unlhd A,   $ siendo $ P $ primo.  Probar que existe un primo minimal sobre $ I $, pongamos $ q $ tal que $ q \subset P $
\end{exercise}

\begin{lemma}
    \textbf{Lema de Krull}

    $ A $ anillo, $ I \unlhd $ A y $ S \subset A $ un subconjunto multiplicativo. Suponemos que $ I \cap S = \emptyset $ y consideremos $ \mathcal{L}_{I,S} = \{J \unlhd A:\ I \subset  J,\ J \cap S = \emptyset\} $. Se verifica:
    \begin{enumerate}
        \item $ \mathcal{L}_{I,S}  $ es un conjunto inductivo.
        \item Cualquier elemento maximal de $ \mathcal{L}_{I,S} $ es un ideal primo.
    \end{enumerate}

\end{lemma}

\begin{demonstration}
    1.

    Hemos de probar que si $ \mathcal{J} \subset \mathcal{I}_{I,S}  $ es un subconjunto totalmente ordenado $ \ne \emptyset \implies $ tiene una cota superior en $ \mathcal{L}_{I,S} $.
    
    Habría que comprobar que $ \widetilde{J} = \bigcup_{J \in \mathcal{J}}J $ es un ideal.

    Como tenemos que $ I \subset  \widetilde{J} $ y $ S \cap \widetilde{J} = S \cap (\bigcup J) = \bigcup_{J \in \mathcal{J}} (S\cap J) = \emptyset $

    Entonces $ \widetilde{J} $ es una cota superior de $ \mathcal{J} $ en $ \mathcal{L}_{I,S} $.\\
    2.

    Sean $ a,b \in A $ tales que $ ab \in P $. Por reducc. al absurdo, supongamos que $ a \not \in P $ y $ b \not \in P $. Entonces:
    $$ \left\{
    \begin{array}{l}
        P \subsetneq P + (a)\\
        P \subsetneq P + (b)
    \end{array}
    \right\} \implies P+(a),P+(b) \not \in \mathcal{L}_{I,S} \iff \left\{
    \begin{array}{l}
        (P+(a)) \cap S \ne \emptyset\\
        (P+(b)) \cap S \ne \emptyset
    \end{array}
    \right\} $$

    Sean entonces $ s \in (P+(a)) \cap S  $ y $ s' \in (P+(b))\cap S $. Entonces:
    $$ \left\{
    \begin{array}{l}
        s = p+ar\\
        s'=p'+br'
    \end{array}
    \right.\hspace{10mm} p,p' \in P,\ r,r' \in A $$

    $$ ss' = (p+ar)(p'+br') = pp'+pbr' + arp' + abrr' \in P \implies P \cap S \ne \emptyset $$

    Con lo que llegamos a una contradicción
\end{demonstration}


\begin{proposition}
    Sea $ A  $ un anillo e $ I \unlhd A $ un ideal \textbf{propio}. Son equivalentes:
    \begin{enumerate}
        \item Si $ a \in A $ y $ a^{n} \in I $, para algún $ n>0 $, entonces $ a \in I $
        \item Śi $ a \in A $ y $ a^2 \in I $, entonces $ a \in I $
        \item $ I $ es una intersección de ideales primos.
        \item $ I $ es la intersección de los ideales primos minimales sobre $ I $.
    \end{enumerate}

\end{proposition}

\begin{demonstration}
    $ 1\implies 2. $

    Directa.\\
    $ 2\implies 1 $

    Si $ n = 1 \implies a'=a \in I$, podemos suponer que $ a \not \in I $ y que existe $ n > 1 $, $ a^{n} \in I $ tal que $ a^{n-1}\not \in I $. Entonces tenemos:
    $$ (a^{n-1})^2 = a^{2n-2} = \underbrace{a^{n}}_{\in I}\underbrace{a^{n-2}}_{\in A} \implies (a^{n-1})^2 \in I \implies a^{n-1} \in I$$

    Con lo que tenemos una contradicción y $ a \in I $.\\
    $ 4\implies 3. $

    Directa.\\
    $ 3\implies 4. $

    Supongamos que $ \exists (P_{\lambda})_{\lambda \in \Lambda} $ ideales primos tales que $ I = \bigcap_{\lambda \in \Lambda}P_{\lambda} $

    $$ \forall  \lambda \in \Lambda,\ I \subset P_{\lambda} \implies\footnote{Por el último ejercicio propuesto.} \exists Q_{\lambda} \text{ primo minimal sobre $ I $ tal que } I \subset  Q_{\lambda} \subset  P_{\lambda}\implies  $$
    $$ \implies I \subset  \bigcap_{\lambda \in \Lambda} Q_{\lambda} \subset  \bigcap_{\lambda \in \Lambda} P_{\lambda} = I \implies I = \cap _{\lambda \in Q_{\lambda}}  $$
    $$ I \subset  \bigcap _{\substack{Q \in \operatorname{Spec}(A)\\ Q\ minimal\ I}} Q \subset  \bigcap _{\lambda \in \Lambda}Q _{\lambda} = I $$

    Con lo que tenemos 4.\\
    $ 3\implies 2. $

    Si $ a ^2 \in I = \bigcap_{\lambda \in \Lambda}P_{\lambda} \iff a^2 \in P_{\lambda},\ \forall  \lambda \in \Lambda \implies a \in P_{\lambda},\ \forall \lambda \in \Lambda \iff a \in \bigcap _{\lambda \in \Lambda }P_{\lambda } = I $\\
    $ 1\implies 4. $

    Sean $ \mathcal{Q} = \{\text{ideales primos minimales sobre }I\} $. Queremos probar que $ I = \cap_{Q \in \mathcal{Q}}Q $.  La inclusión $ \subset  $ es directa.

    Supongamos ahora que $ I\subsetneq \bigcap _{Q \in \mathcal{Q}}Q \implies  $ tomamos $ x \in \cap_{Q \in \mathcal{Q}}Q  $ tal que $  x \not \in I $.

    Como $ x \not \in I \implies x^{n}\not \in I,\ \forall n \geq  0$. Aplicamos ahora el lema de Krull con $ I $ y $ S = \{x^{n}:\ n\geq  0\} $. 

    Entonces $ \mathcal{L}_{I,S} = \{J \unlhd A:\ I \subset  J,\ J \cap S = \emptyset\} $ tiene un elemento maximal, pongamos $ P $, que es primo. Entonces:
    $$ \left\{
    \begin{array}{l}
        S \cap P = \emptyset\\
        I \subset  P 
    \end{array}
    \right\} \implies\footnote{Por el ejercicio de nuevo.} \exists Q \text{ primo minimal sobre } I:\ I \subset  Q' \subset  P \implies S \cap Q' = \emptyset$$

    Con lo que llegamos a una contradicción porque $ x \in Q' $

\end{demonstration}

\begin{definition}
    \textbf{Ideal radical}

    Un ideal que cumpla las condiciones de la anterior proposición se dice que es \textbf{radical}.

\end{definition}


\begin{definition}
    \textbf{Radical de un ideal}

    Sea $ I \unlhd A $ ideal propio, $ \sqrt{I}:= \{x \in A:\ x^{n}\in I,\ \text{ para algún }n>0\} $
\end{definition}

\begin{proposition}
    \textbf{Sustituye al Corolario 1.4.6}

    Dado $ I \properideal A $ ideal propio, el subconjunto $ \sqrt{I}  $ es un ideal radical de $ A $ y puede ser descrito por cada una de las siguientes formas equivalentes:
    \begin{enumerate}
        \item El menor ideal radical que contiene a $ I $.
        \item La intersección de todos los ideales radicales que contienen a $ I $.
        \item La intersección de todos los ideales primos que contienen a $ I $.
        \item La intersección de todos los ideales primos minimales que contienen a $ I $.
    \end{enumerate}
\end{proposition}


\begin{demonstration}
    Vemos primero que $ \sqrt{I} $ es un ideal radical de $ A $.

    Hemos de probar:
    $$ \left\{
    \begin{array}{l}
        \left.
        \begin{array}{l}
            a)\ x+y \in \sqrt{I}\ \forall x,y \in \sqrt{I}\\
            b)\  ax \in \sqrt{ I}\ \forall x \in \sqrt{I},\ a \in A
        \end{array}
        \right\}\ ideal\\
        c)\ Si\ a^{n}\in \sqrt{I},\ con\ n>0 \implies a \in \sqrt{I}
    \end{array}
    \right. $$

    Vemos en primer lugar b):

    $$(ax)^{n} = a^{n}x^{n} \implies (Como\  x^{n} \in I,\ a^{n}x^{n}\in I) \implies (ax)^{n} \in I \implies ax \in \sqrt{I} $$

    a) se demuestra utilizando el binomio de Newton:
    $$ y,x \in \sqrt{I} \implies \exists m,n>0:\ x^{m} \in I,\ y^{n} \in I $$

    Sin pérdida de generalidad, supongamos $ m = n $
    $$ (x+y)^{2n} = \sum\limits_{i=0}^{2n} \binom{2n}{i} x^{i}y^{2n-i} \in I \implies x+y \in I $$

    Para ver c), sea ahora $ a^{n}\in \sqrt{I} \implies \exists m >0:\ (a^{n})^{m} \in I \implies a^{nm}\in I \implies a \in \sqrt{I} $.

    Con lo que $ \sqrt{I} $ es un ideal radical.\\
    1.

    Sea $ J \unlhd A$ ideal radical y propio tal que $ I \subset J $. Queremos ver que $ \sqrt{I}\subset J $.

    Sea $ x \in \sqrt{I} \implies \exists n> 0:\ x^{n}\in I\implies x^{n}\in J \implies_{J\ radical}x \in J $\\
    2.

    Es consecuencia inmediata de 1.\\
    3.

    Sea $ \mathcal{V}(I) = \{P \in \operatorname{Spec}(A):\ I \subset P\}\implies? \sqrt{I} = \bigcap_{P \in \mathcal{V}(I)}P $.

    La inclusión $ \subset  $ es directa con la afirmación 1 y por ser la intersección un ideal radical. Para la otra, sabemos que $ \sqrt{I} =  $ intersección de los ideales primos minimales sobre $ \sqrt{I} $. Entonces:
    $$ \sqrt{I} = \bigcap _{\substack{Q \in \operatorname{Spec}(A)\\ Q\ minimal / \sqrt{I}}} Q \supseteq \bigcap_{P \in \mathcal{V}(I)}P  $$

    Luego ya tenemos la igualdad.\\
    4.

    Se demuestra aplicando el ejercicio.

\end{demonstration}

\begin{example}
    Tomamos el caso $ (I) = 0 $
    $$ \sqrt{(0)} = \{x \in A:\ x^{n} = 0\} = \{\text{nilpotentes de } A\} =: \operatorname{Nil}(A) $$

    \begin{enumerate}
        \item $ \operatorname{Nil}(A) $ es el menor ideal radical de $ A $ 
        \item $ \operatorname{Nil}(A) $ es la intersección de todos los ideales radicales de $ A $.
        \item $ \operatorname{Nil}(A) $ es la intersección de todos los ideales primos de $ A $.
        \item $ \operatorname{Nil}(A) = \bigcap_{P \in \operatorname{MinSpec}(A)}P$
    \end{enumerate}
\end{example}

\section{Ejercicios}


\setcounter{ex}{1}
\begin{exercise}
    $$ x,y \in \mathcal{U}(A)\implies xyy ^{-1}x ^{-1} = 1 \implies xy \in \mathcal{U}(A) $$

    $$  xy \in \mathcal{U}(A  ) \implies \exists w \in A:\ xyw = 1 \implies \left\{
    \begin{array}{l}
        x ^{-1}= yw \\
        y ^{-1} = wx
    \end{array}
    \right. $$

\end{exercise}


\begin{exercise}$  $

    \begin{flushright}
        \textbf{En este ejercicio hay una errata, está por solucionar}
    \end{flushright}

    Sabemos que en un anillo finito, las unidades y los elementos cancelables son los mismos. Luego $ |\mathcal{U}(\mathbb{Z}_{n})| = |\{cancelables\}|$. Además sabemos que $ |\{divisores\ de\ cero\}| = n - |\mathcal{U}(\mathbb{Z}_{n})|  $ . Además sabemos que:
    $$ |\mathcal{U}(\mathbb{Z})_{n}| = \phi(n ) = p_1^{\alpha_1-1}\cdots p_{r}^{\alpha_{r}-1} (p_1-1)\cdots (p_{r}-1) $$

    Entonces,
    $$ |\{divisores\ de\ cero\}| = p_1^{\alpha_1-1} \cdots p_{r}^{\alpha_{r}-1} (p_1\cdots p_{r}- \prod_{i=1}^{r}(p_i-1)) $$

    Vemos entonces el cardinal de $ \operatorname{Nil}(\mathbb{Z}_{n}) $:

    $$ \overline{k} = k+n\mathbb{Z} \in  \operatorname{Nil}(\mathbb{Z})_{n} \iff \text{todos los $  p_i$ dividen a }k $$
    $$ \overline{k} \in \operatorname{Nil}(\mathbb{Z})_{n} \iff \exists t>0:\ \overline{k}^{t} = \overline{0}\ en\ \mathbb{Z}_{n}\iff \exists t>0:\ n/k^{t} \implies \text{todos los $ p_i $ dividen a }k$$

    Recíprocamente:
    $$ k = p_1^{\beta_1}\cdots p_{r}^{\beta_{r}},\ con\ 0<\beta_i \leq  \alpha_i\ \forall i = 1,...,r $$

    $$ |\operatorname{Nil}(\mathbb{Z})_{n}| = \alpha_1\cdots \alpha_{r} $$
\end{exercise}


\begin{exercise}
    $$ \mathcal{U}(\mathbb{Z}_{24}) = \{\overline{k}:\ \operatorname{mcd}(k,n) = 1\} = \{cancelables\} = \{\overline{1},\overline{5},\overline{7},\overline{11},\overline{13},\overline{17},\overline{19},\overline{23}\} $$
    $$ \{\text{divisores de cero}\} = \mathbb{Z}_{24} \setminus \mathcal{U}(\mathbb{Z}_{24}) $$


\end{exercise}

\setcounter{ex}{5}

\begin{exercise}
    $  $
    
    Recordemos primero que $ p \in A $ es primo sii $ (p) $ es un ideal primo.

    $ f:A \to B\ homomorf $. Si $ a $ satisface $ (P) $,¿ $ f(a) $ cumple $ (P) $?
    
    \begin{flushright}
        \textbf{Apartado a)}
    \end{flushright}
    
    Si $ a \in \mathcal{U}(A) \implies \exists a ^{-1} \in A:\ a\cdot a ^{-1} = 1 \implies f(a)f(a ^{-1}) = f(1) = 1 \implies f(a) \in \mathcal{U}(B)$.

\begin{flushright}
    \textbf{Apartado b)}
\end{flushright}

    % Encontraremos un contraejemplo, sea $ f: \mathbb{Z} \to \mathbb{Z}_{6} $ con la proyección canónica, basta tomar $ a = 3 $


Tomando $ \mathbb{Z} \to \dfrac{\mathbb{Z}[X]}{(2x)} $ homomorfismo inyectivo. El 2 es cancelable en $ \mathbb{Z} $ pero no lo es en el anillo destino.


\begin{flushright}
    \textbf{Apartado c)}
\end{flushright}

Sea $ a \in A $ divisor de 0 $ \implies \exists b \in A \setminus \{0\}:\ ab = 0 \implies f(a)f(b) = 0  $

Cuando $ f $ es inyectiva: sí, porque $ f(b)  \ne 0 $. En otro caso:

Sean $ m,n > 1,\ mn\mathbb{Z} \subset  n\mathbb{Z} \implies  $ tomamos un homomorfismo de anillos suprayectivo:

$$ \dfrac{\mathbb{Z}}{mn\mathbb{Z}} \to \mathbb{Z}\dfrac{\mathbb{Z}}{n\mathbb{Z}} $$

Tomando $ m $, $ n $ tales que $ \operatorname{mcd}(n,m) = 1 $ tenemos que $ \overline{m} $ es divisor de cero pero su imagen, $ [m] \in \mathcal{U}(\mathbb{Z}_{n}) $

\begin{flushright}
    \textbf{Apartado d)}
\end{flushright}

Si $ a \in A $, existe un exponente $ n > 0 $ tal que  $ a^{n} = 0 \implies f(a)^{n} = f(a^{n}) = 0 $, entonces $ f(a) $ es nilpotente.

\begin{flushright}
    \textbf{Apartado e)}
\end{flushright}

De forma parecida al apartado anterior, vemos que si $ e = e^2  $ en $ A $, al aplicar $ f $ tenemos que $ f(e) = f(e)^2 \implies f(e) $ es idempotente.

\begin{flushright}
    \textbf{Apartado f)}
\end{flushright}

Basta tomar la inclusión de $ \mathbb{Z}  $ en $ \mathbb{Q} $ para tener un contraejemplo (no suprayectivo). Para el caso suprayectivo planteamos un ejercicio:

\textbf{Ejercicio:} Sea $ \overline{k} = kp^{t}\mathbb{Z} $ es irreducible en $ \mathbb{Z}_{p^{t}} \iff \overline{k} = \overline{p} \overline{u}$, siendo $ \overline{u} \in \mathcal{U}(\mathbb{Z}_{p^{t}}) $. Más generalmente: Sea $A $ un anillo y  $ p \in A $ tales que $ (p)  $ es el único ideal maximal de $ A $- Entonces los elementos irreducibles de $ A $ son los de la forma $ pu $, siendo $ u \in \mathcal{U}(A) $ ($ p $ es el único irreducible de $ A $ salvo asociados)

Construimos en base a este ejercicio el homomorfismo suprayectivo formado por la proyección $ \mathbb{Z} \to \mathbb{Z}_{p^{t}} $. Dado $ q \ne p $ primo, su imagen es $ \overline{q} \in \mathcal{U}(\mathbb{Z}_{p^{t}}) \implies \overline{q}  $ no es irreducible.

\begin{flushright}
    \textbf{Apartado g)}
\end{flushright}

\textbf{Ejercicio:} Sea $ A $ un dominio y $ p \in A $. Si $ p $ es primo entonces es irreducible. Cuando $ A $ es un DIP, se verifica también el recíproco.

Como los contraejemplos del apartado anterior parten de $ \mathbb{Z} $ y los irreducibles y los primos son iguales en $ \mathbb{Z} $, podemos usar los mismos contraejemplos en este apartado.

\hrulefill

Vamos a resolver ahora el primero de los ejercicios planteados:

\textbf{Ejercicio:} Sea $ \overline{k} = kp^{t}\mathbb{Z} $ es irreducible en $ \mathbb{Z}_{p^{t}} \iff \overline{k} = \overline{p} \overline{u}$, siendo $ \overline{u} \in \mathcal{U}(\mathbb{Z}_{p^{t}}) $. Más generalmente: Sea $A $ un anillo y  $ p \in A $ tales que $ (p)  $ es el único ideal maximal de $ A $- Entonces los elementos irreducibles de $ A $ son los de la forma $ pu $, siendo $ u \in \mathcal{U}(A) $ ($ p $ es el único irreducible de $ A $ salvo asociados)

Dado $ p = ab $, veamos si $ p $ es irreducible. Supongamos que $ a \not \in \mathcal{U}(A) \implies (a) \unlhd A \implies (a) \subset  (p)\ \text{porque $ (p) $ es el único ideal maximal.} \implies a= pa',\ siendo\ a' \in A $

$$  \implies p = ab = pa'b \iff p(1-a'b) = 0 \left\{
\begin{array}{l}
    1-a'b \in \mathcal{U}(A) \text{ no, porque implicaría}\\ \hspace{5mm} \text{una contradicción } (p=0) \\
    1-a'b \not \in \mathcal{U}(A)
\end{array}
\right. $$

$$ 1-a'b \not \in \mathcal{U}(A) \implies (1-a'b) \subset (p),\ \text{pero no puede darse $(a'b) \subset (p)$, porque tendríamos }$$
$$ 1 = 1-a'b+a'b \in (p) \implies a'b \in \mathcal{U}(A) \implies b \in \mathcal{U}(A)$$

Sea $ q \in A $ irreducible $ \implies q \not \in \mathcal{U}(A) \iff (q) \properideal A \implies (q) \subset (p) \implies q=pu$, para algún $ u \in A $

\hrulefill

Vemos ahora los recíprocos.

\begin{flushright}
    \textbf{Apartado a)}
\end{flushright}

La inclusión de $ \mathbb{Z} $ a $ \mathbb{Q} $ y tomando $ a = f(a) = 3 $ tenemos un contraejemplo no suprayectivo, para el sobre, tomamos la proyecctión de $ \mathbb{Z} $ en $ \mathbb{Z}_{3} $.

\begin{flushright}
    \textbf{Apartados b,c)}
\end{flushright}

Basta aplicar el contrarrecíproco de $ f(a)\ cancelable \implies a\ cancelable $ y $ f(a)\ divisor\ de\ 0 \implies a\ divisor\ de\ 0 $

\begin{flushright}
    \textbf{Apartado d)}
\end{flushright}

$ f(a) $ es nilpotente  $\iff f(a) $ tal que $ \exists n>0 $ tal que $ f(a)^{n} = 0 \implies f(a^{n}) = 0\iff a^{n} \in \operatorname{Ker}(f) $.

Si $ f $ es inyectiva, sí se cumple la cadena de sii.

Si $ f $ es sobre, tomamos el contraejemplo de la proyección de $ \mathbb{Z} $ en $ \mathbb{Z}_{n} $ con un producto de primos

\begin{flushright}
    \textbf{Apartado e)}
\end{flushright}

De forma parecida al apartado anterior:

$$ f(a) = f(a^2)\iff a-a^2 \in \operatorname{Ker}(f) $$

Si $ f $ es inyectiva, sí se cumple. 

En el caso sobre, tomamos la proyección de $ \mathbb{Z} $ en $ \mathbb{Z}_{6} $, entonces 7 no es idempotente y $ f(7) = \overline{1} $ no lo es.

\begin{flushright}
    \textbf{Apartado f)}
\end{flushright}


% $ a $ no puede ser una unidad, porque lo sería también $ f(a) $.

% Análogamente $ a $ no puede ser 0, porque lo sería también $ f(a) $.

% Si $ a=bc \implies f(a) = f(b)f(c) \implies f(b) \in \mathcal{U}(B)\ \acute o f(c) \in \mathcal{U}(B)$

Para el caso sobre, tomamos la aplicación $ \mathbb{Z} \to \mathbb{Z}_{p^{t}} $ y el elemento $ (p^{t}+1)p \leadsto \overline{p} $

La idea para obtener el caso inyectivo es tomar un elemento como $ 2\cdot 3 $ no irreducible, y llevar uno de sus factores a una unidad. Tomamos la aplicación:
$$ \mathbb{Z} \hookrightarrow \mathbb{Z}\left[\dfrac{1}{2}\right] = \{q \in \mathbb{Q}:\ q = \dfrac{m}{2^{r}},\ m \in \mathbb{Z},\ r\geq  0\} $$

Dejamos como ejercicio ver que 3 es irreducible en $ \mathbb{Z}[1/2] $

\end{exercise}

\setcounter{ex}{6}

\begin{exercise}
    $  $
    \begin{flushright}
        \textbf{Apartado a)}
    \end{flushright}
    Si $ m<0 \implies \mathcal{U}(\mathbb{Z}(\sqrt{m}))$ es finito. 
    $$ N(a+b\sqrt{m}) = 1\iff a^2-mb^2 = 1 \iff a^2+b\sqrt{-m}^2 = 1 $$

    $ \implies (a,b\sqrt{-m}) $ está en la circunferencia de centro $ (0,0) $ y radio 1 y su 1ª componente $ a $ es entera.

    $$ \implies \mathcal{U}(Z(\sqrt{m})) \subset  \{a+b\sqrt{m}:\ (a,b\sqrt{-m}) \in \{(1,0),(-1,0),(0,1),(0,-1)\}\} $$

    \begin{flushright}
        \textbf{Observación}

    \end{flushright}

    $$ a = 0\iff b\sqrt{-m} = \pm 1 \implies \left\{
    \begin{array}{l}
        b = \pm 1\\ \sqrt{-m} = 1
    \end{array}
    \right. 
    \implies -m = 1 \implies m = -1 $$

    Luego $ \mathcal{U}(\mathbb{Z}[\sqrt{m}]) = \{-1,1\} $ salvo cuando $ m = -1 $ en que $ \mathcal{U}(\mathbb{Z}(i)) = \{1,-1,i,-i\} $


    \begin{flushright}
        \textbf{Apartado b)}
    \end{flushright}
    
    % Si $ m > 0 $y $ |\mathcal{U}(\mathbb{Z}[\sqrt{m}])| > 2 \implies |\mathcal{U}(\mathbb{Z}[\sqrt{m}])| = \infty $

    Supongamos que $ |\mathcal{U}(\mathbb{Z}[\sqrt{m}]) | > 2 $ y cojamos $ \alpha = a +b\sqrt{m} \ne \pm 1 $

    Tomamos $X:= \{1,\alpha,\alpha^2,...\} =  $ el subgrupo multiplicativo de $ \mathcal{U}(\mathbb{Z}[\sqrt{m}])$ generado por$ \alpha $.

    Si $ X $ es finito $ \implies \exists n > 0 :\ \alpha ^{n} = 1 $. Elegimos $ n $ mínimo con esa propiedad $  \implies  $ $ \alpha  $ raíz $ n- $ésima (primitiva)  de 1.

    Como $ m > 0 \implies \alpha = a+b\sqrt{m} \in \mathbb{R} \implies  \alpha = \pm 1\ \text{(contradice que el que }\alpha \ne \pm 1) $

    \begin{flushright}
        \textbf{Apartado c)}
    \end{flushright}
    
    Por las conclusiones tomadas en el apartado a). Se tiene que $ \mathcal{U}(\mathbb{Z}(\sqrt{-11})) = \{1,-1\} $

    Se trata de ver ahora que $ x = 1+\sqrt{-11} $ e $ y = 1-\sqrt{-11} $ son irreducibles. Como son conjugados, bastará con ver que uno solo de ellos es irreducible.

    En primer lugar, no es cero ni una unidad. Pongamos $ x = (a+b\sqrt{-11})(c+d\sqrt{-11}) $. Tomando normas:
    $$ 12 = N(x) = N(a+b\sqrt{-11})N(c+d\sqrt{-11}) $$

    Si ni $ a+b\sqrt{-11} $ ni $  c+d\sqrt{-11}$ son unidades $ \implies $ las combinaciones posibles de normas son $ (2,6),(3,4),(4,3),(6,2) $. En cualquier caso, la norma de uno de ambos es 2 o 3. Sin pérdida de generalidad, vamos a suponer que la norma de $ (a+b\sqrt{-11}) $ es 2 o 3, en cualquiera de los casos un primo $ p $.

    $$ \{2,3\} \ni p = N(a+b\sqrt{-11}) = a^2+11b^2 \implies_{b\ entero} b = 0  \implies a^2 = p $$

    Lo cual es imposible porque $ a $ es entero, hemos llegado a una contradicción  y $ x $ es irreducible.

    Ahora tenemos que $ xy = (1+\sqrt{-11})(1-\sqrt{-11}) = 12 = 2\cdot 2\cdot 3 $

    Basta ver ahora que 2 y 3 son irreducibles en $ \mathbb{Z}[\sqrt{-11}] $
    $$  p = (a+b\sqrt{-11})(c+d\sqrt{-11}) \implies_{tomando\ normas} p^2 = N(a+b\sqrt{-11})N(c+d\sqrt{-11})  $$

    Supongamos que ninguno de estos dos es 1, tenemos que $ N(a+b\sqrt{-11}),N(c+d\sqrt{-11}) = p $ y aplicando un razonamiento como el anterior, tenemos que es imposible y entonces $ p $ es irreducible.

    \begin{flushright}
        \textbf{Apartado d)}
    \end{flushright}
    
    Nos preguntamos si cuando un primo entero $ p>1 $, ¿ es irreducible en $ \mathbb{Z}[\sqrt{-3}] $?

    Tomamos una factorización $ p = (a+b\sqrt{-3})(c+d\sqrt{-3}) $ y tomamos normas:
    $$ p^2 = N(a+b\sqrt{-3})N(c+d\sqrt{-3}) $$

    Entonces tenemos:
    $$ p\ \text{irreducible} \iff N(a+b\sqrt{-3}) = 1\ \acute o N(c+d\sqrt{-3}) = 1 $$
    $$ p\ \text{no es irreducible}\iff N(a+b\sqrt{-3}) = p = N(c+d\sqrt{-3}) \iff\footnote{utilizando la ecuación de antes y que $  \mathbb{Z} $ es un dominio} N(a+b\sqrt{-3}) = p $$

    Como conclusión, tenemos que $ \mathbb{Z} $ es irreducible en $ \mathbb{Z}[\sqrt{-3}] $ sii la ecuación $ x^2+3y^2 = p $ no tiene solución en $ \mathbb{Z} \times \mathbb{Z}  $.

    Basta aplicar ahora este resultado a los 4 números a los que nos piden comprobar si son o no irreducibles.

\end{exercise}

\setcounter{ex}{8}

\begin{exercise}
    $  $

    Supongamos que $ (b,X) $ es principal y tenemos $ f \in A[X]:\ (b,X) = (f) \implies $ 
    $$ \implies \left\{
    \begin{array}{l}
        X = f(X)g(X),\ con\ g,h \in A[X]\\
        b = f(X)h(X)
    \end{array}
    \right. 
    \implies_{X = 0} \left\{
    \begin{array}{l}
        0 = f(0)g(0)\\ b = f(0)h(0)
    \end{array}
    \right.\ \text{(igualdades en }A) $$

    Notemos que $ b $ cancelable $  \implies f(0) $ cancelable:

    Si fuese $ f(0)  $ no cancelable (= divisor de 0) $ \implies $
    $$ \exists c \in A \setminus \{0\}:\ f(0)c = 0 \implies \left\{
    \begin{array}{l}
        bc = 0\\ c\ne 0 
    \end{array}
    \right\} \implies b\ \text{no es cancelable (contradicción)} $$

    $$ f(0) \implies g(0) =0 \implies g(X) = X\cdot g'(X) \implies X = f(X)g(X) = X f(X)g'(X)\implies$$
    $$\implies_{X\ cancelable\ en\ A[X]} 1 = f(X)g'(X) \implies  f \in \mathcal{U}(A[X]) \implies (b,X) = (f) = A[X] $$

    Entonces $ 1 = br(X)+Xs(X) $ para ciertos $ r,s \in A[X] \implies_{X=0} 1 = br (0)\implies b \in \mathcal{U}(A)$, lo cual es una contradicción ya que sabemos que $ b $ no es invertible.

    Falta ver que $ (X,Y) $ no es principal en $ A[X,Y] $
    $$ A[X,Y] \cong (A[Y])[X] $$

    $ Y $ no es cancelable y no unidad en $ A[X] $, basta aplicar ahora el ejercicio.
\end{exercise}


\begin{exercise}
    $  $

    \begin{flushright}
        \textbf{Apartado a)}
    \end{flushright}
    

    $$ IJ_1 = IJ_2 \cancel{\implies} J_1 = J_2 $$

    Tomaremos $ J_2 = 0 $ y $ I = J_1 = (\overline{2})  $ en $ \mathbb{Z}_{4} $

    \begin{flushright}
        \textbf{Apartado b) Enunciado modificado}
    \end{flushright}
    
    Todo ideal principal en un dominio cancela (para el producto de ideales)

    $ I = (y) $ y tenemos que $ IJ_1 = IJ_2 \implies? J_1 = J_2 $

    Basta con probar que $ J_1 \subset  J_2 $.

    Sea $ z \in J_1 \implies yz \in IJ_1 = IJ_2 \implies yz = \sum\limits_{i=1}^{t}y_iz_i $
    $$ y_i \in I = (y) \implies y_i = a_iy \text{ , para algún }a_i \in A \implies yz = \sum\limits_{i=1}^{t}(a_iy)z_i = y \sum\limits_{i=1}^{t}a_iz_i  $$ 
    $$  \implies z = \sum\limits_{i=1}^{t}a_iz_i \implies z \in J_2$$


\end{exercise}

\begin{exercise}
    Este ejercicio no está resuelto pero es muy importante. 
\end{exercise}

\begin{exercise}
    \textbf{bis}
    
    \textbf{Sea $ A = A_1\times ... \times A_m $, donde los $ A_i $ son anillos locales (Ej1.21). Probar que los idempotentes de $ A $ son las $ m- $uplas $ (e_1,...,e_m)  $ tales que $ e_i \in \{0,1\}\ \forall i = 1,...,m$. Como aplicación, describir un método para calcular todos los elementos idempotentes de $ \mathbb{Z}_{n} $, para $ n>1 $. Particularizarlo a $ \mathbb{Z}_{4200} $.}

    Utilizando el ejercicio 5.e), tenemos que $ e = (e_1,...,e_m) $ es idempotente en $ A \iff e_i $ es idempotente en $ A\ \forall i = 1,...,m $

    La primera parte se reduce a probar que si $ B $ es un anillo local, entonces sus únicos idempotentes son 0,1.

    \begin{demonstration}
        Supongamos que $ e = e^2 \in B,\ e \not \in \{0,1\} \implies e,1-e$ son idempotentes (1.12(b)) y $ e,1-e \not \in \mathcal{U}(B) $ (1.12(c)). Por tanto $ (e),(1-e) $ son ideales propios de $ B \implies (e),(1-e) \subset  m:=$ único ideal maximal de $ B $. Entonces, $ (e)+(1-e)\subset m \implies 1=e+(1-e) \in m $

        Con lo que tenemos una contradicción porque $m \subsetneq B$

    \end{demonstration}

    \textbf{Describimos el método:} $ n = p_1^{\mu_1}\cdots p_{t}^{\mu_{t}}\implies\footnote{Teorema chino de los restos} \mathbb{Z}_{n} \cong \mathbb{Z}_{p_1^{\mu_1}}\times \mathbb{Z}_{p_t^{\mu_t}}$ que lleva $ \overline{a} \hookrightarrow (\overline{a},...,\overline{a}) $

    Entonces en $ \mathbb{Z}_{p^{t}} $, el único ideal maximal es $ (\overline{p}) $ ($ p $ primo).

    Vemos el caso de $ n = 4200 = 2^3\cdot 3\cdot 5^2\cdot 7$, tomamos el isomorfismo de anillos:
    $$ \mathcal{U}:\mathbb{Z}_{4200} \to \mathbb{Z}_{8}\times\mathbb{Z}_{3}\times\mathbb{Z}_{25}\times\mathbb{Z}_{7} $$
    $$ \overline{a} \hookrightarrow (a+8\mathbb{Z},a+32\mathbb{Z},a+25\mathbb{Z},a+7\mathbb{Z}) $$

    Hay $ 2^{4} = 16 $ idempotentes:
    Calculamos el idempotente $ \overline{e} \in Z_{4200}    $ tal que $ \phi(\overline{e}) = (\overline{1},\overline{0},\overline{1},\overline{0}) $

    Luego se nos queda el sistema de congruencias:
    $$ \left\{
    \begin{array}{l}
        e \equiv 1\ (mod\ 8)\\
        e \equiv 0\ (mod\ 3)\\
        e \equiv 1\ (mod\ 25)\\
        e \equiv 0\ (mod\ 7)\\
    \end{array}
    \right. $$

    Las ecuaciones que son congruentes con 1 se pueden agrupar en $ x \equiv\ (mod\ 200 = 8\cdot 25) $, de forma análoga nos queda, $ x\equiv\ (mod\ 21) $.
    $$ \left\{
    \begin{array}{l}
        x = 1+200t\\ 
        x = 21s
    \end{array}
    \right.
    \implies 1+200t = 21s \implies 1 = 21s+200(-t) $$

    Y utilizando la identidad de Bézout y el algoritmo de Euclides obtendremos una solución, en este caso es ($ s = -19,\ t = -2 $)

\end{exercise}

\setcounter{ex}{11}
\begin{exercise}$  $
    \begin{flushright}
        \textbf{Apartado a)}
    \end{flushright}
    
    $$ a \in (e) \iff a = ea $$
    $ \implies $

    Clara.\\
    $ \implies $
    
    Sea $ a \in (e) \implies a = ex $, con $ x \in A \implies ea = e^2x = ex = a $

    $$ \left\{
    \begin{array}{l}
        a = ea\\ b = eb
    \end{array}
    \right\} ab = e^2 ab = eab $$
    
    Si $ e  $ es una unidad, entonces $ e = 1 $
    $$ e = ^2 \implies 1 = e $$

    \begin{flushright}
        \textbf{Apartado c)}
    \end{flushright}
    % $$ (1-e)^2 = 1+e^2-2e = 1+e-2e = 1-3 $$
    Sea $ a \in (e) \cap (f) \implies \left\{
    \begin{array}{l}
        a = ea\implies fa = fea = 0 \\ a = fa 
    \end{array}
    \right\} \implies a = 0 $

    Y tenemos que $ (e)+(f) = A $ ya que $ e+f = 1 $.

    Como anillos (no ideales), tenemos el isomorfismo de anillos $ A \to (e) \times (f) $ dado por $ a\hookrightarrow (ae,af) $

    \begin{flushright}
        \textbf{Apartado d)}
    \end{flushright}
    
    $$ 1 = e+f,\ e \in I,\ f \in J \implies e+f = 1 = 1^2 = (e+f)^2 = e^2+\underbrace{2ef}_{I \cap J = \{0\}}+f^2 = e^2+f^2 $$

    Hemos descompuesto el 1 como suma de elementos de $ I,J $ de dos formas distintas. Como la suma es directa, tenemos entonces que $ e = e^2,\ f= f^2 $. Luego $ e $ es idempotente, $ f = 1-e $ y tenemos $ (e) \subset\ I,\ (1-e) \subset J $. Falta ver que se da la igualdad:
    $$ Sea\ x \in I \implies x =x\cdot 1 = x(e+f)=xe+\underbrace{\cancel{xf}}_{\in I \cap J = \{0\}} $$


\end{exercise}

\setcounter{ex}{13}

\begin{exercise}
    Supongamos que $ f $ es suprayectivo y vamos a probar que si $ P \unlhd B$ y $ f ^{-1}(P) $ es primo (en $ A $). Entonces $ P $ es primo en $ B $.

    Usando el primer teorema de isomorfía, $ \dfrac{A}{\operatorname{Ker}(f)}\cong B $:

    $$ \operatorname{Spec}(B)\to \operatorname{Spec}\left(\dfrac{A}{\operatorname{Ker}(f)}\right) $$
    $$ Q \hookrightarrow \overline{f}^{-1}(q) = \{a \in \dfrac{A}{\operatorname{Ker}(f)}:\ f(a) \in Q\} = \dfrac{f ^{-1}(q)}{\operatorname{Ker}(f)} $$

    Y utilizando 1.38.2 de los apuntes de Alberto tenemos la biyección:
    $$ \{\substack{\text{ideales primos que}\\\text{contienen a }A}\} \to \operatorname{Spec}(\dfrac{A}{\operatorname{Ker}(f)}) $$

    Esta biyección lleva $ f ^{-1}(P) \hookrightarrow \dfrac{f ^{-1}(P)}{\operatorname{Ker}(f)} \in \operatorname{Spec}(\dfrac{A}{\operatorname{Ker}(f)}) $
\end{exercise}

\setcounter{ex}{16}

\begin{exercise}
    \textbf{Consideración previa general.}

    Sea $ I \properideal A $ y queremos identificar $ (\overline{a}_{1,...,\overline{a}_{m}}) = $ ideal de $ A/I $ generado por $ \{\overline{a}_{1},...,\overline{a}_{m}\} $.
    $$ (\overline{a}_{1},...,\overline{a}_{m}) = \dfrac{J}{I},\ \text{para cierto }J \ideal A:\ I \subset J $$

    Sea ahora $ J = (a_1,...,a_m) +I $, tendremos que comprobar si:
    $$ (\overline{a}_{1},...,\overline{a}_{m}) =? \dfrac{(a_1,...,a_m)+I}{I} $$

    La inclusión $ \subset  $ es directa porque cada uno de los $ \overline{a}_{i} $ se incluye en $\dfrac{(a_1,...,a_m)+I}{I}$.

    Para la inclusión $ \supset $, tenemos:
    $$ \overline{z} \in \dfrac{(a_1,...,a_m)+I}{I} \implies z +I = b+y+I,\ con\ b \in (a_1,...,a_{r}),\ y \in I \implies_{y \in I} z+I = b+I$$

    Por tanto, todos los elementos de $ \dfrac{(a_1,...,a_m)+I}{I} $ son de la forma $ b+I =\overline{b}$, donde $ b \in (a_1,...,a_m) $, pero $ b = r_1a_1+...+r_ma_m $ con $ r_i \in A\ \forall i =1,...,m $. Tomando ahora clases tenemos:
    $$ \overline{b} = \overline{r}_1\overline{a}_1+...+\overline{r}_m\overline{a}_m \implies \overline{b} \in (\overline{a}_{1},...,\overline{a}_{m}) $$

\hrulefill

    \begin{flushright}
        \textbf{Observación:}
    \end{flushright}

    Llamamos $ B = \dfrac{K[X,Y,Z]}{(XY,XZ)},\ A = K[X,Y,Z],\ I = (XY,XZ) $.

    \begin{flushright}
        \textbf{Apartado a)}
    \end{flushright}
    
    $$ (\overline{X},\overline{Y})\ = \dfrac{(X,Y)+(XY,XZ)}{(XY,XZ)} \left( \substack{Obs:\ (XY,XZ)\\ \subseteq (X,Y)} \right) = \dfrac{(X,Y)}{(XY,XZ)}$$

    \begin{flushright}
        \textbf{Apartado b)}
    \end{flushright}
    
    Razonamiento parecido al apartado anterior:
    $$ (\overline{X},\overline{Z}) = \dfrac{(X,Z)}{(XY,XZ)} $$

    \begin{flushright}
        \textbf{Apartado c)}
    \end{flushright}
    
    Razonamiento como en a).
    $$ (\overline{Y},\overline{Z}) = \dfrac{(Y,Z)}{(XY,XZ)} $$
    
    \begin{flushright}
        \textbf{Apartado d)}
    \end{flushright}

    Razonamiento como en a).
    $$ (\overline{X}) = \dfrac{(X)}{(XY,XZ)} $$

    \begin{flushright}
        \textbf{Apartado e)}
    \end{flushright}
    
    $$ (\overline{Y}) = \dfrac{(Y)+(XY,XZ)}{(XY,XZ)} = \left( \substack{Obs:\ (XY)\\ \subseteq (Y)} \right) = \dfrac{(Y,XZ)}{XY,XZ} $$
    
    \begin{flushright}
        \textbf{Apartado f)}
    \end{flushright}
    $$ (\overline{Z}) = \dfrac{(Z)+(XY,XZ)}{(XY,XZ)} = \dfrac{(Z,XY)}{(XY,XZ)} $$

    \hrulefill

    Usaremos ahora que $ P  $ es primo $ \iff\ B/P$ es dominio, tomaremos $ P = J/I $ y $ B = A/I $, lo que nos queda:
    
    $ J/I  $ es primo en $ A/I \iff \dfrac{A/I}{J/I}$ es dominio $ \iff\footnote{Segundo teorema de isomorfía} A/J $ dominio. 

    Volvamos ahora a cada caso particular:

    \begin{flushright}
        \textbf{Apartado a)}
    \end{flushright}
    
    $$ \dfrac{(X,Y)}{(XY,XZ)}\ \text{primo }\iff \dfrac{K[X,Y.Z]}{(X,Y)}\cong K[Z]\ dominio $$

    \begin{flushright}
        \textbf{Apartado b,c)}
    \end{flushright}

    Análogos al a).

    \begin{flushright}
        \textbf{Apartado d)}
    \end{flushright}
    $$ \dfrac{K[X,Y,Z]}{(X) \cong_{ejercicio} K[Y,Z]} $$

    \begin{flushright}
        \textbf{Apartado e)}
    \end{flushright}
    
    $$ \dfrac{K[X,Y,Z]}{(Y,XZ)}\ \text{no es dominio porque }\overline{X}\overline{Z} = \overline{0}\ y\  \overline{X}\ne \overline{0} \ne \overline{Z}  $$

    \begin{flushright}
        \textbf{Apartado f)}
    \end{flushright}
    
    Análogo al anterior

\end{exercise}

\textbf{Ejercicio utilizado en el anterior ejercicio:} Sea $ B $ anillo y $ X_1,...,X_n $ variables sobre $ B $. Para cada subconjunto $ J \subset\ \mathbb{N}_{n} = \{1,...,n\} $ consideremos la composición de homomorfismos de anillos:
$$ B[X_i:\ i \in \mathbb{N}_{n}] \hookrightarrow^i B[X_1,...,X_n] \xrightarrow{\pi} \dfrac{B[X_1,...,X_m]}{(X_j: j \in J)} $$

Probar que $ \pi \circ i  $ es un isomorfismo de anillos.


\begin{exercise}
    


    $$ I+(x) = \{a+xf:\ a \in I,\ f \in A[X]\} = \{f \in A[X]:\ g(0) \in I\} $$

    % \begin{flushright}
    %     \textbf{Apartado a)}
    % \end{flushright}
    
    Se reduce a probar que cada uno es primo si y solo si $ A/I $ es dominio si y solo si $ A[X]/I[X] $ es dominio si y solo si $ A[X]/I+(x) $ es dominio.

    Tenemos un homomorfismo de anillos:
    $$ \phi:\ \dfrac{A}{I} \to \dfrac{A[X]}{I+(x)}\ tal\ que\ \overline{a}=a+I \hookrightarrow [a] $$

    Claramente está bien definido y es homomorfismo de anillos (conserva suma y multiplicación).
    
    Tenemos:
    $$ \dfrac{A[X]}{I+(x)}\ni [f(X)] = [f(0)+Xg(X) ] = [f(0)]+\cancel{[Xg(X)]} = \phi(\overline{f(0)}) $$
    
    Luego $ \phi $ es suprayectiva. Comprobamos la inyectividad.

    $$ \operatorname{Ker}(\phi) = \{\overline{a} = a+I:\ [a] = [0]\} = \{\overline{a} \in A/I:\ a \in I+(x)\} = \{\overline{a} \in A/I:\ ain I\} = \{\overline{0}\} $$

    Por tanto $ \phi  $ es un isomorfismo de anillos. Con esto tenemos el apartado b) y la mitad del apartado a). Veamos ahora la relación entre $ A/I $ y $ A[X]/I[X] $. Consideremos el homomorfismo:
    $$ \dfrac{A}{I} \to \dfrac{A[X]}{I[X]}\ dado\ por\ \overline{a}\hookrightarrow[a] $$

    A partir de este formamos:
    $$ \psi: \dfrac{A}{I}[X] \to \dfrac{A[X]}{I[X]}\ dado\ por\ \psi:\sum\limits_{i=1}^{n}\overline{a}_{i}X^{i} \hookrightarrow \sum\limits_{i=1}^{n}[a_i][X]^{i} = \left[ \sum\limits_{i=0}^{n}a_iX^{i}\right] $$

    Es directo ver que $ \psi $ es suprayectiva, y tenemos que:
    $$ \operatorname{Ker}(\psi) = \{\sum\limits_{}^{} \overline{a}_{i}X^{i}: \sum\limits_{i=0}^{n}a_iX^{i} \in I[X]\} \implies \operatorname{Ker}(\psi) = \{\sum\limits_{i=0}^{n}\overline{a}_{i}X^{i}:\ a_i \in I\ \forall i =0,1,...,n\} =  $$
    $$ = \{\sum\limits_{i=0}^{n} \overline{a}_{i} X^{i}: \overline{a}_{i} = \overline{0}:\ \forall i = 0,1,2,...,n\} = \{\overline{0}\} $$

    Como conclusión llegamos a que $ A[X]/I[X]\cong \dfrac{A}{I}[X] $ lo que nos lleva a demostrar c). Porque $ \dfrac{A}{I}[X] $ nunca será un cuerpo.





    
\end{exercise}


\begin{exercise}
    La última parte se queda como ejercicio planteado.

    $$ 0 = (-a)^{n} ? (1-b)^{n} = \sum\limits_{i=0}^{n}b^{i}1^{n-i} = 1-nb + \binom{n}{2} b^2 +...+\binom{n}{n-1} (-b)^{n-1}+(-b)^{n} \implies $$
    $$\implies 1 = b(n-\binom{n}{2}b+...-\binom{n}{n-1}(-b)^{n-2}+(-b)^{n-1}) $$
    
\end{exercise}

\begin{exercise}
    \textbf{Notación modificada}.

    Denotamos al radical de Jacobson de $ A $ como $ J(A) $.

    \begin{flushright}
        \textbf{Apartado a)}
    \end{flushright}
    
    Demostramos que es un si y solo si.\\
    $ \impliedby $

    Supongamos que $ a \not \in J(A)$:
    $$ \implies \exists M \in \operatornamewithlimits{MaxSpec}(A):\ a \not \in M \implies M \subsetneq M + (a)\implies M +(a) = A \implies$$
    $$ 1 =m+ra,\ m \in M,\ r \in A \implies m = 1+(-r)a \in 1+(a) \implies m \in \mathcal{U}(A) \implies A = (m) \subseteq\ M $$ 

    Con lo que tenemos una contradicción ya que $ M $ es propio al ser maximal.\\
    $ \implies $

    Supongamos que $ 1+(a) \not \subseteq  U(A) $, entonces:
    $$ \implies \exists r \in A:\ q+ra \not \in \mathcal{U}(A) \implies (1+ra) \subseteq\ M\ \text{para algún $ M $ maximal} $$
    $$ \implies 1 = \underbrace{1+ra}_{\in M}+\underbrace{(-r)a}_{\in J(A) \subseteq\ M} \implies 1 \in M  $$

    Y llegamos de nuevo a la misma contradicción, $ M $ es propio, luego no puede contener al 1 (sería el total).

    \begin{flushright}
        \textbf{Apartado b)}
    \end{flushright}
    
    Sea $ e  $ idempotente, $ e = e^2 \in J(A) $. Por el apartado a). Tenemos que $ 1-e \in \mathcal{U}(A) $ y sabemos por el problema 12 que $ 1-e $ es idempotente. Además, por este ejercicio también sabemos que al ser unidad e idempotente, $ 1-e = 1 \implies e = 0 $.

\end{exercise}

\begin{exercise}
    \begin{flushright}
        \textbf{Apartado a)}
    \end{flushright}
    $  $\\
    $ \implies $

    Se trata de probar que $ M = A \setminus \mathcal{U}(A) $. La inclusión $ \subseteq  $ es directa. Basta probar $ \supseteq $: Si $ a \in A \setminus \mathcal{U}(A) \implies (a) $ es un ideal propio $ \implies (a) \subseteq M \implies a \in M $\\
    $ \impliedby $

    Si $ I \properideal A $ es ideal propio, $ I \subseteq\ A \setminus \mathcal{U}(A) $

    \begin{flushright}
        \textbf{Apartado b)}\\
        (ver ejercicio anterior)
    \end{flushright}
    
    Como $ J(A) = M \implies 1+M \subseteq \mathcal{U}(A)$. Lo único que tenemos que ver es que sea subgrupo. Que sea cerrado para la multiplicación es trivial. Veamos que existen los inversos.

    Sea $ m \in M \implies 1+m \in 1+M \subseteq \mathcal{U}(A)\implies $ escribimos $ (1+m)^{-1} = 1+m' $, con $ m' \in A $. Veamos que $ m' \in A $:
    $$ 1 = (1+m)(1+m') $$


    % \begin{flushright}
    %     \textbf{Apartado c)}
    % \end{flushright}
    
    \begin{flushright}
        \textbf{Apartado d)}
    \end{flushright}
    $ \overline{1}+m = \overline{1}+(\overline{3})  $ es un subgrupo multiplicativo de $ \mathcal{U}(\mathbb{Z}_{27}) $
    $$ \overline{1}+(\overline{3}) = \{\overline{1+a}:\ a \in 3\mathbb{Z}\} = \{\overline{b}:\ b \equiv 1\ (mod\ 3)\} = \{\overline{1},\overline{4},\overline{7},\overline{10},\overline{13},\overline{16},\overline{19},\overline{22},\overline{25}\} $$

    Vemos que lo genera $ \overline{4} $:
    $$ <\overline{4}> = \{1,\overline{4},\overline{16},\overline{10},...\}\ (\text{tamaño mayor que 4}) \implies \overline{1}+(\overline{3}) = <\overline{4}> $$

\end{exercise}

\setcounter{ex}{21}

\begin{exercise}
    $$ \mathbb{Z}_{(p)} = \{q \in \mathbb{Q}:\ q = \dfrac{a}{b},\ donde\ p \not| b\} $$

    \begin{flushright}
        \textbf{Apartado a)}
    \end{flushright}
    
    $$ \mathcal{U}(\mathbb{Z}_{(p)}) = \{q \in \mathbb{Z}_{(p)}:\ q = \dfrac{a}{b},\ con\ a \not \in p\mathbb{Z}\} $$

    \begin{flushright}
        \textbf{Apartado b)}
    \end{flushright}
    $ \mathbb{Z}_{(p)} $ anillo local con $ \mathbb{Z}_{(p)} \setminus \mathcal{U}(\mathbb{Z}_{(p)})\ = m$ el único idea maximal que está generado por $ \dfrac{p}{1}=p $ 
    $$ \mathbb{Z}_{(p)} \setminus \mathcal{U}(\mathbb{Z}_{(p)}) = p\mathbb{Z}_{(p)} = \{p \dfrac{a}{b} :\ \dfrac{a}{b} \in \mathbb{Z}_{(p)}\} $$

    Que se ve (en parte) con el ejercicio 1.21.a

    \begin{flushright}
        \textbf{Apartado c)}
    \end{flushright}
    $$ \dfrac{\mathbb{Z}_{(p)}}{p\mathbb{Z}_{(p)}} \cong? \mathbb{Z}_{p} := \dfrac{\mathbb{Z}}{p\mathbb{Z}} $$

    Definimos el homomorfismo:
    $$ \mathbb{Z}_{p} \to \dfrac{\mathbb{Z}_{(p)}}{p\mathbb{Z}_{(p)}}\hspace{5mm}\overline{a} = [a] = a+p\mathbb{Z}_{(p)} $$

    $$ \operatorname{Ker}(\phi) = \{\overline{a} = a+p\mathbb{Z}:\ a+p\mathbb{Z}_{(p)} = p\mathbb{Z}_{(p)}\} $$
    
    Luego los elementos serán de la forma:
    $$ \overline{a},\ con\ a = p \dfrac{r}{s},\ con\ p \not | s \implies \left\{
    \begin{array}{l}
        sa = pr \\
        p \not | s
    \end{array}
    \right.
    \implies p | a \implies \overline{a} = \overline{0} \implies \operatorname{Ker}(\phi) = \{\overline{0}\} $$

    Luego $ \phi $ es inyectivo. Sin embargo, esto lo podríamos haber demostrado diciendo simplemente que los homomorfismos que salen de un cuerpo son inyectivos.

    Comprobemos ahora que $ \phi $ es sobre. Sea $ [a/b] = a/b +p\mathbb{Z}_{(p)} \in \dfrac{\mathbb{Z}_{(p)}}{p\mathbb{Z}_{(p)}} $. Queremos ver que $ [a/b] = [r/1] = \phi(\overline{r}) $, para cierto $ r \in \mathbb{Z} $

    Si $ [a/b] = [0]$, no hay nada que probar. Podemos suponer que $ [a/b] \ne [0] \iff a/b \not \in p\mathbb{Z}_{(p)} = m = \mathbb{Z}_{(p)} \setminus \mathcal{U}(\mathbb{Z}_{(p)})$
    $$ \implies ab \in \mathcal{U}(\mathbb{Z}_{(p)})\ :\ p\not | a\ (y\ p\not | b) $$

    Entonces hacemos:
    $$ \dfrac{a}{b} = a\cdot b ^{-1} \implies [\dfrac{a}{b}] = [a][b ^{-1}] = [a]\cdot [b] ^{-1} = \phi(a)\phi(b) ^{-1} = \phi(a)\phi(b ^{-1}) = \phi (\overline{a} \overline{b} ^{-1}0)$$

    \begin{flushright}
        \textbf{Apartado d)}
    \end{flushright}
    
    Hecho en un ejercicio planteado anteriormente de forma más general.

\end{exercise}

\begin{exercise}
    \textbf{Modificado}
    
    Sea $ I \properideal A $ ideal propio tal que $ I \subseteq J(A) $. Demostrar:
    \begin{enumerate}
        \item Para $ a \in A $, se verifica:
        $$ a \in \mathcal{U}(A) \iff a +I \in \mathcal{U}(A/I) $$
        \item Si $A/I$ no tiene elementos idempotentes no triviales $ \implies $ lo mismo pasa con $ A $.
        \item Si $ I $ es maximal, entonces $ A $ es local.
    \end{enumerate}
    
    \begin{flushright}
        \textbf{Apartado a)}
    \end{flushright}
    $ \implies $

    Trivial ($ ab = 1 \implies \overline{a}\overline{b} = \overline{1} $)\\
    $ \impliedby $

    $$Si\   \overline{a} \in \mathcal{U}(\overline{A}) \implies \exists \overline{b} \in \overline{A}: \overline{a}\overline{b} = \overline{1} \implies ab-1 \in I \subseteq\ J(A) \implies $$ 
    $$ \implies 1+(ab-1) \in \mathcal{U}(A) \implies ab \in \mathcal{U}(A) \implies a \in \mathcal{U}(A) $$

    \begin{flushright}
        \textbf{Apartado b)}
    \end{flushright}
    $$ Sea\ e = e^2 \in A \implies \overline{e} = \overline{e}^2 \implies$$
    $$ \implies \left\{
    \begin{array}{l}
        \overline{e}=\overline{0} \iff e \in I \subseteq J(A) \implies e =0\\
        \acute o\\
        \overline{e} = \overline{1} \implies e-1 \in I \subseteq J(A) \implies 1+(e-1) \in \mathcal{U}(A) \iff e \in \mathcal{U}(A) \implies e=1
    \end{array}
    \right. $$

    
\end{exercise}

\begin{exercise}

    \begin{flushright}
        \textbf{Apartado a)}
    \end{flushright}
    


    Tomamos $ t = t(n,m) = n+m $ y hacemos inducción en $ t\geq 2 $. El caso de $ t=2 $ es claro. Sea $ t >2  $ y supongamos que es cierto siempre que la suma de los exponentes sea < $ t $. 
    
    La hipótesis de inducción nos dice entonces que $ I^{n},\ J^{m-1} $ comaximales y $ I^{n},\ J $ comaximales. Ambas propiedades implican entonces que $ I^{n} $ es comaximal con $ J^{m-1}J = J $ 

    \begin{flushright}
        \textbf{Apartado b)}
    \end{flushright}
    $ \impliedby $

    Tomemos $ x = 1,\ y=0 \implies (1+I) \cap J \ne \emptyset \implies \exists a \in I, b \in J:\ 1+a=b \implies_1 = (-a)_{\in I}+b_{\in J} $\\
    $ \implies $

    Sean $ x,y \in A \implies x-y \in A = I+J \implies x-y i+j,\ con\ i \in I,\ j \in J \implies$
    $$ x-i = y+j \in (x+I) \cap (y+J) $$

\end{exercise}


\chapter{Anillos noetherianos}

\noindent Se han cambiado algunas definiciones respecto a los apuntes de Alberto del Valle.



\begin{definition}
    \textbf{Retículo}

    Un conjunto (parcialmente) ordenado $ (\mathcal{L},\leq )  $ se dice que es un \textbf{retículo} cuando cualquier subconjunto de dos elementos tiene ínfimo y supremo. $ (\mathcal{L},\leq ) $ se dice \textbf{retículo completo} cuando cualquier subconjunto no vacío tiene ínfimo y supremo.
\end{definition}

\begin{flushright}
    \textbf{Notación}
\end{flushright}

    Si $ 0 \ne S \subseteq\ \mathcal{L} \implies $
    $$ \left\{
    \begin{array}{l}
        \bigvee_{s \in S} = \sup_{\mathcal{L}}(S)\\ 
        \bigwedge_{s \in S} = \inf_{\mathcal{L}}(S)\\ 
    \end{array}
    \right. $$

\begin{definition}
    \textbf{Compacidad y cocompacidad}

    Sea $ (\mathcal{L},\leq ) $ un retículo completo. Diremos que $x \in \mathcal{L} $ es \textbf{compacto} (resp. \textbf{cocompacto}) si dado cualquier subconjunto $ \emptyset \ne S \subseteq\ \mathcal{L} $ tal que $ \bigvee_{s \in S} s = x $ (resp. $ \bigwedge_{s \in S}s = x $), existe $ F \subseteq S $ finito tal que $ x = \bigvee_{s \in F} s $ (resp. $ x = \bigwedge_{s \in F}s $).
\end{definition}

\setcounter{ex}{0}

\begin{exercise}
    \textbf{Ejercicio propuesto}

    Sea $ A $ un anillo. Probar:
    \begin{enumerate}
        \item $ (\mathcal{L}(A),\subseteq ) $ es un retículo completo.
        \item Un ideal $ I \unlhd A $ es un elemento compacto de $ \mathcal{L}(A) $ sii es un ideal finitamente generado.
    \end{enumerate}
\end{exercise}

\begin{proposition}
    Sea $ (\mathcal{L},\leq ) $ un conjunto ordenado. Las siguientes afirmaciones son equivalentes.
    \begin{enumerate}
        \item $ (\mathcal{L},\leq ) $ satisface la condición de cadena ascendente (ACC en inglés): Si $ s_1\leq s_2 \leq\ ... \leq\ s_n\leq ... \implies m \in \mathbb{Z}^{+}:\ s_m=s_{m+1}=...$ 
        \item Todo subconjunto $ \emptyset \ne S \subseteq\ \mathcal{L} $ tiene algún elemento maximal.
    \end{enumerate}
    
    Si además $ (\mathcal{L},\leq ) $ es un retículo completo, dichas condiciones son equivalentes a:

    3. Todo elemento $ x \in \mathcal{L} $ es compacto.
\end{proposition}

\begin{flushright}
    \textbf{Observación}
\end{flushright}

    $ (\mathcal{L},\leq ) $ es conj. ordenado (retículo completo) $ \iff $ ($ \mathcal{L},\geq  $) es conjunto ordenado (retículo completo).

Luego podemos hacer una proposición equivalente a la anterior cambiando la condición de cadena ascendente por descendente y $ \leq  $ por $ \geq  $.

\begin{demonstration}
    $ 1\implies 2 $

    Por reducción al absurdo, supongamos que existe un subconjunto $ \emptyset \ne S \subseteq\ \mathcal{L} $ tal que $ S $ no tiene elementos maximales. 

    Sea $ s_1 \in S $ arbitrario. Tenemos que $ s_1 $ no es maximal, luego $ \exists s_2 \in S $ tal que $ s_1 < s_2 $ con $ s_2 $ no maximal, luego podemos tomar $ s_3 $. Así construimos una cadena estrictamente ascendente $ s_1<s_2<...\ $, lo que es una contradicción con ACC.\\
    $ 2\implies 1 $

    Sea $ s_1\leq se\leq s_3\leq ...\leq s_n\leq  $ una cadena ascendente $ \implies S: = \{s_n : n \in \mathbb{Z}^{+}\} $ tiene un elemento maximal, pongamos $ \mu = s_m $ para algún $ m \in \mathbb{Z}^{+} \implies \mu = s_m \leq  s_{m+k} \forall k = 0,1,..., \implies S_m = S_{m+k}\ \forall k \geq  0$

    En adelante supondremos que $ (\mathcal{L},\leq ) $ es un retículo completo.\\
    $ 3\implies 1 $

    Sea $ s_1\leq s_2\leq ... $ una cadena ascendente en $ \mathcal{L} $ y tomamos $ x = \bigvee_{n \in \mathbb{Z}^{+}}s_n \implies x = \bigvee_{k=1}^{r} s_{n_{k}} $ para cierto subconjunto finito $ \{n_1<...<n_{r}\} \subseteq \mathbb{Z}^{+} \implies x=s_{n_{r}}$. Como $ s_{n_{r}} $ es el supremo, $ s_{n_{r}+k} = s_{n_{r}}\ \forall k >0 $\\
    $ 2\implies 3 $

    Sea $x \in \mathcal{L}$ arbitrario y $ \emptyset \ne S \subseteq \mathcal{L}  $ tal que $ x = \bigvee _{s \in S}s $. Tomamos ahora $ x_{F} = \bigvee_{s \in F}s\ \forall F \subseteq S $ finito, ¿$ x = \bigvee _{\substack{F \subseteq S\\F\ finito}} x_{F}$?
    
    Pero sabemos que $ \Sigma = \{x_{F}:  F \subseteq S\ finito\} $, luego $ \Sigma $ tiene un elemento maximal: $ \exists F' \subseteq  S $ finito tal que $ x_{F'} = \bigvee_{s \in F'} s  $ es maximal en $ \Sigma $.
    
    Se trata de probar que $ x = x_{F'} $, sea $ s \in S $ arbitrario $ \implies$
    $$F'' = F' \cup \{s\} \implies x_{F'} = \bigvee_{s \in F'} s \leq  x_{F''} = \bigvee_{s \in F''} s \implies_{x_{F'}\ maximal}  $$
    $$ \implies x_{F'} = x_{F''} \implies t \leq  x_{F'}\ \forall t \in S \implies x_{F'}\leq x = \bigvee_{s \in S}s \leq  x_{F'} $$
    


\end{demonstration}

\begin{definition}
    \textbf{Anillo noetheriano}

    Un anillo $ A $ se dice que es \textbf{noetheriano} cuando $ (\mathcal{L}(A , \subseteq )) $ cumple las tres condiciones equivalentes de la proposición anterior\footnote{Recordemos que $ (\mathcal{L}(A),\subseteq ) $ es un retículo completo por el ejercicio propuesto.} .
    % Cuando un anillo $ A $ cumple las condiciones (las tres) de la proposición anterior se dice que es \textbf{noetheriano}.
\end{definition}

\setcounter{propositiont}{4}
%2.5
\begin{proposition}
    Si $ A $ es noetheriano, de cualquier subconjunto $ X \subseteq A $ se puede extraer un subconjunto finito (minimal) $ X_0 $ tal que $ (X) = (X_0) $
\end{proposition}

\begin{demonstration}
    $$ \Omega = \{I = (X'):\ X' \subseteq  X,\  X'\ finito\} $$
    $$ \mathcal{L}(A)\ noetheriano \implies \exists I_0 = (X_0)\text{ elemento maximal de } \Omega \implies X \subseteq\footnote{Ejercicio} I_0 = (X_0) $$
\end{demonstration}

%2.6
\begin{proposition}
    Si $ D $ es un dominio noetheriano $ \implies D $ es un dominio de factorización (posiblemente no única)

\end{proposition}

\begin{demonstration}
    Supongamos que no es así, luego $ \exists a \in A \setminus (\mathcal{U}(A) \cup \{0\}) $ que no es producto de irreducibles $ \implies \Omega \ne \emptyset \implies \exists (b) \in \Omega:\ (b) $ es maximal en $ \Omega \implies b $ no es irreducible $ \iff_{\mathcal{L}(A)\ noeth} \exists x,y \in A \setminus \mathcal{U}(A):\ xy=b$. Entonces:

    $$ \left\{
    \begin{array}{l}
        (b) \subsetneq (x)\\
        (b) \subsetneq (y)\\
    \end{array}
    \right\} \implies (x) ,(y) \not \in \Omega \implies  $$

    $ \implies  x$ e $ y $  son producto finito de irreducibles $ \implies b=xy$ también lo es, luego hemos llegado a una contradicción. 
\end{demonstration}

%2.7
\begin{proposition}
    Si $ A $ es noetheriano, entonces todo ideal contiene un producto finito de ideales principales
\end{proposition}


\begin{demonstration}
    Supongamos que no es cierto $ \iff \exists I \unlhd A:\ I $ no contiene ningún producto finito de ideales primos.
    $$ \emptyset \ne \Omega = \{I' \unlhd A:\ I' \text{ no contiene ningún producto finito de ideales primos}\} $$

    Como es $ \Omega \ne \emptyset $, podemos tomar $ I_0 \in \Omega  $ maximal $ \implies I_0 $ no es primo $ \iff \exists\ a,b \in A \setminus I_0:\ ab \in I_0 \implies  $
    $$ \implies \left\{
    \begin{array}{l}
        I_0 \subsetneq I_0+(a)\\
        I_0 \subsetneq I_0+(b)
    \end{array}
    \right\} \implies I_0+(a) ,\ I_0+(b) \not \in \Omega \implies \exists P_1,...,P_{r}, Q_1,...,Q_{s}  $$

    De forma que $ P_i,Q_i $ son ideales primos tales que $ P_1\cdots P_{r}\subseteq I_0+(a)  $ y $ Q_1\cdots Q_{s} I_0+(b) \implies P_1\cdots P_{r}Q_{1}\cdots Q_{s} \subseteq  (I_0+(a))(I_0+(b))\subseteq I_0$
\end{demonstration}

\begin{theorem}
    \textbf{De la base de Hilbert}

    Sea $ A  $ un anillo y $ n > 0 $ un entero. Las siguientes afirmaciones son equivalentes:
    \begin{enumerate}
        \item $ A $ es noetheriano.
        \item $ A[X_1,...,X_n] $ es noetheriano.
    \end{enumerate}
\end{theorem}

\begin{demonstration}
    $ 2\implies 1 $
    \begin{flushright}
        \textbf{Observación}
    \end{flushright}

    $$ I \unlhd A \implies I[X] \unlhd A[X],\ A \cap I[X] = I $$

    Sea $ I_0 \subseteq I_1 \subseteq ...  $ una cadena ascendente de ideales de $ A \implies I_0[X] \subseteq I_1[X] \subseteq ... $ es una cadena en $ A[X] \implies_{A[X]\ noetheriano} \exists m>0 :\ I_{m}[X] = I_{m+k}[X]\ \forall k\geq  0 \implies_{ver\ obs.} A \cap I_{m}[X] = A \cap I_{m+k}[X]\ \forall k \geq  0$ 
    
    \noindent$ 1 \implies 2 $

    Basta probarla cuando $ n = 1,\ A[X_1,...,X_n] \cong A[X_1,...,X_{n-1}][X_n] $

    Vamos a probar que $ A[X] $ es noetheriano. Supongamos que no lo es $ \implies \exists I \unlhd A[X]:\ I $ no es f.g. $ \implies $ elegimos $ f_{1} \in I \setminus \{0\} $ con grado máximo $ (n_1) $ y ponemos $ b_1$ como el coeficiente principal de $ f_{1} $:

    $ (0) \subsetneq (f_{1}) \subsetneq I \implies $ tomo $ f_{2} \in I \setminus (f_{1})$ con grado mínimo $ (n_1 \geq  n_2)  $ y ponemos $ b_2  $ el coeficiente prinicpal de $ f_{2} $. Entonces $ (f_{1},f_{2}) \subsetneq I \implies  $ tomo $ f_{3} \in I \setminus (f_{1},f_{2}) $ con grado mínimo $ n_3 (\geq  n_2 \geq  n_1) $ y tomamos $ b_3$ el coeficiente principal de $ f_{3} $ ...

    Probaremos entonces que la cadena $ (b_1) \subseteq (b_1,b_2) \subseteq (b_1,b_2,b_3) $ es una cadena \textbf{estrictamente} ascendente, lo que nos llevará a una contradicción.

    Si $ (b_1,...,b_{k-1}) = (b_1,...,b_{k})  \implies b_{k} = a_1b_1+...+a_{k-1}b_{k-1}$ para ciertos $ a_i \in A $, entonces:
    $$ g:= f_{k} a_1X^{n_{k}-n_1}f_{1} -...-a_{k-1}X^{n_{k}-n_{k-1}}f_{k} \in I $$
    $$ 0 = b_{k}-a_1b_1-...-a_{k-1}b_{k-1}  \text{ es el coeficiente prinicpal de $ X^{n_{k}} $ en $g  $} $$

    Si fuese $ g \in (f_{1},...,f_{k-1}) \implies f_{k} = g+ \sum\limits_{i=1}^{k-1} a_iX^{n_{k}-n_i}f_{i} $
    
    Por tanto $ g \not \in (f_{1},..,f_{k-1}) $

    $ b_{k} $ es el coeficiente principal de $ f_{k}\ \forall k\geq  1 \implies g \in I \setminus(f_{1},...,f_{k-1})$ y $ \operatorname{def}(g) < \operatorname{deg}(f_{k}) $
\end{demonstration}

\setcounter{propositiont}{11}
\begin{theorem}
    \textbf{Cohen}

    Sea $ A $ un anillos. Son equivalentes:
    
    \begin{enumerate}
        \item $ A $ es noetheriano.
        \item Todo ideal primo es f.g.
    \end{enumerate}
\end{theorem}

\begin{flushright}
    \textbf{Notación}
\end{flushright}

Si $ I \ideal A $ y $ X \subseteq A \implies (I:X) = \{a \in A:\ aX \subseteq I\} $. Además, $ (I:X) $ es un ideal e $ I \subseteq (I:X) $

Además, $ (I:x) = (I:\{x\}) $

\begin{demonstration}
    $ 2\implies 1 $

    Supongamos que $ A $ no es noetheriano $ \implies I \ideal A:\ I $ no es f.g. $  \implies \Omega = \{I' \ideal A:\ I'\ no\ es\ f.g.\} \ne \emptyset $

    ¿Toda cadena $ (I_{\lambda })_{\lambda  \in \Lambda } $ en $ \Omega $ tiene una cota superior en $ \Omega $?
    $$ J:= \bigcup_{\lambda  \in \Lambda }I_{\lambda } $$

    Si $ J = (a_1,...,a_m  ) \implies \exists \mu \in \Lambda :\ a_1,...,a_m \in I_{\mu} \implies (I_{mu} \subseteq ) J \subseteq  I_{\mu} \implies J = I_{\mu} \implies I_{\mu}\ f.g.\ (contradicci\acute on) $

    Por el lema de Zorn, $ \exists P \in \Omega $, elemento maximal. Demostraremos que $ P $ es un ideal primo (lo que nos llevará a una contradicción).

    Supongamos que $ P $ no es primo, sean $ a,b \in A \setminus P $ y $ ab \in P $
    $$ \implies \left\{
    \begin{array}{l}
        P \subsetneq P+(a)\\
        y\\
        b \in (P:(a)) = (P:a) 
    \end{array}
    \right\} \implies P \subsetneq (P:(a)) \implies$$

    $$ \implies P + (a), (P:a) \not \in \Omega \implies P+(a) = (p_1+r_1a,...,p_{s}r_{s}a),\ p_i \in P,\ r_i \in A\ (P:a) = (q_1,...,q_{s})$$ 
     $$\implies ? P = (p_1,...,p_{s},aq_1,...,aq_{s})\ (\implies\ \text{contradicción})$$

     Sea $ p \in P \implies p = b_1(p_1+r_1a)+...+b_{s}(p_{s}+r_{s}a)\implies p = \sum\limits_{i=1}^{s}b_ip_i + a \sum\limits_{i=1}^{s}b_i{r_i} (*) \implies a \sum\limits_{i=1}^{s}b_ir_i \in P \iff \sum\limits_{i=1}^{s} b_ir_i \in (P:a) = (q_1,...,q_{t}) \implies \sum\limits_{i=1}^{s}b_ir_i = \sum\limits_{j=1}^{t}c_jq_j  \implies $ (falta el final, consultar apuntes de Alberto del Valle)
\end{demonstration}

\begin{theorem}
    $  $

    Sea $ A $ anillo y $ n>0 $ un entero. Son equivalentes:
    \begin{enumerate}
        \item $ A $ es noetheriano.
        \item $ A[[X_1,...,X_n]] $ noetheriano.
    \end{enumerate}

\end{theorem}

\begin{demonstration}
    $ 2\implies1 $

    Como en el caso de polinomios (teorema de la base de Hilbert)

    \noindent$ 1\implies 2 $

    Es la reducción al caso $ n = 1 $ como en polinomios (Si $ n>1,\ A[[X_1,...,X_n]]\cong A[[X_1,...,X_{n-1}]][X_n] $).

    Vamos a probar que $ A[[X]] $ es noetheriano.

    Sea $ P \ideal A[[X]] $ un ideal primo $ \implies I_0 = \{a \in A:\ a = f(0),\ para\ alguna\ \subseteq f \in I\} \implies I_0 \ideal A \implies_{A\ noeth.} I_0 = (b_1,...,b_n)$.

    Como $ b_i \in I_0 \implies $ podemos fijar $ f_{i} \in P:\ f_i(0) =b_i\ \forall i = 1,...,n$

    Tenemos entonces dos casos posibles:
    \begin{enumerate}
        \item $ X \in P \implies P = (f_1,...,f_{m},X)$. Justificamos esta igualdad:

        El lado $ \supseteq $ es directo. Sea ahora $ f \in P \implies f = \underbrace{f(0)}_{\in I_0}+Xg(X) $ con $ f(0) = a_1b_1+...+a_nb_n $ con los $ a_i \in A $. Entonces $ f = a_1b_1+...+a_nb_n+Xg(X) $
        \item $ X \not \in P  $. Probaremos entonces que $ P = (f_{1,...,f_{n}}) $.
        \item 
        De nuevo el lado $ \supseteq  $ es directo. Sea $ f \in P \implies f(0) \in I_0 \implies f(0) = a^{0}_1b_1+...+a^{0}_nb_n$. Entonces $g = f - \sum\limits_{i=1}^{n} a_i^{0}f_{i} \implies$ tiene término independiente nulo $ \implies g = Xg_1(X) \in P \implies_{X \not \in P}\  g_1 \in P $

        Si $ g_{1}(0) = \sum\limits_{i=1}^{n}a_i^{1}b_i \implies g_{1} - \sum\limits_{i=1}^{n} a_i^{1}f_i $ es un polinomio en $ P $ con término indep. nulo. $ \implies g_{1}-\sum\limits_{i=1}^{n}a_i^{1}f_{i} = Xg_{2} \implies g_1 = \sum\limits_{i=1}^{n}a^{1}f_{i} + g_2 $. Con $ g_2 $ múltiplo de $ X $, $ g_2 = Xg_3 \implies g_3 \in P $ 

        Con lo que queda una suma infinita:
        $$ \sum\limits_{i=0}^{n}a^{0}_{i}f_i + \sum\limits_{i=1}^{n}a_i^{1}Xf_{i}+ \sum\limits_{i=1}^{n}a_i^{2}X^2f_{i}+... = \sum\limits_{i=1}^{n}h_i(X)f_i $$
    \end{enumerate}

\end{demonstration}

\begin{definition}
    \textbf{Dimensión de Krull}

    Sea $ A  $ un anillo. Se llama \textbf{dimensión de Krull} de $ A $ al número $ n \in \mathbb{N} \cup \{x\}$ tal que:
    $$ \operatorname{dim}(A) = \operatorname{Kdim}A = \sup \{n \in \mathbb{N}:\ \text{existe una cadena } P_0 \subsetneq P_1 \subsetneq ... \subsetneq P_n\ en\ \operatorname{Spec}(A)\}$$
\end{definition}

\begin{proposition}
    Si $ A $ es una anillo artiniano entonces:
    \begin{enumerate}
        \item $ \operatorname{Spec}(A) = \operatorname{MaxSpec}(A) $, osea todo ideal primo es maximal, o '' $ A $ tiene dimensión $ \operatorname{dim}(A) = 0 $.
        \item $ \operatorname{Spec}(A) = \operatorname{MaxSpec}(A) $ es finito.
        \item $ J:= \operatorname{Jac}(A) = \operatorname{Nil}(A) $ es \textbf{nilpotente}
    \end{enumerate}
    (no esta copiado entero)
\end{proposition}

\begin{demonstration}
    \noindent 1.

    $ P \in \operatorname{Spec}(A) \implies \dfrac{A}{P} $ dominio artiniano y por el ejemplo 2.4.3 tenemos que $ A/P $ es cuerpo $ \iff P \in \operatorname{MaxSpec}(A)$  

    \noindent 2.

    Definimos:

    $$ \Omega = \{I \properideal A :\ I = \text{ intersección finita de ideales maximales}\} $$

    Tenemos que $ \Omega \ne \emptyset $ porque $ A $ tiene un ideal maximal $ \implies \exists I_0 \in \Omega$ minimal $ \implies I_0 = M_1 \cap ... \cap M_{r} $.

    Sea entonces $ M \in \operatorname{MaxSpec}(A) \implies I_0 \cap M \in \Omega \implies $
    $$ \implies \left\{
    \begin{array}{l}
        I_0 \cap M \in \Omega \\
        I_0 \cap M \in I_0
    \end{array}
    \right\} \implies_{I_0\ minimal} I_0 \cap M = I_0 \iff I_0 \subseteq  M  $$

    $$  M_1\cdots M_r \subseteq M_1 \cap ... \cap M_{r} = I_0 \subseteq M \implies_{ M\ primo} \exists j :\ M_j \subseteq M \implies $$  
    $$ \implies_{M_j\ maximal} M_j = M \implies \operatorname{MaxSpec}(A ) = \{M_1,...,M_{r}\}$$

    \noindent 3.

    Como $ A $ es artiniano y se tiene la cadena descendente $ J \supseteq J^2 \supseteq J^3 \supseteq ... \implies \exists m \in \mathbb{Z}^{+}$ (minimal) con $ J^{m} = J^{m+k}\ \forall\  k\geq  0 $. Definimos $ I:= J^{m}\ (I^2 = I) $.

    Supongamos que $ I \ne 0 \implies  $
    $$ \emptyset \ne \Omega' := \{K \ideal A:\ KI \ne 0,\ K \subseteq I\} \ni I$$

    $ \implies \Omega' $ tiene un elemento minimal $ K_0 $ tal que $ K_0I \ne 0 \implies \exists x \in K_0 :\ xI \ne 0 \implies$

    $$ \left\{
    \begin{array}{l}
        (x)I \ne 0 \implies (x) \in \Omega'\\
        (x) \subseteq  K_0\ minimal
    \end{array}
    \right\} \implies (x) = K_0  $$

    Entonces $ xI \ne 0 \implies 0 \ne xI = xI^2 = (xI)I \implies xI \in \Omega'  $ y se tiene $ xI_{\in \Omega'} \subseteq (x) = K_0 \implies_{K_0\ minimal\ en\ \Omega'}\ xI = (x) = K_0$

    Entonces $ x \in xI \implies x = xy $ para cierto $ y \in I $, luego $ x = xy = xy^2 = ... = xy^{n}\ \forall  n > 0$ y además $ y \in I \subseteq  J(A) = \operatorname{Nil}(A) \implies \exists n> 0:\ y^{n} = 0 $. Entonces $ x = 0 $, lo que nos lleva a una contradicción.

    \noindent 4.

    $ J = M_1 \cap ... \cap M_{r} $, donde $ \operatorname{MaxSpec}(A) = \{M_1,...,M_{r}\} $. Como cada $ M_i $ son maximales, los $ M_i $ son comaximales dos a dos. Por tanto, tenemos que:
    $$ J = M_1 \cap ... M_{r} = M_1 \cdots M_{r} \implies 0 = J^{m} = M_1^{m}\cdots M_{r}^{m} $$


\end{demonstration}

\begin{theorem}
    \textbf{(Akizuki)}

    Sea $ A $ un anillo. Son equivalentes:
    \begin{enumerate}
        \item $ A $ es artiniano.
        \item $ A $ es noetheriano y $ \operatorname{dim}(A) = 0 $.
    \end{enumerate}
\end{theorem}

Aún no tenemos todos los conceptos necesarios para demostrar este teorema, nos dejaremos algún detalle sin resolver.

\begin{demonstration}
    
    \noindent$ 1\implies 2 $

    Ya hemos visto que $ \operatorname{dim}(A) = 0 $. Queda pendiente probar que $ A $ es noetheriano.

    \noindent $ 2\implies 1 $

    Sabemos que $ \implies \operatorname{MaxSpec}(A) = \operatorname{Spec}(A) = \operatorname{MinSpec}(A) \implies\footnote{Ejercicio 2.7} $ este conjunto es finito.

    Si tomamos $ \operatorname{Spec}(A) = \{M_1,...,M_{r}\} $ tenemos que $ J := J(A) = \operatorname{Nil}(A) = M_1\cap ... \cap M_{r} = M_1\cdots M_{r} $ (ya que los elementos son maximales dos a dos).

    Por ser $ A $ noetheriano, $ J $ es f.g y como también es nil (todos sus elementos son nipotentes), tenemos que $ J $ es nilpotente (ejercicio 2.4) $ \iff\ \exists  m > 0 $ (minimal) tal que $ J^{m} = 0 \implies 0 = M_1^{m}\cdots M_{r}^{m}$

    Utilizando ahora el teorema chino de los restos (la versión general de esta asignatura), tenemos que

    $$ 
    \begin{aligned}
        \phi: & A & \to & \dfrac{A}{M_1^{m}} \times ... \times \dfrac{A}{M^{m}_r}\\ 
        & a & \hookrightarrow & (\overline{a},...,\overline{a})
    \end{aligned}
    $$
    es un isomorfismo de anillos.

    Entonces $ A  $ es artiniano $ \iff\ \dfrac{A}{M^{m}_{i}}$ es artiniano $ \forall i = 1,...,r $ 

    Afirmamos ahora que $ A/M^{m}_{i} $ es un anillo local (noetheriano) con $ M_i/M^{m}_{i} $ como único ideal maximal ( = primo).

    Basta con ver que es el único ya que usando el teorema de correspondencia podremos ver que es maximal.

    Sea $ M / M_i^{m} $ un ideal maximal de $ A / M_i^{m} (\implies M \in \operatorname{MaxSpec}(A)) \implies  M_i^{m} \subseteq M \implies_{M\ primo} M_i \subseteq  M \implies_{M_i\ maximal}M_i = M$

    La prueba entonces queda reducida a probar que si $ A $ es un anillo noetheriano local con $ \operatorname{dim}(A) = 0 $ (y $ M $ como único ideal maximal), entonces $ A $ es artiniano.

    \begin{flushright}
        \textbf{Observación}
    \end{flushright}
    $ M $ es nilpotente $ \iff \exists q > 0\ (minimal)$ tal que $ M^{q} = 0 $
    
    \begin{flushright}
    \textbf{Observación}
\end{flushright}

$ M $ es f.g. $ \implies $ fijo $ \{x_1,...,x_{d}\} $ conjunto de generadores de $ M \implies_{ejerc.} M^{t} = (x_{i_1},...,x_{i_{t}}:\ i_1,...,i_{t} \in \{1,2,...,d\})$

\textbf{Crucial:} Cada cociente $ M^{t} / M^{t+1} $ ( en particular $ M^{q-1} = M^{q-1} / M^{q} $) es un $ A /M $-esp. vectorial con $ \{\overline{x_{i_1},...,x_{i_{t}}}\} $ como conjunto de generadores. Definimos entonces:

$$ 
\begin{aligned}
    \dfrac{A}{M} \times \dfrac{M^{t}}{M^{t+1}} &  \to &  \dfrac{M^{t}}{M^{t+1}}\\
    (a+m,y+m^{t+1}) & \hookrightarrow & ay+m^{t+1}
\end{aligned}
$$

Probad que está bien definida y transforma $ M^{t} / M^{t+1} $ es un $ A /M $-esp. vectorial.

Además, si $ y \in M^{t} \implies y= \sum\limits_{}^{}a_ix_{i_1}\cdots x_{i_{t}} \implies y+M^{t+1} = \sum\limits_{}^{} a_{i}+m (x_{i_{1}\cdots x_{i_{t}}M^{t+1}}) \implies M^{t / M^{t+1}}$ está generado como $ A / M  $-esp. vectorial por $ \{\overline{x_{i_1}\cdots x_{i_{t}}}\} $

Entonces cada $ M^{t} / M^{t+1} $ es un $ \dfrac{A}{M} $-esp. vectorial de dimensión finita.

Entonces $ M^{q} = 0 \ne M^{q-1} $. Probaremos por inducción en $ q\geq  1 $ que $ A  $ es artiniano.

Si $ q = 1 \implies M = 0 \implies A = A / M  $ es un cuerpo y terminamos.

Sea $ q > 1 $ y lo suponemos cierto para $ q -1 \implies A /M^{q-1}$ es artiniano.

Sea $ I_0 \supseteq I_1 \supseteq ... \supseteq I_n \supseteq...$ una cadena descendente de ideales de $ A $. Entonces:
$$ \left\{
\begin{array}{ll}
    \dfrac{I_0+M^{q-1}}{m^{q-1}} \supseteq \dfrac{I_1+^{q-1}}{m^{q-1}} \supseteq ...& \text{se estaciona por ser $ A / M^{q-1} $ artiniano}\\ 
    I_0 \cap M^{q-1} \supseteq I_1 \cap M^{q-1 } \supseteq ... & \text{se estaciona por ser $  M^{q-1}$ un $ A /M $-esp. vectorial de dimensión finita}
\end{array}
\right. $$

Entonces $ \exists s > 0 $ tal que:
$$ 
\left\{
    \begin{array}{l}
    I_{s}+M^{q-1} = I_{s+k} + M^{q-1}\\
    I_{s}\cap M^{q-1} = I_{s+k} \cap M^{q-1}\\
\end{array}
\right\} \forall k \geq  0
$$

Se trata ahora de probar que si $ I,J \ideal A:\ I \supseteq J $ y se tiene
$$ \left\{
    \begin{array}{l}
        I + M^{q-1} = J+M^{q-1}\\
    I \cap  M^{q-1} = J\cap M^{q-1}\\
\end{array}
\right. \implies? I =  J$$

Sea $ y \in I \subseteq  I + M^{q-1}= J+ M ^{q-1} \implies y = z+h $ donde $ h \in M^{q-1},\ z \in J \implies h = \underbrace{y}_{\in I}-\underbrace{z}_{\in J \subseteq I} \implies h \in I \cap M^{q-1} = J \cap M^{q-1} \implies h \in J \implies y = z+h \in J  $

\end{demonstration}


\section{Ejercicios}

\setcounter{ex}{0}
\begin{exercise}$  $

    \begin{flushright}
        \textbf{Apartado c)}
    \end{flushright}
    $$ X \subseteq  (X) \implies_{2.1.b)} (I: (X)) \subseteq  (I:X) $$

    Probamos que $ (I:X) \subseteq  (I:(X)) $.

    Sea $ a \in (I:X) \iff ax \in I\ \forall x \in X $.

    Quiero probar que si $ z \in (X) \implies az \in I $:
    $$ z \in (X) \iff z = \sum\limits_{i=1}^{n}b_ix_i\ (b_i \in A,x_i \in X) \implies az = \sum\limits_{i=1}^{n}b_iax \implies az \in I$$


    \begin{flushright}
        \textbf{Apartado e)}
    \end{flushright}
    
    \noindent 1.

    Sea $ a \in A $:
    $$ a \in ((I:X):Z) \iff aZ \subseteq  (I:X) \iff az \in (I:X)\ \forall z \in Z \iff (az)X \subseteq I\ \forall  z \in Z  $$
    $$ (az)x = a(zx) \in I\ \forall z \in Z,\ \forall x \in X \iff aw \in I\ \forall w \in X\cdot Z = Z\cdot X \iff a \in (I:X\cdot Z)$$

    \begin{flushright}
        \textbf{Apartado f)}
    \end{flushright}
    Sea $ a \in A $:
    $$ a \in (I: \bigcup_{t} X_{t}) \iff az \in I\ \forall  z \in \bigcup_{t}X_{t}\iff az \in I\ \forall z \in X_{t} \text{ con }t \in T \text{ arbitrario} $$
    $$ \iff a \in (I:X_{t})\ \forall t \in T \iff a \in \bigcap_{t \in T} (I:X_{t})$$

    \noindent 2.

    Sabemos que $ \sum\limits_{t \in T}^{} J_{t} = \left( \bigcup_{t \in T}J_{t} \right) $ y aplicando el apartado c) y el caso anterior:
    $$ \left(I: \sum\limits_{t \in T}^{} J_{t}\right) = \left(I : ( \bigcup_{t \in T} J_{t})\right) = \left( I : \bigcup_{t \in T} J_{t} \right) ) \bigcap_{t \in T}(I:J_{t}) $$




\end{exercise}

\begin{exercise}
    Hay una errata en el tercer caso del a). El enunciado correcto es:
    $$ \left(\dfrac{J}{I}\right) \left(\dfrac{J'}{I}\right) = \dfrac{JJ'+I}{I} $$


\end{exercise}

\begin{exercise}
    $  $

    \noindent $ \supseteq $

    Directa. Basta con ver que el conjunto de generadores de $ (X\cdot Y) $ está contenido en $ (X)(Y) $, que es directo.

    \noindent $ \subseteq  $

    $ (X)(Y) = ((X)\cdot (Y)) \implies $ sus elementos son las sumas $ \sum\limits_{i=1}^{n}z_iw_i   $ donde $ z_i \in (X),\ w_i \in (Y) $. Entonces $ z_i \in (X) \implies z_i = \sum\limits_{j=1}^{m_i}a_{ij}x_j $ donde $ a_{ij} \in A,\ x_j \in X $ y $ w_{j} \in (Y) \implies w_i = \sum\limits_{k=1}^{q_{i}}b_{ik} y_{k} $, donde $ b_{ik} \in A,\ y_{k}\in Y $ $ \implies z_iw_i = \sum\limits_{j,k}^{} a_{ij}b_{ik}x_jy_{k} \in (X\cdot Y) $

\end{exercise}

\begin{exercise}$  $

    Tenemos que $ I = (b_1,...,b_n) $. Hacemos inducción en $ n \geq 1 $.  

    Para $ n = 1 $, $ I = (b_1) $. Como $ b_1 $ es nilpotente, $ \exists m > :\ b_1^{m} = 0 \implies I^{m} =0 $

    Sea $ n > 1 $ y cierto para ideales nil generados por menos de $ n $ elementos. Si tomamos $ I' = (b_1,...,b_{n-1}) $, $ I' $ es nil ($ I' \subseteq I $). La hipótesis de inducción nos da además un $ p > 0 $ entero tal que $ I'^{p} =0$.

    Por otra parte, $  (b_n)^{q} = 0 $ para un cierto entero $ q > 0 $. Observamos además que $  I = I'+(b_n) $. ¿Existe entonces $ m $ tal que $ I^{m} = 0 $? Esto ocurre si y solo si $ \forall y_1,...,y_m \in I $ se tiene que $ y_1\cdots y_m  = 0$

    Ahora, podemos poner $ y_i = y_i' + z_i $ con $ y'_{i} \in I' $ y $ z_i \in (b_n) $. Si tomamos $ m = p+q $ entonces $ (y_i+z_i)^{m} = 0\ \forall i $. Luego $ I $ es nilpotente.
\end{exercise}

\begin{exercise}$  $

    Tomamos el cociente $ \dfrac{A}{I} $, tenemos que $ \operatorname{Nil}\left( \dfrac{A}{I} \right) = \dfrac{\sqrt{I}}{I} $ es nil y f.g. Utilizando ahora el ejercicio anterior, tenemos que $ \exists m $ tal que $ \left( \dfrac{\sqrt{I}}{I} \right)^{m} $. Entonces:
    $$ 0 = \left( \dfrac{\sqrt{I}}{I} \right)^{m} = \dfrac{(\sqrt{I})^{m}+I}{I} \implies (\sqrt{I})^{m} \subseteq  I  $$
\end{exercise}

\begin{exercise}$  $

    Si $ x \in \bigcap _{n>0} (b^{n}) \implies \forall  n > 0,\ \exists x_n \in A$ tal que $ x = b^{n}x_n \implies \forall  n > 0  $ se tiene:
    $$ \cancel{b^{n}}x_n = x = b^{\cancel{n}+1}x_{n+1} \implies x_n = bx_{n+1} $$
    $$ \implies (x_1) \subseteq  (x_2) \subseteq ... \subseteq (x_n) \subseteq ...  $$

    Luego tenemos una cadena ascendente en $ \mathcal{L}(A) $, pero como $ A $ es noetheriano, la cadena se estaciona, es decir, $ \exists m > 0 $ tal que $ (x_m) = (x_{m+1})\ \forall k \geq  0 $

    En particular, tenemos que:
    $$ x_{m+1} \in (x_{m}) \implies \exists c \in A:\ x_{m+1} = cx_{m} \implies x_{m} = b x_{m+1} = bcx_m $$

    Tenemos ahora dos casos:
    
    Si $ x $ es cancelable, $ x_m $ es cancelable (siguiente línea) $ \implies bc = 1 \implies b \in \mathcal{U}(A)$ (contradicción).
    
    Si $ x_n $ no fuese cancelable, entonces existe $ y_n \in A \setminus \{0\} $ tal que $ x_ny_n = 0 \implies xy_n = 0 $ (contradicción ya que $ x $ es cancelable)

    Si $ x $ no fuera cancelable, ¿podríamos tomar $ x_n = x_{n+1}\ \forall n > 0$? Le llamamos $ y $ a ese elemento. $ x = by = b^2 y = ... \implies_{b\ cancelable} y = by $

    
    Si probamos que $ \overline{y} \ne \overline{0} $ y que $ \overline{z} $ es cancelable en $ A $, entonces $ 0 \ne \overline{y} \in \bigcap_{n > 0}(\overline{z}^{n}) $

    Veamos primero que $ \overline{y} \ne \overline{0} $. Supongamos que $ \overline{y} = \overline{0} \implies y \in (y(1-z)) \implies y = y(1-z)f(y,z)$ con $ f \in \operatorname{K}[y,z] $. COmo $ y $ es cancelable, $ (1-z)f(y,z) = 1 \implies 1-z \in \mathcal{U}(\operatorname{K}[y,z]) $. Lo cual es una contradicción, las unidades de un anillo de polinomios son las unidades del ''anillo origen''.

    Supongamos ahora que $ \overline{z} $ no es cancelable. Lo que es equivalente a  que $ \overline{z} $ sea divisor de cero en $ A $. Entonces $ \exists g \in \operatorname{K}[y,z] $ tal que $ \overline{z}\cdot \overline{g} = \overline{0},\ \overline{g} \ne 0 \iff $
    $$ \iff zg(y,z) \in (y(1-z)) \iff \exists h = h(y,z):\ zg(y,z) = y(1-z)h(y,z)\ (*)$$

    Entonces $ zg(y,z) \in (y) $, $ z \not \in (y) $. Luego $g(y,z) \in (y) \implies g(y,z) = y\widetilde{g}(y,z) \implies (*) z\widetilde{g} (y,z) = (1-z)h(y,z)\ (**)$

    $$ \left\{
    \begin{array}{l}
        (1-z) h(y,z) \in (z) \\
        1-z \not \in (z)
    \end{array}
    \right\} \implies h(y,z) \in (z) \iff h(y,z) = z\widetilde{h}(y,z) \text{ para algún } \widetilde{h} \in \operatorname{K}[y,z] \implies $$ 
    
    $$ \implies (**) \widetilde{g}(y,z) = (1-z)\widetilde{h}(y,z) $$

    Entonces tenemos:
    $$ g(y,z) = y\widetilde{g}(y,z) = y(1-z)\widetilde{h}(y,z) \implies g(y,z) \in (y(1-z))  \iff \overline{g} = \overline{0} \text{ (contradicción)}$$

\end{exercise}


\begin{exercise}
    $  $

    Si aplicamos la proposición 2.7, tenemos que $ (0) =  P_1\cdots P_{r} $, donde los $ P_i $ son primos (quizá algunos repetidos). Sea $ P \in \operatorname{MinSpec}(A) \implies (0) = P_1 \cdots P_{r} \subseteq  P$. Pero como $ P $ es primo, $ \exists j  $ tal que $ P_j \subseteq  P $ y como $ P $ es minimal en $ \operatorname{Spec}(A) $, $ P_j = P \implies P \in \{P_1,...,P_r\} $. Entonces el número de primos minimales es finito (hay hasta $ r $)

    Para la segunda parte, tomamos $ I \properideal A  $ y usamos el teorema de la correspondencia para ideales primos:
    $$ 
    \begin{aligned}
        \{P \in \operatorname{Spec}(A):\ I \subseteq P\} & \xrightarrow{biyecci\acute on} & \operatorname{Spec}(\dfrac{A}{I})\\
        P & \longmapsto & \dfrac{P}{I}\\ 
        \left\{\stackrel{primos\ minimales}{sobre\ I}\right\} & \leftrightarrow & \operatorname{MinSpec}\left(\dfrac{A}{I}\right)\  (finito)
    \end{aligned}
    $$
\end{exercise}

\begin{exercise}
    $  $

    Visto en la prueba del teorema de Akizuki.
\end{exercise}

\begin{exercise}
    $  $

    Si $ a \in \mathcal{U}(A    ) \implies a = u = up^{0}$

    Si $ 0 \ne a \not \in \mathcal{U}(A)$:
    $$ \implies (a) \subsetneq A \implies (a) \subseteq J = (p) \implies a = pa_1 \implies a \in (p) \setminus \bigcap_{n \in \mathbb{N}} (p^{n}) \implies $$
    $$ \implies \{n \in \mathbb{N}:\ a \in (p^{n})\} \implies \exists m \text{ maximal con } a \in (p^{m})$$

    Entonces $ a = p^{m}u $ y basta probar que $ u \in \mathcal{U}(A) $.

    Si $ u \not \in \mathcal{U}(A) \implies (u) \subseteq  J = (p) \implies u=pv $, con $ v \in A \implies a = p^{m}u = p^{m}(pv) = p^{m+1}v \implies a \in (p^{m+1})$ lo cual es una contradicción.

    \begin{flushright}
        \textbf{Observación}
    \end{flushright}
    Hemos probado que todo ideal principal de $ A $ es 0 o de la forma $ (p^{n}) $ con $ n \in \mathbb{N} $
    
    Entonces tenemos:
    $$  A = (p^{0}) \supsetneq (p^{1}) \supsetneq ... \supsetneq (p^{n}) \supsetneq ... $$

    Sea ahora $ I \ideal A $ f.g. $ \implies I = (x_1,...,x_n) = \sum\limits_{i=1}^{n} (x_i)  $
    
    Si suponemos que $ (x_i) = (p^{m_i}) $ y suponemos $ m_1 \geq  m_2 \geq  ... \geq  m_{n} $, entonces $ \sum\limits_{i=1}^{n}(x_i) = \sum\limits_{i=1}^{n} (p^{m_i}) = (p^{m_n}) $

    Supongamos que $ I \properideal A $ que no es finitamente generado.

    Tomamos $ y_1 \in I \setminus \{0\} $ arbitrario, entonces $ (y_1) \subsetneq I \implies \exists y_2 \in I \setminus (y_1) \implies (y_1,y_2) \subsetneq I\ ...$. Construimos así una cadena ascendente:
    $$
    \begin{aligned}
        0&\ne&(y_1)&\subsetneq&(y_1,y_2)&\subsetneq&...&\subsetneq&(y_1,...,y_n)&\subsetneq... \\
        & & (p^{m_1})& \subsetneq &  (p^{m_2}) & \subsetneq&...&\subsetneq&(p^{m_{n}})&\subsetneq... \\ 
    \end{aligned}
    $$
\end{exercise}

\setcounter{chapter}{3}
\chapter{Módulos}

Está haciendo la introducción bastante rápido, quedan apuntados los conceptos que introduce. Los ha visto conforme a los apuntes de Alberto.

\begin{definition}\textbf{Módulo}

    
    % Si $ A $ es un anillo, $  M $ es un  $A-$módulo si es un grupo abeliano y tiene
    
\end{definition}

\begin{definition}
    \textbf{Submódulo}
\end{definition}

\begin{flushright}
    \textbf{Notación}
\end{flushright}

Al conjunto de submódulos del $ A- $módulo $ M $ lo denotamos por $ \mathcal{L}(_{A}M) $

\begin{proposition}
    \textbf{Extraída de los ejemplos 4.9}

    Un A-módulo $ M$ es cíclico sii es isomorfo a $ \dfrac{A}{I}$, para un ideal $ I$ de $ A$.

\end{proposition}

\begin{demonstration}

    Si $ \dfrac{A}{I}$ es cíclico generado por $ \overline{1}=1+I$ ya que $ a+I = a(1+I)$ 

    Si $ M$ es cíclico, entonces $ M =(x) = Ax = \{ax:\ a \in A\}$. Si defino $ f:_{a}A\to M= Ax$ de forma que $ a\mapsto f(a) = ax$ tenemos un epimorfismo de $ A$-módulos.

    Por el teorema de isomorfía, tenemos que $ \dfrac{A}{\operatorname{Ker}(f)}\cong M$. Siendo $ \operatorname{Ker}(f)$ un ideal de $ A$.

    \begin{flushright}
        \textbf{Observación}
    \end{flushright}
    
    $$ \operatorname{Ker}(f) = \{a \in A:\ ax = 0\} = an n_{A}(x) \stackrel{M\ cicl.}{=} an n_{A}(M) $$ 

    Para $ \subseteq $ en la última igualdad necesitamos probar que si $ \underbrace{ax = 0}_{a \in an n_{A}(x)} \implies \underbrace{a(bx)=0}_{a \in an n_{A}(M)}\ \forall b \in A$
\end{demonstration}
\setcounter{propositiont}{9}

\begin{proposition}
    \textbf{Proposición-Definición}

    Sea $ (M_i)_{i \in I}$ una familia de submódulos del A-módulo $ M$. Decimos que es una familia independiente (de submódulos) cuando satisface cualquiera de las siguientes condiciones equivalentes:
    \begin{enumerate}
        \item La expresión de un $ x \in \sum\limits_{i \in I}^{}M_i$ como $ x=\sum\limits_{i \in I}^{}x_i$ (\textbf{suma finita}) con $ x_i \in M_i\ \forall  i \in I,$ es única.
        \item Si 0 = $ \sum\limits_{i=I}^{x_i}$ (\textbf{suma finita}) con $ x_i \in M_i\ \forall  i \in I$, entonces $ x_i =0 \forall i \in I$.
        \item $ \forall j \in I$, se tiene que $ M_j \cap \left( \sum\limits_{i\ne j}^{} M_i \right) = 0$
    \end{enumerate}
\end{proposition}


\begin{demonstration}
    
    \noindent $ 1\implies 2$.

    Trivial.

    \noindent $ 2 \implies 1$.
    $$ \left\{
    \begin{array}{l}
        x = \sum\limits_{}^{}x_i \\ 
        x = \sum\limits_{}^{}x_i'
    \end{array}
    \right\} \stackrel{Suma\ finita}{\implies} 0 = \sum\limits_{i=I}^{} (x_i- x_i') \implies x_i-x_i' = 0 $$

    \noindent $ 3 \implies 2$.

    Si $  0 = \sum\limits_{}^{}x_i \implies \forall j \in I$ tenemos que $ x_j = \sum\limits_{i\ne j}^{}(-x_i) \implies x_j \in M \cap \left( \sum\limits_{i \ne j}^{}M_i \right) \stackrel{3)}{=} 0 \implies x_j = 0\ \forall j \in I$

    \noindent $ 1,2 \implies 3$.

    Sea $ x \in M_j \cap \left( \sum\limits_{i\ne j}^{}M_i   \right) \implies x = \sum\limits_{i\ne j}^{}x_i$, con $ x_i \in M_i\ \forall  i \in I \setminus \{j\} \implies 0 = \sum\limits_{i \ne j}^{}x_i + (-x) \stackrel{2)}{\implies} -x = 0 \iff x = 0$
\end{demonstration}


Cuando $ (M_i)_{i \in I}$ es una familia independiente de submódulos de $ M$, la suma $ \sum\limits_{i \in I}^{}M_i$ suele denotarse por $ \bigoplus\limits_{i \in I}^{int}M_i = $ suma directa interna de los $ M_i$.

Recordemos que se tiene la suma directa externa $ \bigoplus\limits_{i \in I}^{ext} M_i = \{ (x_i) \in \prod_{i \in I}^{}M_i :\ x_i = 0\ \forall i \in I  \}$.

En general, si $ (M_i)_{i \in I}$ es una familia de submódulos de $ M$, se tiene un homomorfismo inducido:
$$ 
\begin{aligned}
    \phi: & \bigoplus\limits_{i \in I}^{ext}M_i &  \to &  M\\ 
    & (x_i) & \mapsto & \sum\limits_{}^{}x_i
\end{aligned}
$$

Cuya imagen es $ \sum\limits_{i \in I}^{}M_i$, es decir $ \operatorname{Im}(\phi) = \sum\limits_{i \in I}^{}M_i$

Se tiene que $ \phi$ es un monomorfismo $ \iff (M_i)_{i \in I}$ es una familia independiente. En tal caso induce un isomorfismo entre la suma externa y la interna. Por tanto, obviaremos el superíndice ext o int. 

\setcounter{propositiont}{11}

\begin{proposition}
    Sea $ (M_i)_{i \in I}$ una familia independiente de submódulos de $ M$ tal que $ M = \bigoplus\limits_{i \in I}M_i$. Para cada $ j \in I$, se tiene:
    \begin{enumerate}
        \item $ M_j \cong \dfrac{\bigoplus M_i}{\bigoplus\limits_{i \ne j}}$
        \item $ \bigoplus\limits_{i \ne j} M_i \cong \dfrac{M}{M_j}$
    \end{enumerate}

    Como caso particular, se tiene que si $ M = N \oplus N'$, entonces:
    $$ \dfrac{M}{N'} \cong N \hspace{15mm} \dfrac{M}{N} \cong N' $$
\end{proposition}

\begin{demonstration}
    
    \noindent 1.
    
    Utilizamos la proyección de $ M$ a $ M_j$ que tiene por núcleo $ \bigoplus\limits_{i \ne j}M_i$

    \noindent 2.

    De nuevo, tomamos la proyecctión de $ M $ a $ \bigoplus\limits_{i \ne j} M_i$ cuyo núcleo es $ M_j$
\end{demonstration}

\noindent\textbf{Vemos ahora una proposición que no está incluida en los apuntes de Alberto del Valle}

\begin{flushright}
    \textbf{Observación previa}
\end{flushright}

    Si $ M$ es un $ A$-módulo, entonces $ \operatorname{End}_{A}(M)$ es un anillo no conmutativo en general (con la composición como producto).

\begin{proposition}
    $ M$ es indescomponible sii los únicos idempotentes de $ \operatorname{End}_{A}(M)$ son $ 0$ y $ 1_{M}$
\end{proposition}

\begin{demonstration}
    \noindent$ \implies$

    Sea $ \varepsilon \in \operatorname{End}_{A}(M)$ idempotente $ (\implies 1_{M}- \varepsilon$ también lo es) $ \stackrel{?}{\implies} M = Im(\varepsilon) \oplus Im(1_{M}-\varepsilon)$

    Si eso está probado, entonces al ser $ M$ indescomponible $ Im(\varepsilon) = 0$ o $ Im(1_{M}- \varepsilon) = 0 \iff \varepsilon = 0$ o $ 1_{M}-\varepsilon = 0$

    $$  M = Im(\varepsilon)+ Im(1-\varepsilon):\ x = \varepsilon(x) + (1_{M}-\varepsilon)(x) $$

    $$ x \in Im(\varepsilon) \cap Im(1-\varepsilon) \implies \left\{
    \begin{array}{ll}
        x = \varepsilon(y) & \implies (1_{M}-\varepsilon)(x) = \underbrace{[(1_{M}-\varepsilon)\cdot \varepsilon]}_{0}(x)\\ 
        y\\ 
        x = (1_{M}-\varepsilon)(z) & \implies \varepsilon(x) = 0
    \end{array}
    \right. $$

    \noindent $ \impliedby$

    Si $ M = N \oplus N'$ (suma directa interna), entonces:

    $$ 
    \begin{aligned}
        \varepsilon_{N}: & M = &   N \oplus N' & \to & N & \hookrightarrow & M\\ 
        & & v+v' & \mapsto & \mapsto & v+0
    \end{aligned}
    $$

    Entonces $ \varepsilon_{N}$ es idempotente, luego $ \varepsilon_{N} = 0\ (\iff N = 0)$ o bien $ \varepsilon_{N} = 1_{M}\ (\iff N = M)$

\end{demonstration}

\begin{lemma}
    Sea $ 0 \ne M$ un $ A$-módulo cíclico, entonces $ M$ es indescomponible sii los únicos idempotentes del anillo $ \dfrac{A}{an n_{A}(M)}$ son $ \overline{0},\overline{1}$.

    Si tenemos entonces un isomorfismo en $ _{A}Mod: \dfrac{A}{an n_{A}(M)}\cong M$, entonces $ \operatorname{End}_{A}(M) \cong \operatorname{End}_{A} \left( \dfrac{A}{an n_{A}(M)} \right)$

\end{lemma}


\begin{exercise}
    Si $ I \properideal A$, entonces la aplicación $ \dfrac{A}{I} \xrightarrow{\mu} \operatorname{End}_{A}(A/I)$ $ (\mu_{\overline{a}}\overline{b} \mapsto \overline{ab} = \overline{a}\cdot \overline{b})$ es un isomorfismo de anillos.

    Que sea un homomorfismo es directo, vemos que:
    $$ \operatorname{Ker}(\mu) = \{\overline{a}: \mu_{\overline{a}} \equiv 0\} = \{\overline{a} \in \dfrac{A}{I}:\ \overline{ab} = \overline{0}\ \forall \overline{b} \in \dfrac{A}{I}\ (\implies \overline{a} = \overline{a}\overline{1} = \overline{0})\} \implies $$
    $$ \operatorname{Ker}(\mu) = 0 \implies \mu \text{ inyectiva} $$

    Vemos que $ \mu$ es suprayectivo. Sea $ f \in \operatorname{End}_{A}\left(\dfrac{A}{I}\right)$ de forma que $ f(\overline{1}) = \overline{a}$. Entonces tenemos que $ \overline{b} = b \overline{1} \mapsto bf(\overline{1}) = b \overline{a} = \overline{b} \overline{a} \implies f = \mu_{\overline{a}}$. Luego $ \mu$ es sobre.
\end{exercise}


\begin{exercise}
    \textbf{Ejercicio planteado}    

    Sea $ M \in \operatorname{MaxSpec}(A)$ y $ n>0$ un entero. Probar que el $ A$-módulo $ A/M^{n}$ es indescomponible.
\end{exercise}

\begin{exercise}
    \textbf{Ejercicio planteado}

    Sea $ a \in A \setminus (\mathcal{U}(A) \cup \{0\})$, donde $ A$ es un DIP. Probar:
    $$ \dfrac{A}{(a)} \text{ indescomponible} \iff a\text{  es asociado a }p^{t}\text{, para algún }p \in A \text{ irreducible y algún } t>0 $$
\end{exercise}

\noindent \textbf{Definición previa a la proposición 4.26}

\begin{definition}
    \textbf{Sucesión exacta corta}

    Se dice que una sucesión de $ A$-módulos y $ A-$-homomorfismos $ 0\to L \to M \to N\to 0$ es una \textbf{sucesión exacta corta} si el núcleo de cada morfismo es la imagen del que la precede. Esto es equivalente a:
    $$ \left\{
    \begin{array}{ll}
        g & epimorfismo\\ 
        f & monomorfismo\\ 
        \operatorname{Im}(f) = \operatorname{Ker}(g)
    \end{array}
    \right. $$
\end{definition}

\begin{exercise}
    $ $
    
    Toda sucesión exacta corta con término central $ M$ es isomorfa a una del estilo:
    $$ 0 \to K \hookleftarrow M \xrightarrow{\pi} M/K \to 0$$

    Donde $ \hookleftarrow$ es la inclusión desde un submódulo y $ \pi$ es la proyección sobre el cociente.
\end{exercise}

\setcounter{propositiont}{26}

\begin{corollary}
    $ \bigoplus\limits_{i=1}^{n} M_i$ es noeth. (resp. artiniano) si y solo si todos los $ M_i$ son noeth. (resp. artinianos)

\end{corollary}

\begin{demonstration}

    Se reduce trivialmente al caso $ n = 2$. Vemos que $ N_1 \oplus N_2 $ noeth $ \iff N_1,\ N_2$ lo son. Sabemos que:
    $$ N_2 \cong \dfrac{N_1 \oplus N_2}{N_1} $$

    Lo cual da la prueba de forma directa.
\end{demonstration}

\begin{corollary}
    \textbf{Apartado a)}

    Sea $ A$ anillo. Son equivalentes:
    \begin{enumerate}
        \item $ A$ anillo noeth. (resp. artiniano)
        \item Para algún (resp. todo) entero $ n>0$, el $ A$-módulo libre $ A^{n}$ es noeth. (resp. artiniano).
        \item Todo $ A$-módulo fin. generado es noeth. (resp. artiniano)
    \end{enumerate}
\end{corollary}

\begin{flushright}
    \textbf{Observación previa a la prueba}
\end{flushright}

Como en los dos casos de este corolario hay una parte fuerte y una débil en 2), para probar esto hay que hacer el caso fuerte para $ 1\implies 2$ y el débil para $ 2\implies 3$

\begin{demonstration}
    \noindent $ 1\implies 2$.

    Hay que probar que $ \forall n > 0 _{A}A^{n}$ es noeth (sale por el corolario anterior).

    \noindent $ 2 \implies 1$.

    Suponemos que $ \exists n>0 $ tal que $ _{A}A^{n}$ noeth. $ \implies _{A}A$ noeth.

    \noindent $ (1,2) \implies 3$.

    $ \exists $ epimorfismo $ \pi: _{A}A^{n} \to M$ y $ _{A}A^{n}$ noeth. $ \implies$ M noeth.

    \noindent $ 3\implies 1$.

    Trivial
\end{demonstration}

\setcounter{propositiont}{27}
\newpage
\begin{corollary}
    \textbf{Apartado b)}

    Sea $ A$ un anillo noeth. (resp. artiniano) y sea $ f: A \to B$ un homomorfismo de anillos tal que $ B$ es f.g. como $ A$-módulo (con la restricción de escalones). Entonces $ B$ es anillo noeth. (resp. artiniano)
\end{corollary}

\begin{demonstration}
    $$ \mathcal{L}(_{B}B) \subseteq \mathcal{L}(_{A}B) $$

    El apartado a) nos dice que $ _{A}B$ es noeth. (resp. artiniano). Entonces sale ''directamente'' la prueba.
\end{demonstration}

\begin{exercise}
    $ $

    Sea $ A = A_1 \times ... \times A_n$ producto finito de anillos. Probar que todo $ A$-módulo es isomorfos aun producto $ M_1\times... M_n$, donde cada $ M_i$ es un $ A_i$-módulo. En particular:
    $$ \mathcal{L}(_{A}M) \cong \mathcal{L}(_{A_1}M_1) \times ... \times\mathcal{L}(_{A_n}M_n) $$
\end{exercise}

\begin{lemma}
    \textbf{Lema de Artin}
\end{lemma}

\begin{demonstration}
    Sean $ 0 = m_1^{n_1} \cdots m_{r}^{n_{r}}$, donde los $ m_i$ son maximales distintos y los $ n_i > 0.$ Aplicamos entonces el teorema chino de los restos.
    $$ A \cong \dfrac{A}{m_1^{n_1}} \times ... \times \dfrac{A}{m_{r}^{n_{r}}} $$

    Entonces si $ A$ es un anillo y $ m \in \operatorname{MaxSpec}(A) \implies$ $ \dfrac{A}{M^{n}}$ es un anillo local (con un único ideal maximal $ \dfrac{m}{m^{n}}$)

    La prueba se reduce ahora al caso en que $ A$ es un anillo local y su ideal maximal $ M$ satisface $ m^{n} = 0$, para algún $ n>0$. Usamos inducción en $ n$.

    Si $ n = 1$, entonces $ A$ es un cuerpo (sus únicos ideales son $ 0$ y $ A$)

    Si $ n>1$ y se cumple para $ n-1$. Consideramos la sucesión exacta corta:
    $$ 0 \to m^{n-1} M \to M \to \dfrac{M}{m^{n-1}M} \to 0 $$

    Donde $ \dfrac{M}{m^{n-1}M}$ es un $ \dfrac{A}{m^{n-1}}$-módulo. Y además, $ m^{n-1}M$ es un $ \dfrac{A}{m}$-esp. vectorial
\end{demonstration}

Con esto podemos completar la demostración del teorema de Akizuki.

\begin{demonstration}
    $$ A \text{ artiniano} \iff _{A}A\ artiniano \implies_{lem.\ Art}\  _{A}A\  noeth. \iff A\ anillo\ noeth.  $$

\end{demonstration}


\begin{definition}
    Un $ A$-módulo $ M$ se dice de longitud finita si es noeth. y artiniano.
\end{definition}

\setcounter{propositiont}{30}

\begin{corollary}
    Un anillo $ A$ es artiniano sii todo $ A$-módulo f.g. es de longitud finita.
\end{corollary}

\begin{demonstration}
    Akizuki $ \implies A$ es noeth. $ \implies $ todo $ A-$módulo y f.g es noeth. y artiniano.
\end{demonstration}

\section{Ejercicios}
\setcounter{ex}{0}

\begin{exercise}
    % $$ M \in _{A} Mod \implies \left\{
    % \begin{array}{ll}
    %     (M,+) & \text{es un gr. abeliano}\\ 
    %     A \xrightarrow{\mu} End_{\mathbb{Z}}(M) & \text{ es hom. de anillos}
    % \end{array}
    % \right. $$

    % Donde $ \mu(a): M \to M$ y $ \mu(a)(x) = ax$ y $ End_{\mathbb{Z}}(M)$ son los endomorfismos de $ M$ como grupo abeliano. 

        $$ \mu(a+b)(x) = (a+b)(x) = ax + b = \mu(a)(x) + \mu(b)(x) \implies \mu(a+b) = \mu(a)+ \mu(b) $$

        $$ \mu(ab)(x) = (ab)(x) = a(bx) = \mu(a)(\mu(b)(x)) = (\mu(a) \circ \mu(b)) (x) \implies \mu(ab) = \mu(a) \circ \mu(b)$$

        $$ \mu(1)(x) = 1x = x \forall x \in M \implies \mu(1) = 1_{M} $$

        Sea $ (M,f)$ un par formado donde $ M$ es un grupo abeliano y $ f: A \to End_{\mathbb{Z}}(M)$ es un homomorfismo de anillos. Entonces $ M$ adquiere una estructura de $ A$-módulo donde el producto es $ A \times M \to M$ que viene definido por $ (a,x) \mapsto ax: = f(a)(x)$. Hay que probar varias propiedades:
        $$ a(x+y) = f(a) (x+y) =\footnote{$ f$ es un homomorfismo de grupos abelianos} f(a)(x) + f(a)(y) = ax+ay  $$

        El resto de propiedades, como este, son rutinarias.

\end{exercise}

\begin{exercise}
    $ $

    ''Pura rutina''
\end{exercise}

\begin{exercise}
    $ $

    \begin{flushright}
        \textbf{Apartado a)}
    \end{flushright}
    

    $$ X = \{x_j:j \in J\} $$

    $$ m, m' \in IX \implies \left\{
    \begin{array}{l}
        m = \sum\limits_{j \in J}^{}a_jx_j,\  con\ a_i \in I, \forall i \in I\ y\ a_i = 0\ \forall  i \in J  \\ 
        m' = \sum\limits_{j \in J}^{} a_j'x_j ,\ ...
    \end{array}
    \right. $$

    $$ m+m' = \sum\limits_{}^{} (a_j+a_j') x_j \in IX $$
    $$ b \in A\  bm = b \sum\limits_{j \in J}^{}a_jx_j = \sum\limits_{j \in J}^{}(ba_j)x_j \in IX $$

    \begin{flushright}
        \textbf{Apartado b)}
    \end{flushright}
    
    Tomamos $ SN = \{m \in M:\ m = \sum\limits_{j \in J}^{} s_jx_j,\ con\ s_j \in S,\ x_j \in N\}$ que es un $ A-$submódulo de $ M$. El resto es parecido al a).

\end{exercise}

\begin{exercise}
    ''Rutinario''
\end{exercise}

\begin{exercise}
    $ $

    Tendremos que probar que $ \phi$ conserva la suma y la multiplicación por elementos de $ K[X]$.

    $$ \phi\left[ \left( \sum\limits_{i=0}^{n}\lambda_iX^{i} \right)v\right] \stackrel{?}{=} \left( \sum\limits_{i=0}^{n}\lambda_iX^{i}  \right) \phi(v) \hspace{10mm} \forall v \in V_1\  \forall  \sum\limits_{i=0}^{n}\lambda_iX^{i} \in K[X]  $$


    Tenemos
    $$ \phi\left[ \left( \sum\limits_{i=0}^{n}\lambda_iX^{i} \right)v\right] = \phi \left( \sum\limits_{i=0}^{n}\lambda_if_{1}^{i}(v)  \right) =_{\phi\ K-lineal}\sum\limits_{i=0}^{n}\lambda_i \phi \left( f_{1}^{i}(v) \right) = \sum\limits_{i=0}^{n}\lambda_i (\phi \circ f_{1}^{i})(v)   $$
    $$ = \sum\limits_{i=0}^{n}\lambda_i (f_{2}^{i}\circ \phi)(v) = \sum\limits_{i=0}^{n}\lambda_if_{2}^{i}(\phi(v)) = \left( \sum\limits_{i=0}^{n}\lambda_iX^{i}  \right) \phi(v)   $$
\end{exercise}


\begin{exercise}
    $ $

    \noindent $ \impliedby$

    Sea $ f:\dfrac{A}{I} \to \dfrac{A}{J}$ un $ A$-homomorfismo. 

    ¿Cómo son los $ A$-homomorfismos $ f: \dfrac{A}{I}\to N$, donde $ N \in  _{A\operatorname{Mod}}$ ?

    Vienen unívocamente determinados por la imagen del $ \overline{1}$, $ f(\overline{1}) \in \{y \in N: Iy = 0\}$

    Tenemos entonces una aplicación inducida:
    $$ 
    \begin{aligned}
        \psi: & \operatorname{Hom}_{A}\left(\dfrac{A}{I},N\right) & \to & \{y \in N: Iy = 0\}\\ 
        & f & \mapsto & f(\overline{1})
    \end{aligned}
    $$

    Vemos que esta aplicación tiene inversa: dado $ y \in \{y \in N:\ Iy = 0\}$, podemos tomar $ \mu_{y}: \dfrac{A}{I}\to N$ determinada por $ \mu_{y}(\overline{1}) = y$. Se deja como ejercicio probar que esta asignación define la inversa.

    Volviendo al inicio, y denotando con $ \overline{\cdot }$ a las clases de $ \dfrac{A}{I}$ y con $ []$ a las de $ \dfrac{A}{J}$ sabemos que $ f = \mu_{[b]}: \overline{a} \mapsto [ab]$ donde $ [b] \in \{[c] \in \dfrac{A}{J} :\ I [c] = [0]\} = \{c+J:\ Ic \subseteq J\} = \dfrac{(J:I)}{J}$.
    
    Como conlusión, llegamos a que $ f = \mu_{[b]}$, donde $ [b] \in \dfrac{(J:I)}{J} (\iff b \in (J:I))$

    Vemos cuando es este $ \mu_{[b]}$ inyectivo. Lo será cuando
    $$ 
    \begin{aligned}
        \mu_{[b]} = (\overline{a}) = [0] &  \implies  & \overline{a} = \overline{0}\\ 
        [ab] = [0] & \implies & \overline{a} = \overline{0}\\ 
        \Updownarrow & & \Updownarrow\\ 
        ab \in J & \implies & a \in I \\ 
        a \in (J:b) & \implies & a \in I & \iff (J:b) \subseteq I
    \end{aligned}
    $$

    En conclusión, $ \mu_{[b]}$ intectivo sii $ (J:b)\subseteq I$

    Comprobarmos ahora cuando es $ \mu_{[b]}$ suprayectivo.

    $$ \operatorname{Im}(\mu_{[b]}) = \{[ab]:\ \overline{a} \in \dfrac{A}{I}\} = \{ab+J:\ a \in A\} = \dfrac{Ab+J}{J} $$

    Luego $ \mu_{[b]}$ es sobre $ \iff \dfrac{Ab+J}{J} = \dfrac{A}{J} \iff Ab+J = A \implies I (Ab+J) = I \implies I = Ib+ IJ \subseteq J+IJ = J$

\end{exercise}

\begin{exercise}
    \begin{flushright}
        \textbf{Apartado a)}
    \end{flushright}
    
    Basta con ver que $ L$ es cíclico de orden 3, luego $ ((0,6)) \cong \mathbb{Z}_{3}$
    
    $$ \dfrac{M}{K} = \dfrac{\mathbb{Z}_{3} \oplus \mathbb{Z}_{9}}{\mathbb{Z}_{3} \oplus 0} \cong \dfrac{\mathbb{Z}_{3}}{\mathbb{Z}_{3}} \oplus \dfrac{\mathbb{Z}_{9}}{0} \cong \mathbb{Z}_{9} $$
    $$ \dfrac{M}{L} = \dfrac{\mathbb{Z}_{3}\oplus \mathbb{Z}_{9}}{0 \oplus (6)} \cong \mathbb{Z}_{3} \oplus \dfrac{\mathbb{Z}_{9}}{(6)} \cong \mathbb{Z}_{3} \oplus \mathbb{Z}_{3} $$

    No son isomorfos porque uno es cíclio y el otro no.
    
    \begin{flushright}
        \textbf{Apartado b)}
    \end{flushright}

    $$ \dfrac{M}{K+L} = \dfrac{\mathbb{Z}_{3}\oplus \mathbb{Z}_{9}}{\mathbb{Z}_{3}\oplus (6)} = 0 \oplus \dfrac{\mathbb{Z}_{9}}{(6)} \cong \mathbb{Z}_{3} $$
    $$ \dfrac{M}{N} = \dfrac{\mathbb{Z}_{3}\oplus \mathbb{Z}_{9}}{0 \oplus \mathbb{Z}_{9}} =\cong \mathbb{Z}_{3} $$


    $$ K+L = \mathbb{Z}_{3} \oplus (6) \cong \mathbb{Z}_{3} \mathbb{Z}_{3} $$
    $$ N \cong \mathbb{Z}_{9} $$


\end{exercise}

\setcounter{ex}{9}

\begin{exercise}
    $ $

    Pongamos que $ N = (x_1,...,x_s)$ y $ \dfrac{M}{N}(\overline{y}_{1},...,\overline{y}_{t})$.
    
    Parece razonable comprobar que $ M = (x_1,...,x_s,y_1,...,y_t)$, donde los $ y_i$ son representantes arbitrarios de las clases $\overline{y}_{i} $.

    Sea $ m \in M \implies \overline{m} = m+N = \sum\limits_{j=1}^{t}b_j \overline{y}_{j}$ donde los $ b_j \in A$

    Pero $ \sum\limits_{j=1}^{t}b_j \overline{y}_{j} = \overline{\sum\limits_{}^{}b_j y_j} = m $. Por lo tanto:
    $$ m - \sum\limits_{j=1}^{t} b_j y_j \in N \implies m- \sum\limits_{j=1}^{t}b_jy_j = \sum\limits_{i=1}^{s} a_ix_i $$

    Con $ a_i \in A$. Por lo tanto:

    $ m = \sum\limits_{i=1}^{s}a_ix_i + \sum\limits_{j=1}^{t}b_jy_j$

\end{exercise}

\setcounter{ex}{11}

\begin{exercise}
    $ $

    Si $ \mathbb{Z}[\dfrac{1}{q}] = \{\dfrac{m}{q^{t}}:\ m \in Z,\ t\geq 0\}$ fuera finitamente generado, tomamos denominadores comunes y podemos expresar $ \mathbb{Z}[\dfrac{1}{q}]$ como:
    $$ Z[\dfrac{1}{q}] = \left( \dfrac{m_1}{q^{t}},..., \dfrac{m_n}{q^{t}} \right) $$ 

    Pero entonces $ \dfrac{1}{q^{t+1}}$ no se puede generar (o algo así era).
\end{exercise}


\end{document}
