\documentclass[openany]{book}
\usepackage[utf8]{inputenc}
\usepackage{verbatim}
\usepackage[hypertexnames=false]{hyperref}
\usepackage{amstext} 
\usepackage{array}   
\newcolumntype{C}{>{$}c<{$}} 


\input{structure}
\usepackage{geometry}
\geometry{
    top=3cm,
    bottom=3cm,
    left=3cm,
    right=3cm,
    headheight=14pt, 
    footskip=1.4cm,
    headsep=10pt,
}
\usepackage{graphicx}
\title{Apuntes de Álgebra Conmutativa}
\author{Paco Mora}
\date{\today}

\begin{document}

\maketitle

\chapter{Tema 1. Apuntes extra}

\begin{exercise}
    \textbf{Ejercicio Propuesto}

    Sea $ A = \mathbb{Z}_{n} $, con $ n  $ entero >1 y $ \overline{r} \in \mathbb{Z}_{n} $. Demostrar:
    \begin{itemize}
        \item $ \overline{r} $ cancelable $ \iff \overline{r}$ invertible $ \iff $ $ mcd(r,n)=1 $
        \item $ \overline{r} $ nilpotente $ \iff $ todos los divisores primos de $ n $ dividen a $ r $.
    \end{itemize}
\end{exercise}

La siguiente proposición generaliza el ejercicio anterior.

\begin{proposition}
    Sea $ A  $ un anillo finito y sea $ a \in A $. Entonces $ a  $ es cancelable sii es invertible.
\end{proposition}

\begin{demonstration}

    Definimos
    $$ \lambda_n: A \to A\hspace{5mm} \lambda_n(x)=ax\ \forall x \in A$$

    Es inyectiva, $ \lambda_n(x) = \lambda_n(y) \iff ax = ay \implies_{a\ cancel.} x=y $

    Por lo tanto, y como $ A $ es finito, $ \lambda_n $ es biyectiva y $ 1 \in Im(\lambda_n) \iff \exists b \in A\ |\ \lambda_n(b)=1 $
\end{demonstration}


\begin{proposition}
    $ A $ reducido $ \iff $ Nil($ A $) = $ \{\text{elem nilpotentes de}\ A\} = \{0\} $
\end{proposition}

\begin{demonstration}

    $ \ \implies$

    $ A $ reducido sii $ \forall a \in A $, $ a^2=0 \implies a=0 $

    $ \impliedby $

    Por reduc. al absurdo, supongamos $ b \in Nil(A) \setminus \{0\}  \implies \exists n >0 $ (mínimo) con $ b^{n}=0 \implies b^{n-1} \ne 0 $

    Pero entonces, $ (b^{n-1})^{2} = b^{2n-2}=0  $ y $ 2n-2\geq n  $ para $ n\geq 2 $, luego llegamos a una contradicción.
\end{demonstration}

\begin{exercise}
    \textbf{Ejercicio Propuesto}

    $ \mathbb{Z}_{n}$ es un anillo reducido $ \iff $ $ n $ es libre de cuadrados.

\end{exercise}\vspace{5mm}
\textbf{Demostración del 1.9(ii)}

\begin{demonstration}
    $ a/b $ y $ a/c \implies \exists b',c' \in D /\ ab'=b,\ ac'=c$\dots
    Sean ahora $ r,s \in D $ arbitrarios y veamos que $ a/rb+sc $
    $$ rb+rc=r(ab)+s(ac') = arb' =  asc' = a(rb'+sc') \implies a | rb+sc \implies \cancel{b}1 = \cancel{b}(dc) $$
\end{demonstration}

\end{document}