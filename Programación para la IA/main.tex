\documentclass[openany]{book}
\usepackage[utf8]{inputenc}
\usepackage{verbatim}
\usepackage[hypertexnames=false]{hyperref}
\usepackage{amstext} 
\usepackage{array}   
\newcolumntype{C}{>{$}c<{$}} 


\input{structure}
\usepackage{geometry}
\geometry{
    top=3cm,
    bottom=3cm,
    left=3cm,
    right=3cm,
    headheight=14pt, 
    footskip=1.4cm,
    headsep=10pt,
}
\usepackage{graphicx}
\title{Ejemplos PIA}
\author{Paco Mora}
\date{\today}

\begin{document}

\maketitle

\textbf{Sustituciones} 

$$ \lambda y.(yz)[y/z] $$
$$ \lambda a.(yz)[a/y][y/z] $$
$$ \lambda a.(az)[y/z] $$

$$ (\lambda y.x (\lambda x.x)z )[\lambda v.vy / z] $$
$$ \lambda a. (x(\lambda x.x)z) [a / y][(\lambda v.vy / z)] $$
$$ \lambda a. (x(\lambda x.x)z) [\lambda v.vy / z] $$
$$ \lambda a.x(\lambda x.x)(\lambda v.vy) $$


//Así sería paso a paso pero es un lío
$$ (\lambda y.(\lambda f.fx)y)[fy / x] $$
$$ \lambda a.((\lambda f.fx)y)[a /y][fy /x] $$
$$ \lambda a.((\lambda f.fx)a) [fy /x] $$
$$ \lambda a.((\lambda f.fx)[fy  /x])(a [fy /x]) $$
$$ \lambda a.\lambda b.(fx)[b /f][fy /x] $$
$$ \lambda a.(\lambda b.(bx)[fy (x)])a $$
$$ \lambda a.(\lambda b.b(fy))a $$

//Se puede hacer a troncho más fácil
$$ (\lambda y.(\lambda f.fx)y)[fy / x] $$
$$ (\lambda a.(\lambda b.bx)a)[fy / x] $$
$$ \lambda a.(\lambda b.b(fy))a $$

\textbf{Alfa, beta y eta reducción }

$$ (\lambda x.\lambda z.x+z)y \xrightarrow{\beta} \lambda z.y+z $$

$$ (\lambda x.\lambda z.x+z)z \xrightarrow{\beta} \lambda a.z+a $$

$$ (\lambda x.x (\lambda y.xyy)x)(\lambda z.\lambda w.z) \xrightarrow{\beta} (\lambda z.\lambda w. z)(\lambda y.(\lambda z.\lambda w.z)yy)(\lambda z.\lambda w.z) $$
$$ \xrightarrow{\beta} (\lambda z.\lambda w.)(\lambda  y.(\lambda w.y)y) (\lambda z.\lambda w.z) \xrightarrow{\beta} (\lambda  z.\lambda w.z)(\lambda y.y)(\lambda z.\lambda w .z) \xrightarrow{\beta} (\lambda w.(\lambda y.y) )(\lambda z.\lambda w.z) \xrightarrow{\beta} \lambda y.y$$
\vspace{5mm} Vemos ahora un ejemplo en el que hay que usar renombramiento:
$$ (\lambda ab. ab) (\lambda x.bx) c \xrightarrow{\beta} (\lambda z.(\lambda x.bx)z)c \xrightarrow{\beta} (\lambda z.bz)c \xrightarrow{\beta} bc $$

$$ (\lambda x.(\lambda y.x) y(\lambda z.z))(\lambda y.yz) $$
Vamos a ver el orden normal y el aplicativo. Empezamos por el normal:
$$ (\lambda y.y\lambda y.yz)y(\lambda z.z) \xrightarrow{\beta} (\lambda y.yz)(\lambda z.z) \xrightarrow{\beta}(\lambda z.z)z \xrightarrow{\beta} z $$
Y el aplicativo:
$$ (\lambda x.x(\lambda z.z))(\lambda y.yz) \xrightarrow{\beta} (\lambda y.yz)(\lambda z.z) \xrightarrow{\beta}(\lambda z.z)z \xrightarrow{\beta} z $$

$$ a(\lambda x. (\lambda y.y)a)((\lambda z.z)b) $$
Puede parecer que hay 3 $ \beta  $-redex, pero solo hay dos, la anterior expresión es equivalente a:
$$ (a(\lambda x. (\lambda y.y)a))((\lambda z.z)b) $$
Y las $ \beta $-redex son:
$$ (a(\lambda x. (\lambda y.y\underline{)}a))((\lambda z.z\underline{)}b) \xrightarrow{\beta} a(\lambda x.a)((\lambda z.z)b) \xrightarrow{\beta} a(\lambda x.a)b $$
Ambos órdenes llevan al mismo resultado.

\subsubsection*{Resolución Ej 11 Hoja 2}

Quitando los paréntesis que no hacen falta queda:
$$ (\lambda x.\lambda y.sum y((\lambda z.mul x z)3))7\  5 $$ 
Utilizando orden normal
$$ \xrightarrow{\beta} (\lambda y. sum\ y((\lambda z. mul\ 7\ z)3))5 \xrightarrow{\beta} sum\ 5((\lambda z.mul\ 7\ z)3) \xrightarrow{\beta} sum\ 5 (mul\ 7\ 3) $$
$$ \xrightarrow{\delta} sum\ 5\ 21 \xrightarrow{\delta}26 $$
Vemos ahora el orden aplicativo:
$$ \xrightarrow{\beta} (\lambda x.\lambda y.sum\ y(mul\ x\ 3))7\ 5\xrightarrow{\beta} (\lambda y.sum\ y(mul\ 7\ 3))5 \xrightarrow{\delta} (\lambda  y.sum\ y 21)5 \xrightarrow{\beta} sum\ 5\ 21 \xrightarrow{\delta} 26 $$


\end{document}